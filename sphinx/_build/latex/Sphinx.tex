% Generated by Sphinx.
\def\sphinxdocclass{report}
\documentclass[a4paper,11pt,french]{rtdsphinxmanual}
\usepackage[utf8]{inputenc}
\DeclareUnicodeCharacter{00A0}{\nobreakspace}
\usepackage{cmap}
\usepackage[T1]{fontenc}
\usepackage[cyr]{aeguill}
\usepackage[francais]{babel}
\usepackage{times}
\usepackage[Bjornstrup]{fncychap}
\usepackage{longtable}
\usepackage{rtdsphinx}
\usepackage{multirow}

\addto\captionsfrench{\renewcommand{\figurename}{Fig. }}
\addto\captionsfrench{\renewcommand{\tablename}{Tableau }}
\floatname{literal-block}{Code source }



\title{Sphinx}
\date{04 septembre 2015}
\release{1.0}
\author{Madeline Veyrenc}
\newcommand{\sphinxlogo}{\includegraphics[height=3cm]{logo_latex.png}\par}
\renewcommand{\releasename}{Version}
\pagestyle{fancy}
\newcommand{\rtdbackgroundimage}{\includegraphics[width=19cm]{latex_background_image.png}}
\newcommand{\rtdheaderbackgroundimage}{\includegraphics[width=19cm]{latex_header_image.png}}
\newcommand{\rtdsubtitle}{DOCUMENTATION}
\newcommand{\rtdreference}{{OWSI-PHP-INT}}
\newcommand{\rtdcopyright}{{Toute reproduction interdite sauf autorisation}}
\newcommand{\rtdfooterimage}{\includegraphics[width=7cm]{latex_address.png}}
\makeindex[intoc]

\makeatletter
\def\PYG@reset{\let\PYG@it=\relax \let\PYG@bf=\relax%
    \let\PYG@ul=\relax \let\PYG@tc=\relax%
    \let\PYG@bc=\relax \let\PYG@ff=\relax}
\def\PYG@tok#1{\csname PYG@tok@#1\endcsname}
\def\PYG@toks#1+{\ifx\relax#1\empty\else%
    \PYG@tok{#1}\expandafter\PYG@toks\fi}
\def\PYG@do#1{\PYG@bc{\PYG@tc{\PYG@ul{%
    \PYG@it{\PYG@bf{\PYG@ff{#1}}}}}}}
\def\PYG#1#2{\PYG@reset\PYG@toks#1+\relax+\PYG@do{#2}}

\expandafter\def\csname PYG@tok@gd\endcsname{\def\PYG@tc##1{\textcolor[rgb]{0.63,0.00,0.00}{##1}}}
\expandafter\def\csname PYG@tok@gu\endcsname{\let\PYG@bf=\textbf\def\PYG@tc##1{\textcolor[rgb]{0.50,0.00,0.50}{##1}}}
\expandafter\def\csname PYG@tok@gt\endcsname{\def\PYG@tc##1{\textcolor[rgb]{0.00,0.27,0.87}{##1}}}
\expandafter\def\csname PYG@tok@gs\endcsname{\let\PYG@bf=\textbf}
\expandafter\def\csname PYG@tok@gr\endcsname{\def\PYG@tc##1{\textcolor[rgb]{1.00,0.00,0.00}{##1}}}
\expandafter\def\csname PYG@tok@cm\endcsname{\let\PYG@it=\textit\def\PYG@tc##1{\textcolor[rgb]{0.25,0.50,0.56}{##1}}}
\expandafter\def\csname PYG@tok@vg\endcsname{\def\PYG@tc##1{\textcolor[rgb]{0.73,0.38,0.84}{##1}}}
\expandafter\def\csname PYG@tok@m\endcsname{\def\PYG@tc##1{\textcolor[rgb]{0.13,0.50,0.31}{##1}}}
\expandafter\def\csname PYG@tok@mh\endcsname{\def\PYG@tc##1{\textcolor[rgb]{0.13,0.50,0.31}{##1}}}
\expandafter\def\csname PYG@tok@cs\endcsname{\def\PYG@tc##1{\textcolor[rgb]{0.25,0.50,0.56}{##1}}\def\PYG@bc##1{\setlength{\fboxsep}{0pt}\colorbox[rgb]{1.00,0.94,0.94}{\strut ##1}}}
\expandafter\def\csname PYG@tok@ge\endcsname{\let\PYG@it=\textit}
\expandafter\def\csname PYG@tok@vc\endcsname{\def\PYG@tc##1{\textcolor[rgb]{0.73,0.38,0.84}{##1}}}
\expandafter\def\csname PYG@tok@il\endcsname{\def\PYG@tc##1{\textcolor[rgb]{0.13,0.50,0.31}{##1}}}
\expandafter\def\csname PYG@tok@go\endcsname{\def\PYG@tc##1{\textcolor[rgb]{0.20,0.20,0.20}{##1}}}
\expandafter\def\csname PYG@tok@cp\endcsname{\def\PYG@tc##1{\textcolor[rgb]{0.00,0.44,0.13}{##1}}}
\expandafter\def\csname PYG@tok@gi\endcsname{\def\PYG@tc##1{\textcolor[rgb]{0.00,0.63,0.00}{##1}}}
\expandafter\def\csname PYG@tok@gh\endcsname{\let\PYG@bf=\textbf\def\PYG@tc##1{\textcolor[rgb]{0.00,0.00,0.50}{##1}}}
\expandafter\def\csname PYG@tok@ni\endcsname{\let\PYG@bf=\textbf\def\PYG@tc##1{\textcolor[rgb]{0.84,0.33,0.22}{##1}}}
\expandafter\def\csname PYG@tok@nl\endcsname{\let\PYG@bf=\textbf\def\PYG@tc##1{\textcolor[rgb]{0.00,0.13,0.44}{##1}}}
\expandafter\def\csname PYG@tok@nn\endcsname{\let\PYG@bf=\textbf\def\PYG@tc##1{\textcolor[rgb]{0.05,0.52,0.71}{##1}}}
\expandafter\def\csname PYG@tok@no\endcsname{\def\PYG@tc##1{\textcolor[rgb]{0.38,0.68,0.84}{##1}}}
\expandafter\def\csname PYG@tok@na\endcsname{\def\PYG@tc##1{\textcolor[rgb]{0.25,0.44,0.63}{##1}}}
\expandafter\def\csname PYG@tok@nb\endcsname{\def\PYG@tc##1{\textcolor[rgb]{0.00,0.44,0.13}{##1}}}
\expandafter\def\csname PYG@tok@nc\endcsname{\let\PYG@bf=\textbf\def\PYG@tc##1{\textcolor[rgb]{0.05,0.52,0.71}{##1}}}
\expandafter\def\csname PYG@tok@nd\endcsname{\let\PYG@bf=\textbf\def\PYG@tc##1{\textcolor[rgb]{0.33,0.33,0.33}{##1}}}
\expandafter\def\csname PYG@tok@ne\endcsname{\def\PYG@tc##1{\textcolor[rgb]{0.00,0.44,0.13}{##1}}}
\expandafter\def\csname PYG@tok@nf\endcsname{\def\PYG@tc##1{\textcolor[rgb]{0.02,0.16,0.49}{##1}}}
\expandafter\def\csname PYG@tok@si\endcsname{\let\PYG@it=\textit\def\PYG@tc##1{\textcolor[rgb]{0.44,0.63,0.82}{##1}}}
\expandafter\def\csname PYG@tok@s2\endcsname{\def\PYG@tc##1{\textcolor[rgb]{0.25,0.44,0.63}{##1}}}
\expandafter\def\csname PYG@tok@vi\endcsname{\def\PYG@tc##1{\textcolor[rgb]{0.73,0.38,0.84}{##1}}}
\expandafter\def\csname PYG@tok@nt\endcsname{\let\PYG@bf=\textbf\def\PYG@tc##1{\textcolor[rgb]{0.02,0.16,0.45}{##1}}}
\expandafter\def\csname PYG@tok@nv\endcsname{\def\PYG@tc##1{\textcolor[rgb]{0.73,0.38,0.84}{##1}}}
\expandafter\def\csname PYG@tok@s1\endcsname{\def\PYG@tc##1{\textcolor[rgb]{0.25,0.44,0.63}{##1}}}
\expandafter\def\csname PYG@tok@gp\endcsname{\let\PYG@bf=\textbf\def\PYG@tc##1{\textcolor[rgb]{0.78,0.36,0.04}{##1}}}
\expandafter\def\csname PYG@tok@sh\endcsname{\def\PYG@tc##1{\textcolor[rgb]{0.25,0.44,0.63}{##1}}}
\expandafter\def\csname PYG@tok@ow\endcsname{\let\PYG@bf=\textbf\def\PYG@tc##1{\textcolor[rgb]{0.00,0.44,0.13}{##1}}}
\expandafter\def\csname PYG@tok@sx\endcsname{\def\PYG@tc##1{\textcolor[rgb]{0.78,0.36,0.04}{##1}}}
\expandafter\def\csname PYG@tok@bp\endcsname{\def\PYG@tc##1{\textcolor[rgb]{0.00,0.44,0.13}{##1}}}
\expandafter\def\csname PYG@tok@c1\endcsname{\let\PYG@it=\textit\def\PYG@tc##1{\textcolor[rgb]{0.25,0.50,0.56}{##1}}}
\expandafter\def\csname PYG@tok@kc\endcsname{\let\PYG@bf=\textbf\def\PYG@tc##1{\textcolor[rgb]{0.00,0.44,0.13}{##1}}}
\expandafter\def\csname PYG@tok@c\endcsname{\let\PYG@it=\textit\def\PYG@tc##1{\textcolor[rgb]{0.25,0.50,0.56}{##1}}}
\expandafter\def\csname PYG@tok@mf\endcsname{\def\PYG@tc##1{\textcolor[rgb]{0.13,0.50,0.31}{##1}}}
\expandafter\def\csname PYG@tok@err\endcsname{\def\PYG@bc##1{\setlength{\fboxsep}{0pt}\fcolorbox[rgb]{1.00,0.00,0.00}{1,1,1}{\strut ##1}}}
\expandafter\def\csname PYG@tok@mb\endcsname{\def\PYG@tc##1{\textcolor[rgb]{0.13,0.50,0.31}{##1}}}
\expandafter\def\csname PYG@tok@ss\endcsname{\def\PYG@tc##1{\textcolor[rgb]{0.32,0.47,0.09}{##1}}}
\expandafter\def\csname PYG@tok@sr\endcsname{\def\PYG@tc##1{\textcolor[rgb]{0.14,0.33,0.53}{##1}}}
\expandafter\def\csname PYG@tok@mo\endcsname{\def\PYG@tc##1{\textcolor[rgb]{0.13,0.50,0.31}{##1}}}
\expandafter\def\csname PYG@tok@kd\endcsname{\let\PYG@bf=\textbf\def\PYG@tc##1{\textcolor[rgb]{0.00,0.44,0.13}{##1}}}
\expandafter\def\csname PYG@tok@mi\endcsname{\def\PYG@tc##1{\textcolor[rgb]{0.13,0.50,0.31}{##1}}}
\expandafter\def\csname PYG@tok@kn\endcsname{\let\PYG@bf=\textbf\def\PYG@tc##1{\textcolor[rgb]{0.00,0.44,0.13}{##1}}}
\expandafter\def\csname PYG@tok@o\endcsname{\def\PYG@tc##1{\textcolor[rgb]{0.40,0.40,0.40}{##1}}}
\expandafter\def\csname PYG@tok@kr\endcsname{\let\PYG@bf=\textbf\def\PYG@tc##1{\textcolor[rgb]{0.00,0.44,0.13}{##1}}}
\expandafter\def\csname PYG@tok@s\endcsname{\def\PYG@tc##1{\textcolor[rgb]{0.25,0.44,0.63}{##1}}}
\expandafter\def\csname PYG@tok@kp\endcsname{\def\PYG@tc##1{\textcolor[rgb]{0.00,0.44,0.13}{##1}}}
\expandafter\def\csname PYG@tok@w\endcsname{\def\PYG@tc##1{\textcolor[rgb]{0.73,0.73,0.73}{##1}}}
\expandafter\def\csname PYG@tok@kt\endcsname{\def\PYG@tc##1{\textcolor[rgb]{0.56,0.13,0.00}{##1}}}
\expandafter\def\csname PYG@tok@sc\endcsname{\def\PYG@tc##1{\textcolor[rgb]{0.25,0.44,0.63}{##1}}}
\expandafter\def\csname PYG@tok@sb\endcsname{\def\PYG@tc##1{\textcolor[rgb]{0.25,0.44,0.63}{##1}}}
\expandafter\def\csname PYG@tok@k\endcsname{\let\PYG@bf=\textbf\def\PYG@tc##1{\textcolor[rgb]{0.00,0.44,0.13}{##1}}}
\expandafter\def\csname PYG@tok@se\endcsname{\let\PYG@bf=\textbf\def\PYG@tc##1{\textcolor[rgb]{0.25,0.44,0.63}{##1}}}
\expandafter\def\csname PYG@tok@sd\endcsname{\let\PYG@it=\textit\def\PYG@tc##1{\textcolor[rgb]{0.25,0.44,0.63}{##1}}}

\def\PYGZbs{\char`\\}
\def\PYGZus{\char`\_}
\def\PYGZob{\char`\{}
\def\PYGZcb{\char`\}}
\def\PYGZca{\char`\^}
\def\PYGZam{\char`\&}
\def\PYGZlt{\char`\<}
\def\PYGZgt{\char`\>}
\def\PYGZsh{\char`\#}
\def\PYGZpc{\char`\%}
\def\PYGZdl{\char`\$}
\def\PYGZhy{\char`\-}
\def\PYGZsq{\char`\'}
\def\PYGZdq{\char`\"}
\def\PYGZti{\char`\~}
% for compatibility with earlier versions
\def\PYGZat{@}
\def\PYGZlb{[}
\def\PYGZrb{]}
\makeatother

\renewcommand\PYGZsq{\textquotesingle}

\begin{document}

\maketitle
\tableofcontents
\phantomsection\label{index::doc}



\chapter{Installation de Sphinx}
\label{installation::doc}\label{installation:sphinx}\label{installation:installation-de-sphinx}

\section{Installation des pré-requis}
\label{installation:installation-des-pre-requis}
\begin{Verbatim}[commandchars=\\\{\}]
\PYG{g+go}{sudo apt\PYGZhy{}get install python\PYGZhy{}pip python\PYGZhy{}dev build\PYGZhy{}essential}
\PYG{g+go}{sudo apt\PYGZhy{}get install graphviz librsvg2\PYGZhy{}bin plotutils}
\PYG{g+go}{sudo pip install \PYGZhy{}\PYGZhy{}upgrade pip}
\PYG{g+go}{sudo pip install \PYGZhy{}\PYGZhy{}upgrade virtualenv}
\PYG{g+go}{sudo pip install sphinx}
\PYG{g+go}{sudo pip install watchdog}
\end{Verbatim}


\section{Installation des thèmes}
\label{installation:installation-des-themes}

\subsection{reveal.js}
\label{installation:reveal-js}
Pour les présentations.

\begin{Verbatim}[commandchars=\\\{\}]
\PYG{g+go}{git submodule add \PYGZbs{}}
\PYG{g+go}{    https://github.com/Open\PYGZhy{}Wide/sphinxjp.themes.revealjs.git \PYGZbs{}}
\PYG{g+go}{    docs/\PYGZus{}themes/sphinxjp.themes.revealjs}
\PYG{g+go}{cd docs/\PYGZus{}themes/sphinxjp.themes.revealjs}
\PYG{g+go}{sudo python setup.py install}
\end{Verbatim}

\begin{Verbatim}[commandchars=\\\{\}]
\PYG{c}{\PYGZsh{} docs/sphinx\PYGZus{}doc/config.py}

\PYG{n}{extensions} \PYG{o}{=} \PYG{p}{[}\PYG{l+s}{\PYGZsq{}}\PYG{l+s}{sphinxjp.themes.revealjs}\PYG{l+s}{\PYGZsq{}}\PYG{p}{]}

\PYG{n}{exclude\PYGZus{}patterns} \PYG{o}{=} \PYG{p}{[}\PYG{l+s}{\PYGZsq{}}\PYG{l+s}{\PYGZus{}build}\PYG{l+s}{\PYGZsq{}}\PYG{p}{]}

\PYG{n}{html\PYGZus{}theme} \PYG{o}{=} \PYG{l+s}{\PYGZsq{}}\PYG{l+s}{revealjs}\PYG{l+s}{\PYGZsq{}}
\PYG{n}{html\PYGZus{}use\PYGZus{}index} \PYG{o}{=} \PYG{n+nb+bp}{False}
\PYG{n}{html\PYGZus{}theme\PYGZus{}options} \PYG{o}{=} \PYG{p}{\PYGZob{}}
    \PYG{l+s}{\PYGZdq{}}\PYG{l+s}{lang}\PYG{l+s}{\PYGZdq{}}\PYG{p}{:} \PYG{l+s}{\PYGZdq{}}\PYG{l+s}{fr}\PYG{l+s}{\PYGZdq{}}\PYG{p}{,}
    \PYG{l+s}{\PYGZdq{}}\PYG{l+s}{theme}\PYG{l+s}{\PYGZdq{}}\PYG{p}{:} \PYG{l+s}{\PYGZdq{}}\PYG{l+s}{openwide}\PYG{l+s}{\PYGZdq{}}\PYG{p}{,}
    \PYG{l+s}{\PYGZdq{}}\PYG{l+s}{slide\PYGZus{}number}\PYG{l+s}{\PYGZdq{}}\PYG{p}{:} \PYG{n+nb+bp}{True}\PYG{p}{,}
\PYG{p}{\PYGZcb{}}
\end{Verbatim}


\subsection{Read the Docs}
\label{installation:read-the-docs}
Pour les autres documents.

\begin{Verbatim}[commandchars=\\\{\}]
\PYG{g+go}{git submodule add \PYGZbs{}}
\PYG{g+go}{    https://github.com/Open\PYGZhy{}Wide/sphinx\PYGZus{}rtd\PYGZus{}theme.git \PYGZbs{}}
\PYG{g+go}{    docs/\PYGZus{}themes/sphinx\PYGZus{}rtd\PYGZus{}theme}
\PYG{g+go}{cd docs/\PYGZus{}themes/sphinx\PYGZus{}rtd\PYGZus{}theme}
\PYG{g+go}{sudo python setup.py install}
\end{Verbatim}

\begin{Verbatim}[commandchars=\\\{\}]
\PYG{c}{\PYGZsh{} docs/sphinx\PYGZus{}doc/config.py}

\PYG{n}{extensions} \PYG{o}{=} \PYG{p}{[}\PYG{l+s}{\PYGZsq{}}\PYG{l+s}{sphinx\PYGZus{}rtd\PYGZus{}theme}\PYG{l+s}{\PYGZsq{}}\PYG{p}{]}

\PYG{n}{exclude\PYGZus{}patterns} \PYG{o}{=} \PYG{p}{[}\PYG{l+s}{\PYGZsq{}}\PYG{l+s}{\PYGZus{}build}\PYG{l+s}{\PYGZsq{}}\PYG{p}{]}

\PYG{n}{html\PYGZus{}theme} \PYG{o}{=} \PYG{l+s}{\PYGZsq{}}\PYG{l+s}{sphinx\PYGZus{}rtd\PYGZus{}theme}\PYG{l+s}{\PYGZsq{}}
\PYG{n}{html\PYGZus{}theme\PYGZus{}options} \PYG{o}{=} \PYG{p}{\PYGZob{}}
    \PYG{l+s}{\PYGZsq{}}\PYG{l+s}{subtitle}\PYG{l+s}{\PYGZsq{}}\PYG{p}{:} \PYG{n}{subtitle}\PYG{p}{,}
    \PYG{l+s}{\PYGZsq{}}\PYG{l+s}{reference}\PYG{l+s}{\PYGZsq{}}\PYG{p}{:} \PYG{n}{reference}
\PYG{p}{\PYGZcb{}}
\end{Verbatim}


\section{Pour construire des PDF}
\label{installation:pour-construire-des-pdf}
\begin{Verbatim}[commandchars=\\\{\}]
sudo apt\PYGZhy{}get install texlive\PYGZhy{}latex\PYGZhy{}recommended texlive\PYGZhy{}latex\PYGZhy{}base \PYG{l+s+se}{\PYGZbs{}}
    texlive\PYGZhy{}lang\PYGZhy{}french fonts\PYGZhy{}linuxlibertine ttf\PYGZhy{}linux\PYGZhy{}libertine \PYG{l+s+se}{\PYGZbs{}}
    python\PYGZhy{}genshi python\PYGZhy{}lxml python\PYGZhy{}jinja2\PYGZhy{}doc ttf\PYGZhy{}bitstream\PYGZhy{}vera \PYG{l+s+se}{\PYGZbs{}}
    jsmath libjs\PYGZhy{}mathjax dvipng texlive\PYGZhy{}latex\PYGZhy{}extra \PYG{l+s+se}{\PYGZbs{}}
    texlive\PYGZhy{}fonts\PYGZhy{}recommended texlive\PYGZhy{}lang\PYGZhy{}cyrillic
\end{Verbatim}

\begin{Verbatim}[commandchars=\\\{\}]
\PYG{c}{\PYGZsh{} docs/sphinx\PYGZus{}doc/config.py}

\PYG{n}{extensions} \PYG{o}{=} \PYG{p}{[}\PYG{l+s}{\PYGZsq{}}\PYG{l+s}{sphinx\PYGZus{}rtd\PYGZus{}theme}\PYG{l+s}{\PYGZsq{}}\PYG{p}{]}
\PYG{n}{latex\PYGZus{}elements} \PYG{o}{=} \PYG{p}{\PYGZob{}}
    \PYG{l+s}{\PYGZsq{}}\PYG{l+s}{pointsize}\PYG{l+s}{\PYGZsq{}}\PYG{p}{:} \PYG{l+s}{\PYGZsq{}}\PYG{l+s}{12pt}\PYG{l+s}{\PYGZsq{}}
\PYG{p}{\PYGZcb{}}
\end{Verbatim}


\chapter{Construction du document}
\label{construction_document::doc}\label{construction_document:construction-du-document}
Placez vous à la racine de votre document.

\begin{Verbatim}[commandchars=\\\{\}]
make html \PYG{c}{\PYGZsh{} construction de la version HTMl}
make latexpdf \PYG{c}{\PYGZsh{} construction du PDF}
make clean \PYG{c}{\PYGZsh{} nettoyage du répertoire de build}
\end{Verbatim}

Vous pouvez installez \code{watchdog} pour surveiller les modifications de fichier et rebuilder votre document quand c'est nécessaire :

\begin{Verbatim}[commandchars=\\\{\}]
sudo pip install watchdog
watchmedo shell\PYGZhy{}command \PYGZhy{}\PYGZhy{}recursive \PYGZhy{}\PYGZhy{}patterns\PYG{o}{=}\PYG{l+s+s2}{\PYGZdq{}*.rst\PYGZdq{}} \PYG{l+s+se}{\PYGZbs{}}
    \PYGZhy{}\PYGZhy{}wait \PYGZhy{}\PYGZhy{}command\PYG{o}{=}\PYG{l+s+s1}{\PYGZsq{}make html\PYGZsq{}}
\end{Verbatim}


\chapter{Création d'un document}
\label{creation_document:creation-d-un-document}\label{creation_document::doc}

\section{Par copie}
\label{creation_document:par-copie}
Copiez le répertoire \code{sphinx\_demo\_doc} ou \code{sphinx\_demo\_slides} puis faites un rechercher/remplacer sur les éléments suivants :
\begin{itemize}
\item {} 
\code{Demo doc Documentation} =\textgreater{} ex : \code{Symfony Documentation}

\item {} 
\code{Demo doc} =\textgreater{} ex : \code{Symfony}

\item {} 
\code{Demodoc} =\textgreater{} ex : \code{Symfonydoc}

\end{itemize}


\section{En ligne de commande}
\label{creation_document:en-ligne-de-commande}
\begin{Verbatim}[commandchars=\\\{\}]
sphinx\PYGZhy{}quickstart
\end{Verbatim}

Après avoir lancé le \emph{quickstart}, éditez le fichier \code{conf.py} pour affecter le bon thème :

\begin{Verbatim}[commandchars=\\\{\}]
\PYG{n}{extensions} \PYG{o}{=} \PYG{p}{[}
    \PYG{c}{\PYGZsh{} ...}
    \PYG{l+s}{\PYGZsq{}}\PYG{l+s}{sphinx\PYGZus{}rtd\PYGZus{}theme}\PYG{l+s}{\PYGZsq{}}\PYG{p}{,}
\PYG{p}{]}
\PYG{c}{\PYGZsh{} ...}
\PYG{n}{html\PYGZus{}theme} \PYG{o}{=} \PYG{l+s}{\PYGZsq{}}\PYG{l+s}{sphinx\PYGZus{}rtd\PYGZus{}theme}\PYG{l+s}{\PYGZsq{}}
\end{Verbatim}


\chapter{Références}
\label{references:references}\label{references::doc}\begin{itemize}
\item {} 
This doc is a fork of \href{http://aert-notes-dev.readthedocs.org/en/latest/content/rest/}{Cristoph Reller reST Memo}
adapted according to my needs, they have diverged now, but it still
inherit from some content and layout.

\item {} 
\href{http://sphinx.pocoo.org/latest/contents.html}{Sphinx documentation}

\item {} 
Sphinx \href{http://sphinx.pocoo.org/latest/rest.html}{reStructuredText Primer}

\item {} 
\href{http://packages.python.org/an\_example\_pypi\_project/sphinx.html}{Documenting Your Project Using
Sphinx}
from \href{http://packages.python.org/an\_example\_pypi\_project/}{an example pypi project’s}

\item {} 
Thomas Cokelaer \href{http://openalea.gforge.inria.fr/doc/openalea/doc/\_build/html/source/sphinx/sphinx.html}{Openalea project: How to use sphinx ?}
and the newer
\href{http://thomas-cokelaer.info/tutorials/sphinx/contents.html}{Sphinx and RST syntax guide}.

\item {} 
The \href{http://docutils.sourceforge.net/docs/}{ReStructuredText Documentation}
\begin{itemize}
\item {} 
\href{http://docutils.sourceforge.net/docs/user/rst/quickstart.html}{Docutil reStructuredText Primer}
you may prefer the python the \emph{Sphinx} nicely formated
documentation cited above, also available \emph{with a distinct layout} as
\href{http://docs.python.org/dev/documenting/rest.html}{docs.python: reStructuredText Primer}

\item {} 
\href{http://docutils.sourceforge.net/docs/ref/rst/quick.html}{Quick reStructuredText}

\item {} 
\href{http://docutils.sourceforge.net/docs/ref/rst/restructuredtext.html}{reStructuredText Markup Specification}

\item {} 
\href{http://docutils.sourceforge.net/docs/ref/rst/directives.html}{reST Directives}

\item {} 
\href{http://docutils.sourceforge.net/docs/ref/rst/roles.html}{Interpreted Text Roles}

\item {} 
\href{http://docutils.sourceforge.net/docs/user/rst/demo.html}{ReStructuredText Demonstration}

\end{itemize}

\item {} 
\href{http://docutils.sourceforge.net/docs/user/emacs.html}{Emacs Support for reStructuredText}

\item {} 
\href{http://docs.python.org/devguide/documenting.html}{Documenting Python}
in the \href{http://docs.python.org/devguide/}{Python Developer’s Guide}

\item {} 
\href{http://matplotlib.sourceforge.net/sampledoc/}{sampledoc tutorial}
from \href{http://matplotlib.sourceforge.neti/}{matplotlib}
\emph{a python 2D plotting library}.

\item {} 
\href{http://code.google.com/p/rst2pdf/}{rst2pdf} is a
tool for transforming reStructuredText to PDF using ReportLab.
It supports Sphinx formatting.

\item {} 
\href{http://epydoc.sourceforge.net/manual-othermarkup.html}{Epydoc reST markup}

\end{itemize}


\section{How to write docstrings.}
\label{references:how-to-write-docstrings}\begin{itemize}
\item {} 
Look at examples in \href{http://sphinx.pocoo.org/examples.html}{Official list of projects using Sphinx}

\item {} 
The last parts of \href{http://docs.python.org/devguide/documenting.html}{Documenting Python}:
\href{http://packages.python.org/an\_example\_pypi\_project/sphinx.html\#function-definitions}{function definitions}
and \href{http://packages.python.org/an\_example\_pypi\_project/sphinx.html\#full-code-example}{Full Code Example}

\item {} 
\href{http://openalea.gforge.inria.fr/wiki/doku.php?id=documentation:doctests:how\_to\_document\_python\_api}{OpenAlea}
has a nice \href{http://openalea.gforge.inria.fr/wiki/doku.php?id=documentation:doctests:sphinx\_proposal\#filling\_the\_docstring}{comparaison of three ways of filling the docstring}.
The source is  \href{https://gforge.inria.fr/scm/viewvc.php/trunk/doc/source/sphinx/template.py?view=markup\&root=openalea}{template.py}

\item {} 
Sources of
\href{https://github.com/mongodb/mongo-python-driver}{mongo python driver}
are also a good example

\end{itemize}


\section{Extending Sphinx.}
\label{references:extending-sphinx}\begin{itemize}
\item {} 
\href{http://sphinx.pocoo.org/ext/tutorial.html}{Sphinx Tutorial: Writing a simple extension}

\item {} 
\href{http://www.doughellmann.com/articles/how-tos/sphinx-custom-roles/index.html}{Defining Custom Roles in Sphinx}
a  \href{http://blog.doughellmann.com/search/label/sphinx}{Sphinx blog post by Doug Hellmann}

\item {} 
\href{http://docutils.sourceforge.net/docs/ref/rst/rst-roles.html}{Creating Interpreted Text Roles}
from docutils project.

\item {} 
\href{http://docutils.sourceforge.net/docs/ref/rst/rst-directives.html}{Creating reStructuredText Directives}
from docutils project.

\end{itemize}


\chapter{Instructions de base}
\label{rtd/instruction_base::doc}\label{rtd/instruction_base:instructions-de-base}

\section{Les directives}
\label{rtd/instruction_base/base:les-directives}\label{rtd/instruction_base/base::doc}
\index{directive}
\begin{Verbatim}[commandchars=\\\{\}]
\PYG{c+cp}{.. \PYGZlt{}name\PYGZgt{}:: \PYGZlt{}arguments\PYGZgt{}}
\PYG{c+cp}{    :\PYGZlt{}option\PYGZgt{}: \PYGZlt{}option values\PYGZgt{}}

\PYG{c+cp}{    content}
\end{Verbatim}


\section{Les rôles}
\label{rtd/instruction_base/base:les-roles}
\index{role}
\begin{Verbatim}[commandchars=\\\{\}]
:\PYG{n+nt}{\PYGZlt{}name\PYGZgt{}}:\PYG{n+nv}{{}`\PYGZlt{}argument\PYGZgt{}{}`}
\end{Verbatim}


\subsection{Customisation de rôle}
\label{rtd/instruction_base/base:customisation-de-role}
\index{directive!role}
La directive \code{role} permet de créer des rôles dynamiquement qui seront ensuite interprétés par le parser.

Pour déclarer un rôle

\begin{Verbatim}[commandchars=\\\{\}]
\PYG{p}{..} \PYG{o+ow}{role}\PYG{p}{::} custom
\end{Verbatim}

Ensuite on l'utilise ainsi

\begin{Verbatim}[commandchars=\\\{\}]
An example of using \PYG{n+na}{:custom:}\PYG{n+nv}{{}`interpreted text{}`}
\end{Verbatim}


\section{Avertissements}
\label{rtd/instruction_base/avertissements::doc}\label{rtd/instruction_base/avertissements:avertissements}
\index{avertissements}

\subsection{Spécifiques}
\label{rtd/instruction_base/avertissements:specifiques}\label{rtd/instruction_base/avertissements:index-0}
\begin{Verbatim}[commandchars=\\\{\}]
\PYG{p}{..} \PYG{o+ow}{warning}\PYG{p}{::}
    Equations within a note
    \PYG{n+na}{:math:}\PYG{n+nv}{{}`G\PYGZus{}\PYGZob{}\PYGZbs{}mu\PYGZbs{}nu\PYGZcb{} = 8 \PYGZbs{}pi G (T\PYGZus{}\PYGZob{}\PYGZbs{}mu\PYGZbs{}nu\PYGZcb{}  + \PYGZbs{}rho\PYGZus{}\PYGZbs{}Lambda g\PYGZus{}\PYGZob{}\PYGZbs{}mu\PYGZbs{}nu\PYGZcb{}){}`}.
\end{Verbatim}

\index{avertissements!attention}\index{directive!attention}\begin{itemize}
\item {} 
attention
\begin{quote}

\begin{notice}{attention}{Attention:}
Equations within a note
\(G_{\mu\nu} = 8 \pi G (T_{\mu\nu}  + \rho_\Lambda g_{\mu\nu})\).
\end{notice}
\end{quote}

\end{itemize}

\index{avertissements!caution}\index{directive!caution}\begin{itemize}
\item {} 
caution
\begin{quote}

\begin{notice}{caution}{Prudence:}
Equations within a note
\(G_{\mu\nu} = 8 \pi G (T_{\mu\nu}  + \rho_\Lambda g_{\mu\nu})\).
\end{notice}
\end{quote}

\end{itemize}

\index{avertissements!danger}\index{directive!danger}\begin{itemize}
\item {} 
danger
\begin{quote}

\begin{notice}{danger}{Danger:}
Equations within a note
\(G_{\mu\nu} = 8 \pi G (T_{\mu\nu}  + \rho_\Lambda g_{\mu\nu})\).
\end{notice}
\end{quote}

\end{itemize}

\index{avertissements!error}\index{directive!error}\begin{itemize}
\item {} 
error
\begin{quote}

\begin{notice}{error}{Erreur:}
Equations within a note
\(G_{\mu\nu} = 8 \pi G (T_{\mu\nu}  + \rho_\Lambda g_{\mu\nu})\).
\end{notice}
\end{quote}

\end{itemize}

\index{avertissements!hint}\index{directive!hint}\begin{itemize}
\item {} 
hint
\begin{quote}

\begin{notice}{hint}{Indice:}
Equations within a note
\(G_{\mu\nu} = 8 \pi G (T_{\mu\nu}  + \rho_\Lambda g_{\mu\nu})\).
\end{notice}
\end{quote}

\end{itemize}

\index{avertissements!important}\index{directive!important}\begin{itemize}
\item {} 
important
\begin{quote}

\begin{notice}{important}{Important:}
Equations within a note
\(G_{\mu\nu} = 8 \pi G (T_{\mu\nu}  + \rho_\Lambda g_{\mu\nu})\).
\end{notice}
\end{quote}

\end{itemize}

\index{avertissements!note}\index{directive!note}\begin{itemize}
\item {} 
note
\begin{quote}

\begin{notice}{note}{Note:}
Equations within a note
\(G_{\mu\nu} = 8 \pi G (T_{\mu\nu}  + \rho_\Lambda g_{\mu\nu})\).
\end{notice}
\end{quote}

\end{itemize}

\index{avertissements!tip}\index{directive!tip}\begin{itemize}
\item {} 
tip
\begin{quote}

\begin{notice}{tip}{Astuce:}
Equations within a note
\(G_{\mu\nu} = 8 \pi G (T_{\mu\nu}  + \rho_\Lambda g_{\mu\nu})\).
\end{notice}
\end{quote}

\item {} 
tip avec paramètre
\begin{quote}

\begin{notice}{tip}{Astuce:}
Equations within a note
\(G_{\mu\nu} = 8 \pi G (T_{\mu\nu}  + \rho_\Lambda g_{\mu\nu})\).
\end{notice}
\end{quote}

\end{itemize}


\subsection{Génériques}
\label{rtd/instruction_base/avertissements:generiques}
\index{avertissements!admonition}\index{directive!admonition}
\begin{Verbatim}[commandchars=\\\{\}]
\PYG{p}{..} \PYG{o+ow}{admonition}\PYG{p}{::} Custom

    Equations within a note
    \PYG{n+na}{:math:}\PYG{n+nv}{{}`G\PYGZus{}\PYGZob{}\PYGZbs{}mu\PYGZbs{}nu\PYGZcb{} = 8 \PYGZbs{}pi G (T\PYGZus{}\PYGZob{}\PYGZbs{}mu\PYGZbs{}nu\PYGZcb{}  + \PYGZbs{}rho\PYGZus{}\PYGZbs{}Lambda g\PYGZus{}\PYGZob{}\PYGZbs{}mu\PYGZbs{}nu\PYGZcb{}){}`}.
\end{Verbatim}

\begin{notice}{note}{Custom}

Equations within a note
\(G_{\mu\nu} = 8 \pi G (T_{\mu\nu}  + \rho_\Lambda g_{\mu\nu})\).
\end{notice}

\begin{notice}{note}{Custom}

Equations within a note
\(G_{\mu\nu} = 8 \pi G (T_{\mu\nu}  + \rho_\Lambda g_{\mu\nu})\).
\end{notice}

\href{http://docutils.sourceforge.net/docs/ref/rst/directives.html\#admonitions}{http://docutils.sourceforge.net/docs/ref/rst/directives.html\#admonitions}


\section{Citations}
\label{rtd/instruction_base/citation:citations}\label{rtd/instruction_base/citation::doc}
\index{citation}\index{lien!citation}\begin{tcolorbox}
\begin{minipage}{0.95\linewidth}
\textbf{Rendu}

\medskip


Lorem ipsum \phantomsection\label{rtd/instruction_base/citation:id1}{\hyperref[rtd/instruction_base/citation:ref]{\emph{{[}Ref{]}}}} (\autopageref*{rtd/instruction_base/citation:ref}) dolor sit amet.

{[}...{]}
\end{minipage}
\end{tcolorbox}

\begin{Verbatim}[commandchars=\\\{\}]
Lorem ipsum \PYG{l+s}{[Ref]\PYGZus{}} dolor sit amet.

[...]

\PYG{p}{..} \PYG{n+nt}{[Ref]} Book or article reference, URL or whatever.
\end{Verbatim}



\begin{notice}{note}{Note:}
Les citations sont globales. Il faut les définir une fois, ensuite on peut les appeler de n'importe où dans le document.

Une citation n'est pas numéroté, contrairement au note de bas de page.
\end{notice}


\section{Code}
\label{rtd/instruction_base/code:code}\label{rtd/instruction_base/code::doc}
\index{code}

\subsection{Bloc littéral}
\label{rtd/instruction_base/code:bloc-litteral}\label{rtd/instruction_base/code:index-0}
\index{code!bloc littéral}
Un bloc n'est pas interprété s'il est précédé de \code{::} et d'une ligne vide :
\begin{tcolorbox}
\begin{minipage}{0.95\linewidth}
\textbf{Rendu}

\medskip


Block one:

\begin{Verbatim}[commandchars=\\\{\}]
\PYG{g+gs}{**No**} interpretation of
\textbar{}special\textbar{} characters.
\end{Verbatim}
\end{minipage}
\end{tcolorbox}

\begin{Verbatim}[commandchars=\\\{\}]
Block one\PYG{l+s+se}{::}

\PYG{l+s}{    }\PYG{l+s}{**No** interpretation of}
\PYG{l+s}{    \textbar{}special\textbar{} characters.}
\end{Verbatim}




\subsection{Bloc littéral interprété}
\label{rtd/instruction_base/code:bloc-litteral-interprete}
\index{code!bloc littéral interprété}\index{directive!parsed-literal}\begin{tcolorbox}
\begin{minipage}{0.95\linewidth}
\textbf{Rendu}

\medskip

\begin{alltt}
( ({\hyperref[rtd/instruction_base/code:code]{\emph{title}}} (\autopageref*{rtd/instruction_base/code:code}), \href{https://www.openwide.fr}{subtitle}?)?,
  \href{https://www.openwide.fr}{decoration}?,
  (\href{https://www.openwide.fr}{docinfo}, \href{https://www.openwide.fr}{transition}?)?,
  \href{https://www.openwide.fr}{\%structure.model;} )
\end{alltt}
\end{minipage}
\end{tcolorbox}

\begin{Verbatim}[commandchars=\\\{\}]
\PYG{p}{..} \PYG{o+ow}{parsed\PYGZhy{}literal}\PYG{p}{::}

   ( (title\PYGZus{}, subtitle\PYGZus{}?)?,
     decoration\PYGZus{}?,
     (docinfo\PYGZus{}, transition\PYGZus{}?)?,
     \PYG{l+s}{{}`\PYGZpc{}structure.model;{}`\PYGZus{}} )

\PYG{p}{..} \PYG{n+nt}{\PYGZus{}title:} \PYGZsh{}code
\PYG{p}{..} \PYG{n+nt}{\PYGZus{}subtitle:} https://www.openwide.fr
\PYG{p}{..} \PYG{n+nt}{\PYGZus{}decoration:} https://www.openwide.fr
\PYG{p}{..} \PYG{n+nt}{\PYGZus{}docinfo:} https://www.openwide.fr
\PYG{p}{..} \PYG{n+nt}{\PYGZus{}transition:} https://www.openwide.fr
\PYG{p}{..} \PYG{n+nt}{\PYGZus{}\PYGZpc{}structure.model;:} https://www.openwide.fr
\end{Verbatim}




\subsection{Bloc de code}
\label{rtd/instruction_base/code:bloc-de-code}
\index{code!bloc de code}\index{directive!code-block}\begin{tcolorbox}
\begin{minipage}{0.95\linewidth}
\textbf{Rendu}

\medskip


\begin{Verbatim}[commandchars=\\\{\}]
\PYG{p}{\PYGZob{}}
\PYG{n+nt}{\PYGZdq{}windows\PYGZdq{}}\PYG{p}{:} \PYG{p}{[}
    \PYG{p}{\PYGZob{}}
    \PYG{n+nt}{\PYGZdq{}panes\PYGZdq{}}\PYG{p}{:} \PYG{p}{[}
        \PYG{p}{\PYGZob{}}
        \PYG{n+nt}{\PYGZdq{}shell\PYGZus{}command\PYGZdq{}}\PYG{p}{:} \PYG{p}{[}
            \PYG{l+s+s2}{\PYGZdq{}echo \PYGZsq{}did you know\PYGZsq{}\PYGZdq{}}\PYG{p}{,}
            \PYG{l+s+s2}{\PYGZdq{}echo \PYGZsq{}you can inline\PYGZsq{}\PYGZdq{}}
        \PYG{p}{]}
        \PYG{p}{\PYGZcb{}}\PYG{p}{,}
        \PYG{p}{\PYGZob{}}
        \PYG{n+nt}{\PYGZdq{}shell\PYGZus{}command\PYGZdq{}}\PYG{p}{:} \PYG{l+s+s2}{\PYGZdq{}echo \PYGZsq{}single commands\PYGZsq{}\PYGZdq{}}
        \PYG{p}{\PYGZcb{}}\PYG{p}{,}
        \PYG{l+s+s2}{\PYGZdq{}echo \PYGZsq{}for panes\PYGZsq{}\PYGZdq{}}
    \PYG{p}{]}\PYG{p}{,}
    \PYG{n+nt}{\PYGZdq{}window\PYGZus{}name\PYGZdq{}}\PYG{p}{:} \PYG{l+s+s2}{\PYGZdq{}long form\PYGZdq{}}
    \PYG{p}{\PYGZcb{}}
\PYG{p}{]}\PYG{p}{,}
\PYG{n+nt}{\PYGZdq{}session\PYGZus{}name\PYGZdq{}}\PYG{p}{:} \PYG{l+s+s2}{\PYGZdq{}shorthands\PYGZdq{}}
\PYG{p}{\PYGZcb{}}
\end{Verbatim}
\end{minipage}
\end{tcolorbox}

\begin{Verbatim}[commandchars=\\\{\}]
.. code\PYGZhy{}block:: json

    \PYGZob{}
    \PYGZdq{}windows\PYGZdq{}: [
        \PYGZob{}
        \PYGZdq{}panes\PYGZdq{}: [
            \PYGZob{}
            \PYGZdq{}shell\PYGZus{}command\PYGZdq{}: [
                \PYGZdq{}echo \PYGZsq{}did you know\PYGZsq{}\PYGZdq{},
                \PYGZdq{}echo \PYGZsq{}you can inline\PYGZsq{}\PYGZdq{}
            ]
            \PYGZcb{},
            \PYGZob{}
            \PYGZdq{}shell\PYGZus{}command\PYGZdq{}: \PYGZdq{}echo \PYGZsq{}single commands\PYGZsq{}\PYGZdq{}
            \PYGZcb{},
            \PYGZdq{}echo \PYGZsq{}for panes\PYGZsq{}\PYGZdq{}
        ],
        \PYGZdq{}window\PYGZus{}name\PYGZdq{}: \PYGZdq{}long form\PYGZdq{}
        \PYGZcb{}
    ],
    \PYGZdq{}session\PYGZus{}name\PYGZdq{}: \PYGZdq{}shorthands\PYGZdq{}
    \PYGZcb{}
\end{Verbatim}



\begin{longtable}{|l|l|}
\hline
\headcol \textsf{\relax\textcolor{white}{
Language
}} & \textsf{\relax\textcolor{white}{
Lexer
}}\\
\hline\endfirsthead

\hline \multicolumn{2}{|r|}{{\textsf{Suite de la page précédente}}} \\ \hline
\hline
\headcol \textsf{\relax\textcolor{white}{
Language
}} & \textsf{\relax\textcolor{white}{
Lexer
}}\\
\hline\endhead

\hline \multicolumn{2}{|r|}{{\textsf{Suite sur la page suivante}}} \\ \hline
\endfoot

\endlastfoot

 & 
none
\\
\hline
Laisser Pygments deviner
 & 
guess
\\
\hline
PHP
 & 
php, php3, php4, php5
\\
\hline
PHP+HTML
 & 
html+php
\\
\hline
Twig+HTML
 & 
html+twig
\\
\hline
Twig
 & 
twig
\\
\hline
HTML+Smarty
 & 
html+smarty
\\
\hline
Smarty
 & 
smarty
\\
\hline
Configuration Apache
 & 
apacheconf, aconf, apache
\\
\hline
Configuration Nginx
 & 
nginx
\\
\hline
Docker
 & 
docker, dockerfile
\\
\hline
Ini
 & 
ini, cfg, dosini
\\
\hline
Properties
 & 
properties, jproperties
\\
\hline
CSS
 & 
css
\\
\hline
Sass
 & 
sass
\\
\hline
Scss
 & 
scss
\\
\hline
JsonLd
 & 
jsonld, json-ld
\\
\hline
Json
 & 
json
\\
\hline
Yaml
 & 
yaml
\\
\hline
Diff
 & 
diff, udiff
\\
\hline
XML
 & 
xml
\\
\hline
HTML
 & 
html
\\
\hline
Javascript
 & 
js, javascript
\\
\hline
Rst
 & 
rst, rest, restructuredtext
\\
\hline
Tex
 & 
tex, latex
\\
\hline
Bash
 & 
bash, sh, ksh, shell
\\
\hline
BashSession
 & 
console
\\
\hline
MySql
 & 
mysql
\\
\hline
Postgres
 & 
postgresql, postgres
\\
\hline
Sqlite
 & 
sqlite3
\\
\hline
Sql
 & 
sql
\\
\hline\end{longtable}



\subsection{Lignes surlignées et numéros de lignes}
\label{rtd/instruction_base/code:lignes-surlignees-et-numeros-de-lignes}
\begin{Verbatim}[commandchars=\\\{\},numbers=left,firstnumber=1,stepnumber=1]
\PYG{k}{def} \PYG{n+nf}{some\PYGZus{}function}\PYG{p}{(}\PYG{p}{)}\PYG{p}{:}
    \PYG{n}{interesting} \PYG{o}{=} \PYG{n+nb+bp}{False}
    \PYG{k}{print} \PYG{l+s}{\PYGZsq{}}\PYG{l+s}{This line is highlighted.}\PYG{l+s}{\PYGZsq{}}
    \PYG{k}{print} \PYG{l+s}{\PYGZsq{}}\PYG{l+s}{This one is not...}\PYG{l+s}{\PYGZsq{}}
    \PYG{k}{print} \PYG{l+s}{\PYGZsq{}}\PYG{l+s}{...but this one is.}\PYG{l+s}{\PYGZsq{}}
\end{Verbatim}

\begin{Verbatim}[commandchars=\\\{\}]
\PYG{p}{..} \PYG{o+ow}{code\PYGZhy{}block}\PYG{p}{::} python
   \PYG{n+nc}{:linenos:}
   \PYG{n+nc}{:emphasize\PYGZhy{}lines:} \PYG{n+nf}{3,5}

   def some\PYGZus{}function():
       interesting = False
       print \PYGZsq{}This line is highlighted.\PYGZsq{}
       print \PYGZsq{}This one is not...\PYGZsq{}
       print \PYGZsq{}...but this one is.\PYGZsq{}
\end{Verbatim}


\section{Commentaires}
\label{rtd/instruction_base/commentaires:commentaires}\label{rtd/instruction_base/commentaires::doc}
\index{commentaire}
Tout ce qui commence comme une directive (\code{.. {}`{}`) mais ne finit pas par {}`{}`::} est un commentaire.
\begin{tcolorbox}
\begin{minipage}{0.95\linewidth}
\textbf{Rendu}

\medskip


Plus dans le commentaire
\end{minipage}
\end{tcolorbox}

\begin{Verbatim}[commandchars=\\\{\}]
\PYG{c+cp}{.. Commentaire}
\PYG{c+cp}{    Encore en commentaire}

Plus dans le commentaire
\end{Verbatim}




\section{Conditions}
\label{rtd/instruction_base/conditions:conditions}\label{rtd/instruction_base/conditions::doc}

\section{Customisation CSS}
\label{rtd/instruction_base/custom_css:customisation-css}\label{rtd/instruction_base/custom_css::doc}
\index{customisation CSS}

\subsection{Container}
\label{rtd/instruction_base/custom_css:container}\label{rtd/instruction_base/custom_css:index-0}
\index{directive!container}
La directive \code{container} permet d'ajouter une \code{div} avec une classe CSS autour d'un bloc.
\begin{tcolorbox}
\begin{minipage}{0.95\linewidth}
\textbf{Rendu}

\medskip


\begin{Verbatim}[commandchars=\\\{\}]
\PYG{n+nt}{\PYGZlt{}div} \PYG{n+na}{class=}\PYG{l+s}{\PYGZdq{}myclass container\PYGZdq{}}\PYG{n+nt}{\PYGZgt{}}
    There is also a general ...
\PYG{n+nt}{\PYGZlt{}/div\PYGZgt{}}
\end{Verbatim}
\end{minipage}
\end{tcolorbox}

\begin{Verbatim}[commandchars=\\\{\}]
\PYG{p}{..} \PYG{o+ow}{container}\PYG{p}{::} myclass

   There is also a general ...
\end{Verbatim}




\subsection{Class}
\label{rtd/instruction_base/custom_css:class}
\index{directive!rst-class}
La directive \code{rst-class} (qui replace \code{class} dans Sphinx), ajoute une classe sur son contenu ou sur le premier élément se trouvant immédiatement après.
\begin{tcolorbox}
\begin{minipage}{0.95\linewidth}
\textbf{Rendu}

\medskip


\begin{Verbatim}[commandchars=\\\{\}]
\PYG{n+nt}{\PYGZlt{}p} \PYG{n+na}{class=}\PYG{l+s}{\PYGZdq{}wy\PYGZhy{}text\PYGZhy{}info\PYGZdq{}}\PYG{n+nt}{\PYGZgt{}}Info text.\PYG{n+nt}{\PYGZlt{}/p\PYGZgt{}}
\PYG{n+nt}{\PYGZlt{}p} \PYG{n+na}{class=}\PYG{l+s}{\PYGZdq{}wy\PYGZhy{}text\PYGZhy{}info\PYGZdq{}}\PYG{n+nt}{\PYGZgt{}}Info text.\PYG{n+nt}{\PYGZlt{}/p\PYGZgt{}}
\end{Verbatim}
\end{minipage}
\end{tcolorbox}

\begin{Verbatim}[commandchars=\\\{\}]
\PYG{p}{..} \PYG{o+ow}{rst\PYGZhy{}class}\PYG{p}{::} wy\PYGZhy{}text\PYGZhy{}info

    Info text.

    Info text.
\end{Verbatim}


\begin{tcolorbox}
\begin{minipage}{0.95\linewidth}
\textbf{Rendu}

\medskip


\begin{Verbatim}[commandchars=\\\{\}]
\PYG{n+nt}{\PYGZlt{}p} \PYG{n+na}{class=}\PYG{l+s}{\PYGZdq{}wy\PYGZhy{}text\PYGZhy{}info\PYGZdq{}}\PYG{n+nt}{\PYGZgt{}}Info text.\PYG{n+nt}{\PYGZlt{}/p\PYGZgt{}}
\PYG{n+nt}{\PYGZlt{}p}\PYG{n+nt}{\PYGZgt{}}Not info text.\PYG{n+nt}{\PYGZlt{}/p\PYGZgt{}}
\end{Verbatim}
\end{minipage}
\end{tcolorbox}

\begin{Verbatim}[commandchars=\\\{\}]
\PYG{p}{..} \PYG{o+ow}{rst\PYGZhy{}class}\PYG{p}{::} wy\PYGZhy{}text\PYGZhy{}info

Info text.

Not info text.
\end{Verbatim}




\subsection{Role}
\label{rtd/instruction_base/custom_css:role}
\index{role!classe CSS}\begin{tcolorbox}
\begin{minipage}{0.95\linewidth}
\textbf{Rendu}

\medskip


\begin{Verbatim}[commandchars=\\\{\}]
\PYG{n+nt}{\PYGZlt{}p}\PYG{n+nt}{\PYGZgt{}}An example of using \PYG{n+nt}{\PYGZlt{}span} \PYG{n+na}{class=}\PYG{l+s}{\PYGZdq{}wy\PYGZhy{}text\PYGZhy{}danger\PYGZdq{}}\PYG{n+nt}{\PYGZgt{}}interpreted text\PYG{n+nt}{\PYGZlt{}/span\PYGZgt{}}\PYG{n+nt}{\PYGZlt{}/p\PYGZgt{}}
\end{Verbatim}
\end{minipage}
\end{tcolorbox}

\begin{Verbatim}[commandchars=\\\{\}]
\PYG{p}{..} \PYG{o+ow}{role}\PYG{p}{::} wy\PYGZhy{}text\PYGZhy{}danger

An example of using \PYG{n+na}{:wy\PYGZhy{}text\PYGZhy{}danger:}\PYG{n+nv}{{}`interpreted text{}`}
\end{Verbatim}




\section{Notes de bas de page}
\label{rtd/instruction_base/footnotes:notes-de-bas-de-page}\label{rtd/instruction_base/footnotes::doc}
\index{note de pied de page}\index{lien!note de pied de page}
Pour les notes de bas de page, utiliser \code{{[}\#name{]}\_} pour marquez n'endroit où insérer la note dans le texte, puis ajouter le corps de la note en base du document après la rubrique :
\begin{tcolorbox}
\begin{minipage}{0.95\linewidth}
\textbf{Rendu}

\medskip


Lorem ipsum\footnote{
Text of the first footnote.
} dolor sit amet ...\footnote{
Text of the second footnote.
}

{[}...{]}
\paragraph{Notes}
\end{minipage}
\end{tcolorbox}

\begin{Verbatim}[commandchars=\\\{\}]
Lorem ipsum \PYG{l+s}{[\PYGZsh{}f1]\PYGZus{}} dolor sit amet ... \PYG{l+s}{[\PYGZsh{}f2]\PYGZus{}}

[...]

\PYG{p}{..} \PYG{o+ow}{rubric}\PYG{p}{::} Notes

\PYG{p}{..} \PYG{n+nt}{[\PYGZsh{}f1]} Text of the first footnote.
\PYG{p}{..} \PYG{n+nt}{[\PYGZsh{}f2]} Text of the second footnote.
\end{Verbatim}




\section{Images}
\label{rtd/instruction_base/images:images}\label{rtd/instruction_base/images::doc}
\index{image}\index{directive!image}\begin{tcolorbox}
\begin{minipage}{0.95\linewidth}
\textbf{Rendu}

\medskip


\includegraphics{Open-Wide_php_logo.png}
\end{minipage}
\end{tcolorbox}

\begin{Verbatim}[commandchars=\\\{\}]
\PYG{p}{..} \PYG{o+ow}{image}\PYG{p}{::} /\PYGZus{}static/images/Open\PYGZhy{}Wide\PYGZus{}php\PYGZus{}logo.png
\end{Verbatim}


\begin{tcolorbox}
\begin{minipage}{0.95\linewidth}
\textbf{Rendu}

\medskip

\href{http://www.openwide.fr}{\scalebox{0.500000}{\includegraphics{Open-Wide_php_logo.png}}}\end{minipage}
\end{tcolorbox}

\begin{Verbatim}[commandchars=\\\{\}]
\PYG{p}{..} \PYG{o+ow}{image}\PYG{p}{::} /\PYGZus{}static/images/Open\PYGZhy{}Wide\PYGZus{}php\PYGZus{}logo.png
    \PYG{n+nc}{:height:} \PYG{n+nf}{100px}
    \PYG{n+nc}{:width:} \PYG{n+nf}{200 px}
    \PYG{n+nc}{:scale:} \PYG{n+nf}{50 \PYGZpc{}}
    \PYG{n+nc}{:alt:} \PYG{n+nf}{alternate text}
    \PYG{n+nc}{:align:} \PYG{n+nf}{center}
    \PYG{n+nc}{:target:} \PYG{n+nf}{http://www.openwide.fr}
\end{Verbatim}



Vous pouvez ajouter la class \code{box} à votre image pour ajouter une bordure :
\begin{tcolorbox}
\begin{minipage}{0.95\linewidth}
\textbf{Rendu}

\medskip


{\hfill\scalebox{0.500000}{\includegraphics{Open-Wide_php_logo.png}}\hfill}
\end{minipage}
\end{tcolorbox}

\begin{Verbatim}[commandchars=\\\{\}]
\PYG{p}{..} \PYG{o+ow}{image}\PYG{p}{::} /\PYGZus{}static/images/Open\PYGZhy{}Wide\PYGZus{}php\PYGZus{}logo.png
    \PYG{n+nc}{:height:} \PYG{n+nf}{100px}
    \PYG{n+nc}{:width:} \PYG{n+nf}{200 px}
    \PYG{n+nc}{:scale:} \PYG{n+nf}{50 \PYGZpc{}}
    \PYG{n+nc}{:align:} \PYG{n+nf}{center}
    \PYG{n+nc}{:class:} \PYG{n+nf}{box}
\end{Verbatim}




\section{Figures}
\label{rtd/instruction_base/images:figures}
\index{figure}\index{image!figure}\index{directive!figure}\begin{figure}[htbp]
\centering
\capstart

\scalebox{1.500000}{\includegraphics{Open-Wide_php_logo.png}}
\caption{This is the caption of the figure (a simple paragraph).}{\small 
The legend consists of all elements after the caption.  In this
case, the legend consists of this paragraph and the following
table:
\begin{quote}
\begin{table}[H]
\centering

\begin{tabulary}{\linewidth}{|L|}
\hline
\headcol \textsf{\relax\textcolor{white}{
Meaning
}}\\
\hline
Campground
\\
\hline
Lake
\\
\hline
Mountain
\\
\hline\end{tabulary}

\end{table}

\end{quote}
}\label{rtd/instruction_base/images:index-1}\end{figure}

\begin{Verbatim}[commandchars=\\\{\}]
\PYG{p}{..} \PYG{o+ow}{figure}\PYG{p}{::} /\PYGZus{}static/images/Open\PYGZhy{}Wide\PYGZus{}php\PYGZus{}logo.png
   \PYG{n+nc}{:scale:} \PYG{n+nf}{150 \PYGZpc{}}
   \PYG{n+nc}{:align:} \PYG{n+nf}{center}
   \PYG{n+nc}{:alt:} \PYG{n+nf}{map to buried treasure}

   This is the caption of the figure (a simple paragraph).

   The legend consists of all elements after the caption.  In this
   case, the legend consists of this paragraph and the following
   table:

    +\PYGZhy{}\PYGZhy{}\PYGZhy{}\PYGZhy{}\PYGZhy{}\PYGZhy{}\PYGZhy{}\PYGZhy{}\PYGZhy{}\PYGZhy{}\PYGZhy{}\PYGZhy{}\PYGZhy{}\PYGZhy{}\PYGZhy{}\PYGZhy{}\PYGZhy{}\PYGZhy{}\PYGZhy{}\PYGZhy{}\PYGZhy{}\PYGZhy{}\PYGZhy{}+
    \PYG{o}{\textbar{}} Meaning               \textbar{}
    +=======================+
    \PYG{o}{\textbar{}} Campground            \textbar{}
    +\PYGZhy{}\PYGZhy{}\PYGZhy{}\PYGZhy{}\PYGZhy{}\PYGZhy{}\PYGZhy{}\PYGZhy{}\PYGZhy{}\PYGZhy{}\PYGZhy{}\PYGZhy{}\PYGZhy{}\PYGZhy{}\PYGZhy{}\PYGZhy{}\PYGZhy{}\PYGZhy{}\PYGZhy{}\PYGZhy{}\PYGZhy{}\PYGZhy{}\PYGZhy{}+
    \PYG{o}{\textbar{}} Lake                  \textbar{}
    +\PYGZhy{}\PYGZhy{}\PYGZhy{}\PYGZhy{}\PYGZhy{}\PYGZhy{}\PYGZhy{}\PYGZhy{}\PYGZhy{}\PYGZhy{}\PYGZhy{}\PYGZhy{}\PYGZhy{}\PYGZhy{}\PYGZhy{}\PYGZhy{}\PYGZhy{}\PYGZhy{}\PYGZhy{}\PYGZhy{}\PYGZhy{}\PYGZhy{}\PYGZhy{}+
    \PYG{o}{\textbar{}} Mountain              \textbar{}
    +\PYGZhy{}\PYGZhy{}\PYGZhy{}\PYGZhy{}\PYGZhy{}\PYGZhy{}\PYGZhy{}\PYGZhy{}\PYGZhy{}\PYGZhy{}\PYGZhy{}\PYGZhy{}\PYGZhy{}\PYGZhy{}\PYGZhy{}\PYGZhy{}\PYGZhy{}\PYGZhy{}\PYGZhy{}\PYGZhy{}\PYGZhy{}\PYGZhy{}\PYGZhy{}+
\end{Verbatim}




\section{Inclusion de fichier}
\label{rtd/instruction_base/inclusion-fichier::doc}\label{rtd/instruction_base/inclusion-fichier:inclusion-de-fichier}
\index{inclusion de fichier}\index{directive!include}
\begin{Verbatim}[commandchars=\\\{\}]
\PYG{p}{..} \PYG{o+ow}{include}\PYG{p}{::} subdir/incl.rst
\end{Verbatim}

Cette directive prend les options suivantes : \code{start-line}, \code{end-line}, \code{start-after}, \code{end-before}.

Pour éviter les warnings concernant des fichiers qui ne sont pas inclus dans le toctree, ajouter le répertoire ou les fichiers dans la configuration :

\begin{Verbatim}[commandchars=\\\{\}]
\PYG{c}{\PYGZsh{} conf.py}

\PYG{n}{exclude\PYGZus{}patterns} \PYG{o}{=} \PYG{p}{[}\PYG{l+s}{\PYGZdq{}}\PYG{l+s}{include/*}\PYG{l+s}{\PYGZdq{}}\PYG{p}{]}
\end{Verbatim}


\section{Inline markup}
\label{rtd/instruction_base/inline-markup:inline-markup}\label{rtd/instruction_base/inline-markup::doc}
\index{inline markup}\index{mise en forme}\index{mise en forme!italique}\index{mise en forme!italique}\index{mise en forme!gras}\index{code!inline markup}\index{inline markup!strong}\index{inline markup!emphasize}\index{role!emphasize}\index{inline markup!code}\index{role!code}\index{role!literal}\index{role!subscript}\index{role!sub}\index{role!title-reference}\index{role!title}\index{role!t}\index{role!PEP}\index{role!pep-reference}\index{role!rfc-reference}\index{role!RFC}

\subsection{Italique}
\label{rtd/instruction_base/inline-markup:italique}\label{rtd/instruction_base/inline-markup:index-0}\begin{tcolorbox}
\begin{minipage}{0.95\linewidth}
\textbf{Rendu}

\medskip


\begin{DUlineblock}{0em}
\item[] \emph{emphasize}
\item[] \emph{emphasize}
\end{DUlineblock}
\end{minipage}
\end{tcolorbox}

\begin{Verbatim}[commandchars=\\\{\}]
\PYG{g+ge}{*emphasize*}
\PYG{n+na}{:emphasis:}\PYG{n+nv}{{}`emphasize{}`}
\end{Verbatim}




\subsection{Gras}
\label{rtd/instruction_base/inline-markup:gras}\begin{tcolorbox}
\begin{minipage}{0.95\linewidth}
\textbf{Rendu}

\medskip


\begin{DUlineblock}{0em}
\item[] \textbf{strong}
\item[] \textbf{strong}
\end{DUlineblock}
\end{minipage}
\end{tcolorbox}

\begin{Verbatim}[commandchars=\\\{\}]
\PYG{g+gs}{**strong**}
\PYG{n+na}{:strong:}\PYG{n+nv}{{}`strong{}`}
\end{Verbatim}




\subsection{Code}
\label{rtd/instruction_base/inline-markup:code}\begin{tcolorbox}
\begin{minipage}{0.95\linewidth}
\textbf{Rendu}

\medskip


\begin{DUlineblock}{0em}
\item[] \code{code}
\item[] \code{code}
\item[] \code{literal}
\end{DUlineblock}
\end{minipage}
\end{tcolorbox}

\begin{Verbatim}[commandchars=\\\{\}]
\PYG{l+s}{{}`{}`}\PYG{l+s}{code}\PYG{l+s}{{}`{}`}
\PYG{n+na}{:code:}\PYG{n+nv}{{}`code{}`}
\PYG{n+na}{:literal:}\PYG{n+nv}{{}`literal{}`}
\end{Verbatim}




\subsection{Expression régulière}
\label{rtd/instruction_base/inline-markup:expression-reguliere}\begin{tcolorbox}
\begin{minipage}{0.95\linewidth}
\textbf{Rendu}

\medskip


\begin{DUlineblock}{0em}
\item[] \code{(.*)}
\end{DUlineblock}
\end{minipage}
\end{tcolorbox}

\begin{Verbatim}[commandchars=\\\{\}]
\PYG{n+na}{:regexp:}\PYG{n+nv}{{}`(.*){}`}
\end{Verbatim}




\subsection{Exemple}
\label{rtd/instruction_base/inline-markup:exemple}\begin{tcolorbox}
\begin{minipage}{0.95\linewidth}
\textbf{Rendu}

\medskip


\begin{DUlineblock}{0em}
\item[] \code{toto}
\end{DUlineblock}
\end{minipage}
\end{tcolorbox}

\begin{Verbatim}[commandchars=\\\{\}]
\PYG{n+na}{:samp:}\PYG{n+nv}{{}`toto{}`}
\end{Verbatim}




\subsection{Commande}
\label{rtd/instruction_base/inline-markup:commande}\begin{tcolorbox}
\begin{minipage}{0.95\linewidth}
\textbf{Rendu}

\medskip


\begin{DUlineblock}{0em}
\item[] \textbf{\texttt{ls -l}}
\end{DUlineblock}
\end{minipage}
\end{tcolorbox}

\begin{Verbatim}[commandchars=\\\{\}]
\PYG{n+na}{:command:}\PYG{n+nv}{{}`ls \PYGZhy{}l{}`}
\end{Verbatim}




\subsection{Programme}
\label{rtd/instruction_base/inline-markup:programme}\begin{tcolorbox}
\begin{minipage}{0.95\linewidth}
\textbf{Rendu}

\medskip


\begin{DUlineblock}{0em}
\item[] \textbf{\texttt{libreoffice}}
\end{DUlineblock}
\end{minipage}
\end{tcolorbox}

\begin{Verbatim}[commandchars=\\\{\}]
\PYG{n+na}{:program:}\PYG{n+nv}{{}`libreoffice{}`}
\end{Verbatim}




\subsection{Fichier}
\label{rtd/instruction_base/inline-markup:fichier}\begin{tcolorbox}
\begin{minipage}{0.95\linewidth}
\textbf{Rendu}

\medskip


\begin{DUlineblock}{0em}
\item[] \code{/etc/hosts}
\end{DUlineblock}
\end{minipage}
\end{tcolorbox}

\begin{Verbatim}[commandchars=\\\{\}]
\PYG{n+na}{:file:}\PYG{n+nv}{{}`/etc/hosts{}`}
\end{Verbatim}




\subsection{Définition}
\label{rtd/instruction_base/inline-markup:definition}\begin{tcolorbox}
\begin{minipage}{0.95\linewidth}
\textbf{Rendu}

\medskip


\begin{DUlineblock}{0em}
\item[] Marquez l'instance la définition d'un terme dans le texte. (Pas d'entrées d'\emph{index} sont générés.)
\end{DUlineblock}
\end{minipage}
\end{tcolorbox}

\begin{Verbatim}[commandchars=\\\{\}]
Marquez l\PYGZsq{}instance la définition d\PYGZsq{}un terme dans le texte. (Pas d\PYGZsq{}entrées d\PYGZsq{}\PYG{n+na}{:dfn:}\PYG{n+nv}{{}`index{}`} sont générés.)
\end{Verbatim}




\subsection{Référence à des titres d'œuvres (livres, films, etc.)}
\label{rtd/instruction_base/inline-markup:reference-a-des-titres-d-oeuvres-livres-films-etc}\begin{tcolorbox}
\begin{minipage}{0.95\linewidth}
\textbf{Rendu}

\medskip


\begin{DUlineblock}{0em}
\item[] \emph{Design Patterns}
\item[] \emph{Design Patterns}
\item[] \emph{Design Patterns}
\item[] \emph{Design Patterns}
\end{DUlineblock}
\end{minipage}
\end{tcolorbox}

\begin{Verbatim}[commandchars=\\\{\}]
\PYG{n+nv}{{}`Design Patterns{}`}
\PYG{n+na}{:title\PYGZhy{}reference:}\PYG{n+nv}{{}`Design Patterns{}`}
\PYG{n+na}{:title:}\PYG{n+nv}{{}`Design Patterns{}`}
\PYG{n+na}{:t:}\PYG{n+nv}{{}`Design Patterns{}`}
\end{Verbatim}




\subsection{Indice}
\label{rtd/instruction_base/inline-markup:indice}\begin{tcolorbox}
\begin{minipage}{0.95\linewidth}
\textbf{Rendu}

\medskip


\begin{DUlineblock}{0em}
\item[] H$_{\text{2}}$O
\item[] H$_{\text{2}}$O
\end{DUlineblock}
\end{minipage}
\end{tcolorbox}

\begin{Verbatim}[commandchars=\\\{\}]
H\PYGZbs{} \PYG{n+na}{:subscript:}\PYG{n+nv}{{}`2{}`}\PYGZbs{} O
H\PYGZbs{} \PYG{n+na}{:sub:}\PYG{n+nv}{{}`2{}`}\PYGZbs{} O
\end{Verbatim}




\subsection{Exposant}
\label{rtd/instruction_base/inline-markup:exposant}\begin{tcolorbox}
\begin{minipage}{0.95\linewidth}
\textbf{Rendu}

\medskip


\begin{DUlineblock}{0em}
\item[] m$^{\text{3}}$
\item[] m$^{\text{3}}$
\end{DUlineblock}
\end{minipage}
\end{tcolorbox}

\begin{Verbatim}[commandchars=\\\{\}]
m\PYGZbs{} \PYG{n+na}{:superscript:}\PYG{n+nv}{{}`3{}`}
m\PYGZbs{} \PYG{n+na}{:sup:}\PYG{n+nv}{{}`3{}`}
\end{Verbatim}




\subsection{Abréviation}
\label{rtd/instruction_base/inline-markup:abreviation}\begin{tcolorbox}
\begin{minipage}{0.95\linewidth}
\textbf{Rendu}

\medskip


\begin{DUlineblock}{0em}
\item[] \textsc{LIFO} (last-in, first-out)
\end{DUlineblock}
\end{minipage}
\end{tcolorbox}

\begin{Verbatim}[commandchars=\\\{\}]
\PYG{n+na}{:abbr:}\PYG{n+nv}{{}`LIFO (last\PYGZhy{}in, first\PYGZhy{}out){}`}
\end{Verbatim}




\subsection{PEP (Python Enhancement Proposal)}
\label{rtd/instruction_base/inline-markup:pep-python-enhancement-proposal}\begin{tcolorbox}
\begin{minipage}{0.95\linewidth}
\textbf{Rendu}

\medskip


\begin{DUlineblock}{0em}
\item[] \index{Python Enhancement Proposals!PEP 287}\href{https://www.python.org/dev/peps/pep-0287}{\textbf{PEP 287}}
\item[] \href{https://www.python.org/dev/peps/pep-0287}{PEP 287}
\end{DUlineblock}
\end{minipage}
\end{tcolorbox}

\begin{Verbatim}[commandchars=\\\{\}]
\PYG{n+na}{:PEP:}\PYG{n+nv}{{}`287{}`}
\PYG{n+na}{:pep\PYGZhy{}reference:}\PYG{n+nv}{{}`287{}`}
\end{Verbatim}




\subsection{RFC (Internet Request for Comments)}
\label{rtd/instruction_base/inline-markup:rfc-internet-request-for-comments}\begin{tcolorbox}
\begin{minipage}{0.95\linewidth}
\textbf{Rendu}

\medskip


\begin{DUlineblock}{0em}
\item[] \index{RFC!RFC 287}\href{https://tools.ietf.org/html/rfc287.html}{\textbf{RFC 287}}
\item[] \href{https://tools.ietf.org/html/rfc287.html}{RFC 287}
\end{DUlineblock}
\end{minipage}
\end{tcolorbox}

\begin{Verbatim}[commandchars=\\\{\}]
\PYG{n+na}{:RFC:}\PYG{n+nv}{{}`287{}`}
\PYG{n+na}{:rfc\PYGZhy{}reference:}\PYG{n+nv}{{}`287{}`}
\end{Verbatim}




\subsection{Label dans l'interface utilisateur}
\label{rtd/instruction_base/inline-markup:label-dans-l-interface-utilisateur}\begin{tcolorbox}
\begin{minipage}{0.95\linewidth}
\textbf{Rendu}

\medskip


\begin{DUlineblock}{0em}
\item[] \emph{Ajouter}
\end{DUlineblock}
\end{minipage}
\end{tcolorbox}

\begin{Verbatim}[commandchars=\\\{\}]
\PYG{n+na}{:guilabel:}\PYG{n+nv}{{}`Ajouter{}`}
\end{Verbatim}




\subsection{Sélection de menus}
\label{rtd/instruction_base/inline-markup:selection-de-menus}\begin{tcolorbox}
\begin{minipage}{0.95\linewidth}
\textbf{Rendu}

\medskip


\begin{DUlineblock}{0em}
\item[] \emph{Start \(\rightarrow\) Programs}
\end{DUlineblock}
\end{minipage}
\end{tcolorbox}

\begin{Verbatim}[commandchars=\\\{\}]
\PYG{n+na}{:menuselection:}\PYG{n+nv}{{}`Start \PYGZhy{}\PYGZhy{}\PYGZgt{} Programs{}`}
\end{Verbatim}




\subsection{Séquence de touches}
\label{rtd/instruction_base/inline-markup:sequence-de-touches}\begin{tcolorbox}
\begin{minipage}{0.95\linewidth}
\textbf{Rendu}

\medskip


\begin{DUlineblock}{0em}
\item[] \code{C-x C-f}
\item[] \code{Control-x Control-f}
\end{DUlineblock}
\end{minipage}
\end{tcolorbox}

\begin{Verbatim}[commandchars=\\\{\}]
\PYG{n+na}{:kbd:}\PYG{n+nv}{{}`C\PYGZhy{}x C\PYGZhy{}f{}`}
\PYG{n+na}{:kbd:}\PYG{n+nv}{{}`Control\PYGZhy{}x Control\PYGZhy{}f{}`}
\end{Verbatim}




\subsection{Entête de mail}
\label{rtd/instruction_base/inline-markup:entete-de-mail}\begin{tcolorbox}
\begin{minipage}{0.95\linewidth}
\textbf{Rendu}

\medskip


\begin{DUlineblock}{0em}
\item[] \emph{\texttt{Content-Type}}
\end{DUlineblock}
\end{minipage}
\end{tcolorbox}

\begin{Verbatim}[commandchars=\\\{\}]
\PYG{n+na}{:mailheader:}\PYG{n+nv}{{}`Content\PYGZhy{}Type{}`}
\end{Verbatim}




\subsection{Mimetype}
\label{rtd/instruction_base/inline-markup:mimetype}\begin{tcolorbox}
\begin{minipage}{0.95\linewidth}
\textbf{Rendu}

\medskip


\begin{DUlineblock}{0em}
\item[] \emph{\texttt{application/pdf}}
\end{DUlineblock}
\end{minipage}
\end{tcolorbox}

\begin{Verbatim}[commandchars=\\\{\}]
\PYG{n+na}{:mimetype:}\PYG{n+nv}{{}`application/pdf{}`}
\end{Verbatim}




\subsection{Variable pour \textbf{\texttt{make}}}
\label{rtd/instruction_base/inline-markup:variable-pour-make}\begin{tcolorbox}
\begin{minipage}{0.95\linewidth}
\textbf{Rendu}

\medskip


\begin{DUlineblock}{0em}
\item[] make \textbf{\texttt{all}}
\end{DUlineblock}
\end{minipage}
\end{tcolorbox}

\begin{Verbatim}[commandchars=\\\{\}]
make \PYG{n+na}{:makevar:}\PYG{n+nv}{{}`all{}`}
\end{Verbatim}




\subsection{Manuel}
\label{rtd/instruction_base/inline-markup:manuel}\begin{tcolorbox}
\begin{minipage}{0.95\linewidth}
\textbf{Rendu}

\medskip


\begin{DUlineblock}{0em}
\item[] \emph{\texttt{ls(1)}}
\end{DUlineblock}
\end{minipage}
\end{tcolorbox}

\begin{Verbatim}[commandchars=\\\{\}]
\PYG{n+na}{:manpage:}\PYG{n+nv}{{}`ls(1){}`}
\end{Verbatim}




\section{Liens}
\label{rtd/instruction_base/liens:top-of-page}\label{rtd/instruction_base/liens::doc}\label{rtd/instruction_base/liens:liens}
\index{lien}

\subsection{Vers un titre de la page}
\label{rtd/instruction_base/liens:index-0}\label{rtd/instruction_base/liens:vers-un-titre-de-la-page}\begin{tcolorbox}
\begin{minipage}{0.95\linewidth}
\textbf{Rendu}

\medskip


{\hyperref[rtd/instruction_base/liens:vers-un-titre-de-la-page]{\emph{Vers un titre de la page}}} (\autopageref*{rtd/instruction_base/liens:vers-un-titre-de-la-page})
\end{minipage}
\end{tcolorbox}

\begin{Verbatim}[commandchars=\\\{\}]
\PYG{l+s}{{}`Vers un titre de la page{}`\PYGZus{}}
\end{Verbatim}




\subsection{Vers une référence dans le document}
\label{rtd/instruction_base/liens:vers-une-reference-dans-le-document}
Placez une référence dans le document, toujours avant un titre :

\begin{Verbatim}[commandchars=\\\{\}]
\PYG{p}{..} \PYG{n+nt}{\PYGZus{}nom\PYGZhy{}du\PYGZhy{}label:}
\end{Verbatim}

Pour faire un lien vers la référence :

\begin{Verbatim}[commandchars=\\\{\}]
\PYG{n+na}{:ref:}\PYG{n+nv}{{}`texte affiché \PYGZlt{}nom\PYGZhy{}du\PYGZhy{}label\PYGZgt{}{}`} ou \PYG{n+na}{:ref:}\PYG{n+nv}{{}`\PYGZlt{}nom\PYGZhy{}du\PYGZhy{}label\PYGZgt{}{}`}
\end{Verbatim}
\begin{tcolorbox}
\begin{minipage}{0.95\linewidth}
\textbf{Rendu}

\medskip


{\hyperref[rtd/instruction_base/liens:top-of-page]{\emph{\DUspan{}{début de la page}}}} (\autopageref*{rtd/instruction_base/liens:top-of-page}) ou {\hyperref[rtd/instruction_base/liens:top-of-page]{\emph{\DUspan{}{Liens}}}} (\autopageref*{rtd/instruction_base/liens:top-of-page})
\end{minipage}
\end{tcolorbox}

\begin{Verbatim}[commandchars=\\\{\}]
\PYG{n+na}{:ref:}\PYG{n+nv}{{}`début de la page \PYGZlt{}top\PYGZhy{}of\PYGZhy{}page\PYGZgt{}{}`}
ou \PYG{n+na}{:ref:}\PYG{n+nv}{{}`top\PYGZhy{}of\PYGZhy{}page{}`}
\end{Verbatim}




\subsection{Lien externe}
\label{rtd/instruction_base/liens:lien-externe}
Première notation, comme les citations :
\begin{tcolorbox}
\begin{minipage}{0.95\linewidth}
\textbf{Rendu}

\medskip


A link to \href{http://sphinx.pocoo.org}{Sphinx Home} in citation style.
\end{minipage}
\end{tcolorbox}

\begin{Verbatim}[commandchars=\\\{\}]
A link to \PYG{l+s}{{}`Sphinx Home{}`\PYGZus{}} in citation style.

\PYG{p}{..} \PYG{n+nt}{\PYGZus{}Sphinx Home:} http://sphinx.pocoo.org
\end{Verbatim}



Seconde notation, en ligne :
\begin{tcolorbox}
\begin{minipage}{0.95\linewidth}
\textbf{Rendu}

\medskip


In-line versions are \href{http://sphinx.pocoo.org}{Sphinx Home} or \href{http://sphinx.pocoo.org}{http://sphinx.pocoo.org} or (in Sphinx) \href{http://sphinx.pocoo.org}{http://sphinx.pocoo.org}
\end{minipage}
\end{tcolorbox}

\begin{Verbatim}[commandchars=\\\{\}]
In\PYGZhy{}line versions are \PYG{l+s}{{}`Sphinx Home }\PYG{l+s+si}{\PYGZlt{}http://sphinx.pocoo.org\PYGZgt{}}\PYG{l+s}{{}`\PYGZus{}} or \PYG{l+s}{{}`\PYGZlt{}http://sphinx.pocoo.org\PYGZgt{}{}`\PYGZus{}} or (in Sphinx) http://sphinx.pocoo.org
\end{Verbatim}




\subsection{Download links}
\label{rtd/instruction_base/liens:download-links}\begin{tcolorbox}
\begin{minipage}{0.95\linewidth}
\textbf{Rendu}

\medskip


\code{Téléchager le logo !}
\end{minipage}
\end{tcolorbox}

\begin{Verbatim}[commandchars=\\\{\}]
\PYG{n+na}{:download:}\PYG{n+nv}{{}`Téléchager le logo !\PYGZlt{}/\PYGZus{}static/images/Open\PYGZhy{}Wide\PYGZus{}php\PYGZus{}logo.png\PYGZgt{}{}`}
\end{Verbatim}




\section{Listes et définitions}
\label{rtd/instruction_base/listes::doc}\label{rtd/instruction_base/listes:listes-et-definitions}
\index{liste}

\subsection{Liste à puce}
\label{rtd/instruction_base/listes:index-0}\label{rtd/instruction_base/listes:liste-a-puce}
\index{liste!liste à puce}
Pour construire une liste, ajoutez le caractère \code{*} au début du paragraphe et indentez la correctement.
\begin{tcolorbox}
\begin{minipage}{0.95\linewidth}
\textbf{Rendu}

\medskip

\begin{itemize}
\item {} 
This is a bulleted list.

\item {} 
It has two items, the second
item uses two lines.

\end{itemize}
\end{minipage}
\end{tcolorbox}

\begin{Verbatim}[commandchars=\\\{\}]
\PYG{l+m}{*} This is a bulleted list.
\PYG{l+m}{*} It has two items, the second
  item uses two lines.
\end{Verbatim}




\subsection{Liste numérotée}
\label{rtd/instruction_base/listes:liste-numerotee}
\index{liste!liste numérotée}
Pour les listes numérotés, remplacez \code{*} par \code{\#}.
\begin{tcolorbox}
\begin{minipage}{0.95\linewidth}
\textbf{Rendu}

\medskip

\begin{enumerate}
\item {} 
This is a numbered list.

\item {} 
It has two items too.

\item {} 
This is a numbered list.

\item {} 
It has two items too.

\end{enumerate}
\begin{enumerate}
\item {} 
Point a

\item {} 
Point b

\item {} 
Automatic point c.

\end{enumerate}
\end{minipage}
\end{tcolorbox}

\begin{Verbatim}[commandchars=\\\{\}]
\PYG{l+m}{1.} This is a numbered list.
\PYG{l+m}{2.} It has two items too.
\PYG{l+m}{\PYGZsh{}.} This is a numbered list.
\PYG{l+m}{\PYGZsh{}.} It has two items too.

a) Point a
b) Point b
\PYG{l+m}{\PYGZsh{})} Automatic point c.
\end{Verbatim}




\subsection{Liste imbriquée}
\label{rtd/instruction_base/listes:liste-imbriquee}
\index{liste!liste imbriquée}
Pour créez des listes imbriquées, faites attention à séparer les listes parent et enfant avec au moins une ligne vide :
\begin{tcolorbox}
\begin{minipage}{0.95\linewidth}
\textbf{Rendu}

\medskip

\begin{itemize}
\item {} 
this is

\item {} 
a list
\begin{itemize}
\item {} 
with a nested list

\item {} 
and some subitems

\end{itemize}

\item {} 
and here the parent list continues

\end{itemize}
\end{minipage}
\end{tcolorbox}

\begin{Verbatim}[commandchars=\\\{\}]
\PYG{l+m}{*} this is
\PYG{l+m}{*} a list

    \PYG{l+m}{*} with a nested list
    \PYG{l+m}{*} and some subitems

\PYG{l+m}{*} and here the parent list continues
\end{Verbatim}




\subsection{Liste horizontale}
\label{rtd/instruction_base/listes:liste-horizontale}
\index{liste!liste horizontale}\index{directive!hlist}\begin{tcolorbox}
\begin{minipage}{0.95\linewidth}
\textbf{Rendu}

\medskip

\begin{itemize}\setlength{\itemsep}{0pt}\setlength{\parskip}{0pt}
\item {} 
list of

\item {} 
short items

\item {} 
that should be

\item {} 
displayed

\item {} 
horizontally

\end{itemize}
\end{minipage}
\end{tcolorbox}

\begin{Verbatim}[commandchars=\\\{\}]
\PYG{p}{..} \PYG{o+ow}{hlist}\PYG{p}{::}
    \PYG{n+nc}{:columns:} \PYG{n+nf}{2}

    \PYG{l+m}{*} list of
    \PYG{l+m}{*} short items
    \PYG{l+m}{*} that should be
    \PYG{l+m}{*} displayed
    \PYG{l+m}{*} horizontally
\end{Verbatim}




\subsection{Liste de définitions}
\label{rtd/instruction_base/listes:liste-de-definitions}
\index{liste!liste de définitions}
Les définitions sont des paragraphes indentés sans marqueur au début :
\begin{tcolorbox}
\begin{minipage}{0.95\linewidth}
\textbf{Rendu}

\medskip

\begin{description}
\item[{term (up to a line of text)}] \leavevmode
Definition of the term, which must be indented

and can even consist of multiple paragraphs

\item[{next term}] \leavevmode
Description.

\end{description}
\end{minipage}
\end{tcolorbox}

\begin{Verbatim}[commandchars=\\\{\}]
term (up to a line of text)
    Definition of the term, which must be indented

    and can even consist of multiple paragraphs
next term
    Description.
\end{Verbatim}




\subsection{Liste de champs}
\label{rtd/instruction_base/listes:liste-de-champs}
\index{liste!liste de champs}\begin{tcolorbox}
\begin{minipage}{0.95\linewidth}
\textbf{Rendu}

\medskip

\begin{quote}\begin{description}
\item[{Name}] \leavevmode
Isaac Newton

\item[{Long}] \leavevmode
Here we insert more
text to show the effect of
many lines.

\item[{Remark}] \leavevmode
Start on the next line.

\end{description}\end{quote}
\end{minipage}
\end{tcolorbox}

\begin{Verbatim}[commandchars=\\\{\}]
\PYG{n+nc}{:Name:} \PYG{n+nf}{Isaac Newton}
\PYG{n+nc}{:Long:} \PYG{n+nf}{Here we insert more}
    text to show the effect of
    many lines.
\PYG{n+nc}{:Remark:}
    Start on the next line.
\end{Verbatim}




\subsection{Liste d'options}
\label{rtd/instruction_base/listes:liste-d-options}
\index{liste!liste d'option}\begin{tcolorbox}
\begin{minipage}{0.95\linewidth}
\textbf{Rendu}

\medskip

\begin{optionlist}{3cm}
\item [-v]  
An option
\item [-o file]  
Same with value
\item [-{-}delta]  
A long option
\item [-{-}delta=len]  
Same with value
\end{optionlist}
\end{minipage}
\end{tcolorbox}

\begin{Verbatim}[commandchars=\\\{\}]
\PYGZhy{}v           An option
\PYGZhy{}o file      Same with value
\PYGZhy{}\PYGZhy{}delta      A long option
\PYGZhy{}\PYGZhy{}delta=len  Same with value
\end{Verbatim}




\section{Math}
\label{rtd/instruction_base/math::doc}\label{rtd/instruction_base/math:math}
\index{math}\index{directive!math}\index{role!math}
\begin{notice}{attention}{Attention:}
Il faut activer l'extension \code{sphinx.ext.mathjax}.
\end{notice}

Voici une équation : \(X_{0:5} = (X_0, X_1, X_2, X_3, X_4)\).

\begin{Verbatim}[commandchars=\\\{\}]
Voici une équation : \PYG{n+na}{:math:}\PYG{n+nv}{{}`X\PYGZus{}\PYGZob{}0:5\PYGZcb{} = (X\PYGZus{}0, X\PYGZus{}1, X\PYGZus{}2, X\PYGZus{}3, X\PYGZus{}4){}`}.
\end{Verbatim}

En voici une autre :
\begin{gather}
\begin{split}\nabla^2 f =
\frac{1}{r^2} \frac{\partial}{\partial r}
\left( r^2 \frac{\partial f}{\partial r} \right) +
\frac{1}{r^2 \sin \theta} \frac{\partial f}{\partial \theta}
\left( \sin \theta \, \frac{\partial f}{\partial \theta} \right) +
\frac{1}{r^2 \sin^2\theta} \frac{\partial^2 f}{\partial \phi^2}\end{split}\notag
\end{gather}
\begin{Verbatim}[commandchars=\\\{\}]
En voici une autre :

\PYG{p}{..} \PYG{o+ow}{math}\PYG{p}{::}

    \PYGZbs{}nabla\PYGZca{}2 f =
    \PYGZbs{}frac\PYGZob{}1\PYGZcb{}\PYGZob{}r\PYGZca{}2\PYGZcb{} \PYGZbs{}frac\PYGZob{}\PYGZbs{}partial\PYGZcb{}\PYGZob{}\PYGZbs{}partial r\PYGZcb{}
    \PYGZbs{}left( r\PYGZca{}2 \PYGZbs{}frac\PYGZob{}\PYGZbs{}partial f\PYGZcb{}\PYGZob{}\PYGZbs{}partial r\PYGZcb{} \PYGZbs{}right) +
    \PYGZbs{}frac\PYGZob{}1\PYGZcb{}\PYGZob{}r\PYGZca{}2 \PYGZbs{}sin \PYGZbs{}theta\PYGZcb{} \PYGZbs{}frac\PYGZob{}\PYGZbs{}partial f\PYGZcb{}\PYGZob{}\PYGZbs{}partial \PYGZbs{}theta\PYGZcb{}
    \PYGZbs{}left( \PYGZbs{}sin \PYGZbs{}theta \PYGZbs{}, \PYGZbs{}frac\PYGZob{}\PYGZbs{}partial f\PYGZcb{}\PYGZob{}\PYGZbs{}partial \PYGZbs{}theta\PYGZcb{} \PYGZbs{}right) +
    \PYGZbs{}frac\PYGZob{}1\PYGZcb{}\PYGZob{}r\PYGZca{}2 \PYGZbs{}sin\PYGZca{}2\PYGZbs{}theta\PYGZcb{} \PYGZbs{}frac\PYGZob{}\PYGZbs{}partial\PYGZca{}2 f\PYGZcb{}\PYGZob{}\PYGZbs{}partial \PYGZbs{}phi\PYGZca{}2\PYGZcb{}
\end{Verbatim}


\section{Paragraphes}
\label{rtd/instruction_base/paragraphes:paragraphes}\label{rtd/instruction_base/paragraphes::doc}

\subsection{Paragraphe simple}
\label{rtd/instruction_base/paragraphes:paragraphe-simple}
Les paragraphes sont tout simplement des morceaux de texte séparés par une ou plusieurs lignes vides.


\subsection{Bloc de lignes}
\label{rtd/instruction_base/paragraphes:bloc-de-lignes}\begin{tcolorbox}
\begin{minipage}{0.95\linewidth}
\textbf{Rendu}

\medskip


\begin{DUlineblock}{0em}
\item[] Line block
\item[] New line and we are still on
the same line
\item[]
\begin{DUlineblock}{\DUlineblockindent}
\item[] Yet a new line
\end{DUlineblock}
\end{DUlineblock}
\end{minipage}
\end{tcolorbox}

\begin{Verbatim}[commandchars=\\\{\}]
\PYG{o}{\textbar{}} Line block
\PYG{o}{\textbar{}} New line and we are still on
    the same line
\PYG{o}{\textbar{}}      Yet a new line
\end{Verbatim}




\subsection{Blockquote}
\label{rtd/instruction_base/paragraphes:blockquote}\begin{tcolorbox}
\begin{minipage}{0.95\linewidth}
\textbf{Rendu}

\medskip


Il faut les indenter par rapport au paragraphe précédent.
\begin{quote}

Neither from itself nor from another,
Nor from both,
Nor without a cause,
Does anything whatever, anywhere arise.

\begin{flushright}
---Nagarjuna - \emph{Mulamadhyamakakarika}
\end{flushright}
\end{quote}
\end{minipage}
\end{tcolorbox}

\begin{Verbatim}[commandchars=\\\{\}]
Il faut les indenter par rapport au paragraphe précédent.

   Neither from itself nor from another,
   Nor from both,
   Nor without a cause,
   Does anything whatever, anywhere arise.

   \PYGZhy{}\PYGZhy{}Nagarjuna \PYGZhy{} \PYG{g+ge}{*Mulamadhyamakakarika*}
\end{Verbatim}




\subsection{Pull quote}
\label{rtd/instruction_base/paragraphes:pull-quote}
\index{directive!pull-quote}
Comme les blockquotes, mais avec une directive.
\begin{tcolorbox}
\begin{minipage}{0.95\linewidth}
\textbf{Rendu}

\medskip

\begin{quote}

Just as a solid rock ...
\end{quote}
\end{minipage}
\end{tcolorbox}

\begin{Verbatim}[commandchars=\\\{\}]
\PYG{p}{..} \PYG{o+ow}{pull\PYGZhy{}quote}\PYG{p}{::}

      Just as a solid rock ...
\end{Verbatim}




\subsection{Epigraph et highlights}
\label{rtd/instruction_base/paragraphes:epigraph-et-highlights}
\index{directive!highlights}\index{directive!epigraph}
Ce sont des blockquotes avec une classe CSS particulière.
\begin{tcolorbox}
\begin{minipage}{0.95\linewidth}
\textbf{Rendu}

\medskip

\begin{quote}

With these \emph{highlights} ...
\end{quote}
\begin{quote}

With these \emph{epigraph} ...
\end{quote}
\end{minipage}
\end{tcolorbox}

\begin{Verbatim}[commandchars=\\\{\}]
\PYG{p}{..} \PYG{o+ow}{highlights}\PYG{p}{::}

      With these \PYG{g+ge}{*highlights*} ...

\PYG{p}{..} \PYG{o+ow}{epigraph}\PYG{p}{::}

      With these \PYG{g+ge}{*epigraph*} ...
\end{Verbatim}




\section{Raw}
\label{rtd/instruction_base/raw:raw}\label{rtd/instruction_base/raw::doc}
\index{raw}\index{raw-html}\index{raw-latex}\index{role!raw}
Ce rôle permet d'insérer du texte écrit directement dans le langage du builder.
\begin{tcolorbox}
\begin{minipage}{0.95\linewidth}
\textbf{Rendu}

\medskip

\end{minipage}
\end{tcolorbox}

\begin{Verbatim}[commandchars=\\\{\}]
\PYG{p}{..} \PYG{o+ow}{raw}\PYG{p}{::} html

    \PYGZlt{}hr width=50 size=10\PYGZgt{}
\end{Verbatim}


\begin{tcolorbox}
\begin{minipage}{0.95\linewidth}
\textbf{Rendu}

\medskip

\setlength{\parindent}{0pt}\end{minipage}
\end{tcolorbox}

\begin{Verbatim}[commandchars=\\\{\}]
\PYG{p}{..} \PYG{o+ow}{raw}\PYG{p}{::} latex

    \PYGZbs{}setlength\PYGZob{}\PYGZbs{}parindent\PYGZcb{}\PYGZob{}0pt\PYGZcb{}
\end{Verbatim}



On peut lui passer les options suivantes :
\begin{quote}\begin{description}
\item[{file}] \leavevmode
fichier à inclure

\item[{url}] \leavevmode
l'URL d'une page à inclure

\item[{encoding}] \leavevmode
l'encodage du fichier ou de la page inclus

\end{description}\end{quote}


\section{Sidebar}
\label{rtd/instruction_base/sidebar:sidebar}\label{rtd/instruction_base/sidebar::doc}
\index{sidebar}\index{directive!sidebar}\begin{tcolorbox}
\begin{minipage}{0.95\linewidth}
\textbf{The sidebar}

\medskip


\includegraphics{Open-Wide_php_logo.png}

\emph{Above} CH'IEN THE CREATIVE, HEAVEN

\emph{Below} CH'IEN THE CREATIVE, HEAVEN
\end{minipage}
\end{tcolorbox}

The first hexagram is made up of six unbroken lines. These unbroken lines stand for the primal power, which is light-giving, active, strong, and of the spirit. The hexagram is consistently strong in character, and since it is without weakness, its essence is power or energy. Its image is heaven. Its energy is represented as unrestricted by any fixed conditions in space and is therefore conceived of as motion. Time is regarded as the basis of this motion. Thus the hexagram includes also the power of time and the power of persisting in time, that is, duration.

\begin{Verbatim}[commandchars=\\\{\}]
\PYG{p}{..} \PYG{o+ow}{sidebar}\PYG{p}{::} The sidebar

\PYG{p}{    ..} \PYG{o+ow}{image}\PYG{p}{::} /\PYGZus{}static/images/Open\PYGZhy{}Wide\PYGZus{}php\PYGZus{}logo.png

    \PYG{g+ge}{*Above*} CH\PYGZsq{}IEN THE CREATIVE, HEAVEN

    \PYG{g+ge}{*Below*} CH\PYGZsq{}IEN THE CREATIVE, HEAVEN

The first hexagram is made up of six unbroken lines. These unbroken lines stand for the primal power, which is light\PYGZhy{}giving, active, strong, and of the spirit. The hexagram is consistently strong in character, and since it is without weakness, its essence is power or energy. Its image is heaven. Its energy is represented as unrestricted by any fixed conditions in space and is therefore conceived of as motion. Time is regarded as the basis of this motion. Thus the hexagram includes also the power of time and the power of persisting in time, that is, duration.
\end{Verbatim}


\subsection{Code with Sidebar}
\label{rtd/instruction_base/sidebar:code-with-sidebar}\begin{tcolorbox}
\begin{minipage}{0.95\linewidth}
\textbf{A code example}

\medskip


With a sidebar on the right.
\end{minipage}
\end{tcolorbox}

\begin{Verbatim}[commandchars=\\\{\},numbers=left,firstnumber=1,stepnumber=1]
\PYG{c}{\PYGZsh{} \PYGZhy{}*\PYGZhy{} coding: utf\PYGZhy{}8 \PYGZhy{}*\PYGZhy{}}
\PYG{c}{\PYGZsh{}}
\PYG{c}{\PYGZsh{} Sphinx documentation build configuration file, created by}
\PYG{c}{\PYGZsh{} sphinx\PYGZhy{}quickstart on Wed Jul 22 14:43:21 2015.}
\PYG{c}{\PYGZsh{}}
\PYG{c}{\PYGZsh{} This file is execfile()d with the current directory set to its}
\PYG{c}{\PYGZsh{} containing dir.}
\PYG{c}{\PYGZsh{}}
\PYG{c}{\PYGZsh{} Note that not all possible configuration values are present in this}
\PYG{c}{\PYGZsh{} autogenerated file.}
\end{Verbatim}

\begin{Verbatim}[commandchars=\\\{\}]
\PYG{p}{..} \PYG{o+ow}{sidebar}\PYG{p}{::} A code example

    With a sidebar on the right.

\PYG{p}{..} \PYG{o+ow}{literalinclude}\PYG{p}{::} /conf.py
    \PYG{n+nc}{:language:} \PYG{n+nf}{python}
    \PYG{n+nc}{:linenos:}
    \PYG{n+nc}{:lines:} \PYG{n+nf}{1\PYGZhy{}10}
\end{Verbatim}


\section{Substitutions}
\label{rtd/instruction_base/substitutions::doc}\label{rtd/instruction_base/substitutions:substitutions}
\index{substitution}
Les substitutions permettent de remplacer un marqueur par du texte :

\begin{Verbatim}[commandchars=\\\{\}]
\PYG{c+cp}{.. \textbar{}substitution\PYGZus{}text\textbar{} directive type:: data directive block}
\end{Verbatim}

Par défault, il est existe trois substitutions :
\begin{itemize}
\item {} 
\code{\textbar{}release\textbar{}}

\item {} 
\code{\textbar{}version\textbar{}}

\item {} 
\code{\textbar{}today\textbar{}}

\end{itemize}


\subsection{Insertion d'images dans du texte}
\label{rtd/instruction_base/substitutions:insertion-d-images-dans-du-texte}\begin{tcolorbox}
\begin{minipage}{0.95\linewidth}
\textbf{Rendu}

\medskip


West led the \includegraphics{face-angel.png} 3, covered by dummy's \includegraphics{face-angel.png} Q, East's \includegraphics{face-angel.png} K,
and trumped in hand with the \includegraphics{face-embarrassed.png} 2.
\begin{itemize}
\item {} 
\includegraphics{flag-red.png} means stop.

\item {} 
\includegraphics{flag-green.png} means go.

\item {} 
\includegraphics{flag-yellow.png} means go really fast.

\end{itemize}

\includegraphics{face-worried.png} is the official symbol of \href{http://www.poee.org/}{POEE}.
\end{minipage}
\end{tcolorbox}

\begin{Verbatim}[commandchars=\\\{\}]
West led the \textbar{}H\textbar{} 3, covered by dummy\PYGZsq{}s \textbar{}H\textbar{} Q, East\PYGZsq{}s \textbar{}H\textbar{} K,
and trumped in hand with the \textbar{}S\textbar{} 2.

\PYG{p}{..} \PYG{n+nt}{\textbar{}H\textbar{}} \PYG{o+ow}{image}\PYG{p}{::} /\PYGZus{}static/images/face\PYGZhy{}angel.png
    \PYG{n+nc}{:height:} \PYG{n+nf}{11}
    \PYG{n+nc}{:width:} \PYG{n+nf}{11}
\PYG{p}{..} \PYG{n+nt}{\textbar{}S\textbar{}} \PYG{o+ow}{image}\PYG{p}{::} /\PYGZus{}static/images/face\PYGZhy{}embarrassed.png
    \PYG{n+nc}{:height:} \PYG{n+nf}{11}
    \PYG{n+nc}{:width:} \PYG{n+nf}{11}

\PYG{l+m}{*} \textbar{}Red light\textbar{} means stop.
\PYG{l+m}{*} \textbar{}Green light\textbar{} means go.
\PYG{l+m}{*} \textbar{}Yellow light\textbar{} means go really fast.

\PYG{p}{..} \PYG{n+nt}{\textbar{}Red light\textbar{}}    \PYG{o+ow}{image}\PYG{p}{::} /\PYGZus{}static/images/flag\PYGZhy{}red.png
\PYG{p}{..} \PYG{n+nt}{\textbar{}Green light\textbar{}}  \PYG{o+ow}{image}\PYG{p}{::} /\PYGZus{}static/images/flag\PYGZhy{}green.png
\PYG{p}{..} \PYG{n+nt}{\textbar{}Yellow light\textbar{}} \PYG{o+ow}{image}\PYG{p}{::} /\PYGZus{}static/images/flag\PYGZhy{}yellow.png

\textbar{}\PYGZhy{}\PYGZgt{}\PYGZlt{}\PYGZhy{}\textbar{} is the official symbol of POEE\PYGZus{}.

\PYG{p}{..} \PYG{n+nt}{\textbar{}\PYGZhy{}\PYGZgt{}\PYGZlt{}\PYGZhy{}\textbar{}} \PYG{o+ow}{image}\PYG{p}{::} /\PYGZus{}static/images/face\PYGZhy{}worried.png
\PYG{p}{..} \PYG{n+nt}{\PYGZus{}POEE:} http://www.poee.org/
\end{Verbatim}




\subsection{Insérer du code HTML}
\label{rtd/instruction_base/substitutions:inserer-du-code-html}\begin{tcolorbox}
\begin{minipage}{0.95\linewidth}
\textbf{Rendu}

\medskip


saut  de  ligne
\end{minipage}
\end{tcolorbox}

\begin{Verbatim}[commandchars=\\\{\}]
\PYG{p}{..} \PYG{n+nt}{\textbar{}br\textbar{}} \PYG{o+ow}{raw}\PYG{p}{::} html

   \PYGZlt{}br /\PYGZgt{}

saut \textbar{}br\textbar{} de \textbar{}br\textbar{} ligne
\end{Verbatim}




\subsection{Remplacement par du texte}
\label{rtd/instruction_base/substitutions:remplacement-par-du-texte}\begin{tcolorbox}
\begin{minipage}{0.95\linewidth}
\textbf{Rendu}

\medskip


Instead ...  use \code{image} \href{http://docutils.sourceforge.net/doc...}{\emph{more in} \textbf{reST} \emph{directives manual}}
\end{minipage}
\end{tcolorbox}

\begin{Verbatim}[commandchars=\\\{\}]
\PYG{p}{..} \PYG{n+nt}{\textbar{}more\PYGZhy{}doc\textbar{}} \PYG{o+ow}{replace}\PYG{p}{::}  \PYG{g+ge}{*more in*} \PYG{g+gs}{**reST**} \PYG{g+ge}{*directives manual*}

\PYG{p}{..} \PYG{n+nt}{\PYGZus{}more\PYGZhy{}doc:} http://docutils.sourceforge.net/doc...


Instead ...  use \PYG{l+s}{{}`{}`}\PYG{l+s}{image}\PYG{l+s}{{}`{}`} \textbar{}more\PYGZhy{}doc\textbar{}\PYGZus{}
\end{Verbatim}


\section{Tableaux}
\label{rtd/instruction_base/tableaux::doc}\label{rtd/instruction_base/tableaux:tableaux}
\index{tableau}

\subsection{Tableaux simple}
\label{rtd/instruction_base/tableaux:tableaux-simple}\label{rtd/instruction_base/tableaux:index-0}
\index{tableau!tableau simple}\begin{tcolorbox}
\begin{minipage}{0.95\linewidth}
\textbf{Rendu}

\medskip

\begin{table}[H]
\centering

\begin{tabulary}{\linewidth}{|L|L|}
\hline

aA
 & 
bB
\\
\hline
cC
 & 
dD
\\
\hline\end{tabulary}

\end{table}

\end{minipage}
\end{tcolorbox}

\begin{Verbatim}[commandchars=\\\{\}]
==  ==
aA  bB
cC  dD
==  ==
\end{Verbatim}


\begin{tcolorbox}
\begin{minipage}{0.95\linewidth}
\textbf{Rendu}

\medskip

\begin{table}[H]
\centering

\begin{tabulary}{\linewidth}{|L|L|}
\hline
\headcol \textsf{\relax\textcolor{white}{
Vokal
}} & \textsf{\relax\textcolor{white}{
Umlaut
}}\\
\hline
aA
 & 
äÄ
\\
\hline
oO
 & 
öÖ
\\
\hline\end{tabulary}

\end{table}

\end{minipage}
\end{tcolorbox}

\begin{Verbatim}[commandchars=\\\{\}]
=====  ======
Vokal  Umlaut
=====  ======
aA     äÄ
oO     öÖ
=====  ======
\end{Verbatim}


\begin{tcolorbox}
\begin{minipage}{0.95\linewidth}
\textbf{Rendu}

\medskip

\begin{table}[H]
\centering

\begin{tabulary}{\linewidth}{|L|L|L|}
\hline
\headcol  \multicolumn{2}{|l|}{\textsf{\relax\textcolor{white}{
Inputs
}}} & \textsf{\relax\textcolor{white}{
Output
}}\\
\hline\headcol \textsf{\relax\textcolor{white}{
A
}} & \textsf{\relax\textcolor{white}{
B
}} & \textsf{\relax\textcolor{white}{
A or B
}}\\
\hline \multicolumn{2}{|l|}{
False
} & 
False
\\
\hline
True
 & 
False
 & 
True
\\
\hline
False
 & 
True
 & 
True
\\
\hline \multicolumn{2}{|l|}{
True
} & 
True
\\
\hline\end{tabulary}

\end{table}

\end{minipage}
\end{tcolorbox}

\begin{Verbatim}[commandchars=\\\{\}]
=====  =====  ======
Inputs        Output
\PYGZhy{}\PYGZhy{}\PYGZhy{}\PYGZhy{}\PYGZhy{}\PYGZhy{}\PYGZhy{}\PYGZhy{}\PYGZhy{}\PYGZhy{}\PYGZhy{}\PYGZhy{}  \PYGZhy{}\PYGZhy{}\PYGZhy{}\PYGZhy{}\PYGZhy{}\PYGZhy{}
  A      B    A or B
=====  =====  ======
False         False
\PYGZhy{}\PYGZhy{}\PYGZhy{}\PYGZhy{}\PYGZhy{}\PYGZhy{}\PYGZhy{}\PYGZhy{}\PYGZhy{}\PYGZhy{}\PYGZhy{}\PYGZhy{}  \PYGZhy{}\PYGZhy{}\PYGZhy{}\PYGZhy{}\PYGZhy{}\PYGZhy{}
True   False  True
False  True   True
True          True
============  ======
\end{Verbatim}


\begin{tcolorbox}
\begin{minipage}{0.95\linewidth}
\textbf{Rendu}

\medskip

\begin{table}[H]
\centering

\begin{tabular}{|p{0.475\linewidth}|p{0.475\linewidth}|}
\hline
\begin{enumerate}
\item {} 
Hallo

\end{enumerate}
 & 
\begin{DUlineblock}{0em}
\item[] blah blah blah
blah blah blah
blah
\item[] blah blah
\end{DUlineblock}
\\
\hline\begin{enumerate}
\setcounter{enumi}{1}
\item {} 
Here

\end{enumerate}
 & 
We can wrap the
text in source
\\
\hline\begin{enumerate}
\setcounter{enumi}{31}
\item {} 
There

\end{enumerate}
 & 
\textbf{aha}
\\
\hline\end{tabular}

\end{table}

\end{minipage}
\end{tcolorbox}

\begin{Verbatim}[commandchars=\\\{\}]
===========  ================
\PYG{l+m}{1.} Hallo     \textbar{} blah blah blah
               blah blah blah
               blah
             \PYG{o}{\textbar{}} blah blah
\PYG{l+m}{2.} Here      We can wrap the
             text in source
\PYG{l+m}{32.} There    \PYG{g+gs}{**aha**}
===========  ================
\end{Verbatim}




\subsection{Grilles}
\label{rtd/instruction_base/tableaux:grilles}
\index{tableau!grille}\begin{table}[H]
\centering

\begin{tabulary}{\linewidth}{|L|L|L|}
\hline
\headcol \textsf{\relax\textcolor{white}{
Header 1
}} & \textsf{\relax\textcolor{white}{
Header 2
}} & \textsf{\relax\textcolor{white}{
Header 3
}}\\
\hline
body row 1
 & 
column 2
 & 
column 3
\\
\hline
body row 1
 & 
column 2
 & 
column 3
\\
\hline
body row 1
 & 
column 2
 & 
column 3
\\
\hline
body row 1
 & 
column 2
 & 
column 3
\\
\hline\end{tabulary}

\end{table}


\begin{Verbatim}[commandchars=\\\{\}]
+\PYGZhy{}\PYGZhy{}\PYGZhy{}\PYGZhy{}\PYGZhy{}\PYGZhy{}\PYGZhy{}\PYGZhy{}\PYGZhy{}\PYGZhy{}\PYGZhy{}\PYGZhy{}+\PYGZhy{}\PYGZhy{}\PYGZhy{}\PYGZhy{}\PYGZhy{}\PYGZhy{}\PYGZhy{}\PYGZhy{}\PYGZhy{}\PYGZhy{}\PYGZhy{}\PYGZhy{}+\PYGZhy{}\PYGZhy{}\PYGZhy{}\PYGZhy{}\PYGZhy{}\PYGZhy{}\PYGZhy{}\PYGZhy{}\PYGZhy{}\PYGZhy{}\PYGZhy{}+
\PYG{o}{\textbar{}} Header 1   \textbar{} Header 2   \textbar{} Header 3  \textbar{}
+============+============+===========+
\PYG{o}{\textbar{}} body row 1 \textbar{} column 2   \textbar{} column 3  \textbar{}
+\PYGZhy{}\PYGZhy{}\PYGZhy{}\PYGZhy{}\PYGZhy{}\PYGZhy{}\PYGZhy{}\PYGZhy{}\PYGZhy{}\PYGZhy{}\PYGZhy{}\PYGZhy{}+\PYGZhy{}\PYGZhy{}\PYGZhy{}\PYGZhy{}\PYGZhy{}\PYGZhy{}\PYGZhy{}\PYGZhy{}\PYGZhy{}\PYGZhy{}\PYGZhy{}\PYGZhy{}+\PYGZhy{}\PYGZhy{}\PYGZhy{}\PYGZhy{}\PYGZhy{}\PYGZhy{}\PYGZhy{}\PYGZhy{}\PYGZhy{}\PYGZhy{}\PYGZhy{}+
\PYG{o}{\textbar{}} body row 1 \textbar{} column 2   \textbar{} column 3  \textbar{}
+\PYGZhy{}\PYGZhy{}\PYGZhy{}\PYGZhy{}\PYGZhy{}\PYGZhy{}\PYGZhy{}\PYGZhy{}\PYGZhy{}\PYGZhy{}\PYGZhy{}\PYGZhy{}+\PYGZhy{}\PYGZhy{}\PYGZhy{}\PYGZhy{}\PYGZhy{}\PYGZhy{}\PYGZhy{}\PYGZhy{}\PYGZhy{}\PYGZhy{}\PYGZhy{}\PYGZhy{}+\PYGZhy{}\PYGZhy{}\PYGZhy{}\PYGZhy{}\PYGZhy{}\PYGZhy{}\PYGZhy{}\PYGZhy{}\PYGZhy{}\PYGZhy{}\PYGZhy{}+
\PYG{o}{\textbar{}} body row 1 \textbar{} column 2   \textbar{} column 3  \textbar{}
+\PYGZhy{}\PYGZhy{}\PYGZhy{}\PYGZhy{}\PYGZhy{}\PYGZhy{}\PYGZhy{}\PYGZhy{}\PYGZhy{}\PYGZhy{}\PYGZhy{}\PYGZhy{}+\PYGZhy{}\PYGZhy{}\PYGZhy{}\PYGZhy{}\PYGZhy{}\PYGZhy{}\PYGZhy{}\PYGZhy{}\PYGZhy{}\PYGZhy{}\PYGZhy{}\PYGZhy{}+\PYGZhy{}\PYGZhy{}\PYGZhy{}\PYGZhy{}\PYGZhy{}\PYGZhy{}\PYGZhy{}\PYGZhy{}\PYGZhy{}\PYGZhy{}\PYGZhy{}+
\PYG{o}{\textbar{}} body row 1 \textbar{} column 2   \textbar{} column 3  \textbar{}
+\PYGZhy{}\PYGZhy{}\PYGZhy{}\PYGZhy{}\PYGZhy{}\PYGZhy{}\PYGZhy{}\PYGZhy{}\PYGZhy{}\PYGZhy{}\PYGZhy{}\PYGZhy{}+\PYGZhy{}\PYGZhy{}\PYGZhy{}\PYGZhy{}\PYGZhy{}\PYGZhy{}\PYGZhy{}\PYGZhy{}\PYGZhy{}\PYGZhy{}\PYGZhy{}\PYGZhy{}+\PYGZhy{}\PYGZhy{}\PYGZhy{}\PYGZhy{}\PYGZhy{}\PYGZhy{}\PYGZhy{}\PYGZhy{}\PYGZhy{}\PYGZhy{}\PYGZhy{}+
\end{Verbatim}


\bigskip\hrule{}\bigskip

\begin{table}[H]
\centering

\begin{tabular}{|p{0.317\linewidth}|p{0.317\linewidth}|p{0.317\linewidth}|}
\hline
\headcol \textsf{\relax\textcolor{white}{
Header
}} &  \multicolumn{2}{l|}{\textsf{\relax\textcolor{white}{
Header with 2 cols
}}}\\
\hline
A
 & 
Lists:
 & 
\textbf{C}
\\
\hline
B:

\begin{OriginalVerbatim}[commandchars=\\\{\}]
\PYG{g+ge}{*hey*}
\end{OriginalVerbatim}
 & \begin{itemize}
\item {} 
aha

\item {} 
yes

\end{itemize}
\begin{enumerate}
\item {} 
hi

\end{enumerate}
 & 
\begin{DUlineblock}{0em}
\item[] a block
of text
\item[] a break
\end{DUlineblock}
\\
\hline\end{tabular}

\end{table}


\begin{Verbatim}[commandchars=\\\{\}]
+\PYGZhy{}\PYGZhy{}\PYGZhy{}\PYGZhy{}\PYGZhy{}\PYGZhy{}\PYGZhy{}\PYGZhy{}+\PYGZhy{}\PYGZhy{}\PYGZhy{}\PYGZhy{}\PYGZhy{}\PYGZhy{}\PYGZhy{}\PYGZhy{}+\PYGZhy{}\PYGZhy{}\PYGZhy{}\PYGZhy{}\PYGZhy{}\PYGZhy{}\PYGZhy{}\PYGZhy{}\PYGZhy{}\PYGZhy{}\PYGZhy{}+
\PYG{o}{\textbar{}} Header \textbar{} Header with 2 cols \textbar{}
+========+========+===========+
\PYG{o}{\textbar{}} A      \textbar{} Lists: \textbar{} \PYG{g+gs}{**C**}     \textbar{}
+\PYGZhy{}\PYGZhy{}\PYGZhy{}\PYGZhy{}\PYGZhy{}\PYGZhy{}\PYGZhy{}\PYGZhy{}+        +\PYGZhy{}\PYGZhy{}\PYGZhy{}\PYGZhy{}\PYGZhy{}\PYGZhy{}\PYGZhy{}\PYGZhy{}\PYGZhy{}\PYGZhy{}\PYGZhy{}+
\PYG{o}{\textbar{}} B::    \textbar{} \PYGZhy{} aha  \textbar{} \textbar{} a block \textbar{}
\PYG{o}{\textbar{}}        \textbar{} \PYGZhy{} yes  \textbar{}   of text \textbar{}
\PYG{o}{\textbar{}}  \PYG{g+ge}{*hey*} \textbar{}        \textbar{} \textbar{} a break \textbar{}
\PYG{o}{\textbar{}}        \textbar{} \PYGZsh{}. hi  \textbar{}           \textbar{}
+\PYGZhy{}\PYGZhy{}\PYGZhy{}\PYGZhy{}\PYGZhy{}\PYGZhy{}\PYGZhy{}\PYGZhy{}+\PYGZhy{}\PYGZhy{}\PYGZhy{}\PYGZhy{}\PYGZhy{}\PYGZhy{}\PYGZhy{}\PYGZhy{}+\PYGZhy{}\PYGZhy{}\PYGZhy{}\PYGZhy{}\PYGZhy{}\PYGZhy{}\PYGZhy{}\PYGZhy{}\PYGZhy{}\PYGZhy{}\PYGZhy{}+
\end{Verbatim}


\subsection{CSV}
\label{rtd/instruction_base/tableaux:csv}
\index{tableau!CSV}\index{directive!csv-table}\begin{table}[H]
\centering


\begin{threeparttable}
\capstart\caption{Balance Sheet}
\label{rtd/instruction_base/tableaux:index-3}\label{rtd/instruction_base/tableaux:id1}
\begin{tabulary}{\linewidth}{|L|L|L|L|}
\hline
\headcol \textsf{\relax\textcolor{white}{
Description
}} & \textsf{\relax\textcolor{white}{
In
}} & \textsf{\relax\textcolor{white}{
Out
}} & \textsf{\relax\textcolor{white}{
Balance
}}\\
\hline
Travel
 &  & 
230.00
 & 
-230.00
\\
\hline
Fees
 &  & 
400.00
 & 
-630.00
\\
\hline
Grant
 & 
700.00
 &  & 
70.00
\\
\hline
Train Fare
 &  & 
70.00
 & 
\textbf{0.00}
\\
\hline\end{tabulary}

\end{threeparttable}

\end{table}


\begin{Verbatim}[commandchars=\\\{\}]
\PYG{p}{..} \PYG{o+ow}{csv\PYGZhy{}table}\PYG{p}{::} Balance Sheet
    \PYG{n+nc}{:header:} \PYG{n+nf}{Description,In,Out,Balance}
    \PYG{n+nc}{:widths:} \PYG{n+nf}{20, 10, 10, 10}
    \PYG{n+nc}{:stub\PYGZhy{}columns:} \PYG{n+nf}{1}

    Travel,,230.00,\PYGZhy{}230.00
    Fees,,400.00,\PYGZhy{}630.00
    Grant,700.00,,70.00
    Train Fare,,70.00,\PYG{g+gs}{**0.00**}
\end{Verbatim}

\href{http://docutils.sourceforge.net/docs/ref/rst/directives.html\#id4}{http://docutils.sourceforge.net/docs/ref/rst/directives.html\#id4}


\subsection{Listes}
\label{rtd/instruction_base/tableaux:listes}
\index{tableau!liste}\index{directive!list-table}\begin{table}[H]
\centering


\begin{threeparttable}
\capstart\caption{Weather forecast}
\label{rtd/instruction_base/tableaux:index-4}\label{rtd/instruction_base/tableaux:id2}
\begin{tabulary}{\linewidth}{|L|L|L|L|L|}
\hline
\headcol \textsf{\relax\textcolor{white}{
Day
}} & \textsf{\relax\textcolor{white}{
Min Temp
}} & \textsf{\relax\textcolor{white}{
Max Temp
}} & \textsf{\relax\textcolor{white}{}} & \textsf{\relax\textcolor{white}{
Summary
}}\\
\hline
Monday
 & 
11C
 & 
22C
 & 
\includegraphics{sunny.png}
 & 
A clear day with lots of sunshine.
However, the strong breeze will bring
down the temperatures.
\\
\hline
Tuesday
 & 
9C
 & 
10C
 & 
\includegraphics{rain.png}
 & 
Cloudy with rain, across many northern regions.
Clear spells across most of Scotland and Northern Ireland,
but rain reaching the far northwest.
\\
\hline\end{tabulary}

\end{threeparttable}

\end{table}


\begin{Verbatim}[commandchars=\\\{\}]
\PYG{p}{..} \PYG{o+ow}{list\PYGZhy{}table}\PYG{p}{::} Weather forecast
    \PYG{n+nc}{:header\PYGZhy{}rows:} \PYG{n+nf}{1}
    \PYG{n+nc}{:widths:} \PYG{n+nf}{7 7 7 7 60}
    \PYG{n+nc}{:stub\PYGZhy{}columns:} \PYG{n+nf}{1}

    \PYG{l+m}{*}  \PYGZhy{}  Day
       \PYG{l+m}{\PYGZhy{}}  Min Temp
       \PYG{l+m}{\PYGZhy{}}  Max Temp
       \PYGZhy{}
       \PYG{l+m}{\PYGZhy{}}  Summary
    \PYG{l+m}{*}  \PYGZhy{}  Monday
       \PYG{l+m}{\PYGZhy{}}  11C
       \PYG{l+m}{\PYGZhy{}}  22C
       \PYG{l+m}{\PYGZhy{}}  .. image:: /\PYGZus{}static/images/sunny.png
             \PYG{n+nc}{:width:} \PYG{n+nf}{30}

       \PYG{l+m}{\PYGZhy{}}  A clear day with lots of sunshine.
          However, the strong breeze will bring
          down the temperatures.
    \PYG{l+m}{*}  \PYGZhy{}  Tuesday
       \PYG{l+m}{\PYGZhy{}}  9C
       \PYG{l+m}{\PYGZhy{}}  10C
       \PYG{l+m}{\PYGZhy{}}  .. image:: /\PYGZus{}static/images/rain.png
             \PYG{n+nc}{:width:} \PYG{n+nf}{30}

       \PYG{l+m}{\PYGZhy{}}  Cloudy with rain, across many northern regions.
          Clear spells across most of Scotland and Northern Ireland,
          but rain reaching the far northwest.
\end{Verbatim}

\href{http://docutils.sourceforge.net/docs/ref/rst/directives.html\#list-table}{http://docutils.sourceforge.net/docs/ref/rst/directives.html\#list-table}


\subsection{Large}
\label{rtd/instruction_base/tableaux:large}\begin{table}[H]
\centering

\begin{tabulary}{\linewidth}{|L|L|L|L|L|L|L|L|L|L|L|L|}
\hline
\headcol \textsf{\relax\textcolor{white}{
Header 1
}} & \textsf{\relax\textcolor{white}{
Header 2
}} & \textsf{\relax\textcolor{white}{
Header 3
}} & \textsf{\relax\textcolor{white}{
Header 1
}} & \textsf{\relax\textcolor{white}{
Header 2
}} & \textsf{\relax\textcolor{white}{
Header 3
}} & \textsf{\relax\textcolor{white}{
Header 1
}} & \textsf{\relax\textcolor{white}{
Header 2
}} & \textsf{\relax\textcolor{white}{
Header 3
}} & \textsf{\relax\textcolor{white}{
Header 1
}} & \textsf{\relax\textcolor{white}{
Header 2
}} & \textsf{\relax\textcolor{white}{
Header 3
}}\\
\hline
body row 1
 & 
column 2
 & 
column 3
 & 
body row 1
 & 
column 2
 & 
column 3
 & 
body row 1
 & 
column 2
 & 
column 3
 & 
body row 1
 & 
column 2
 & 
column 3
\\
\hline
body row 1
 & 
column 2
 & 
column 3
 & 
body row 1
 & 
column 2
 & 
column 3
 & 
body row 1
 & 
column 2
 & 
column 3
 & 
body row 1
 & 
column 2
 & 
column 3
\\
\hline
body row 1
 & 
column 2
 & 
column 3
 & 
body row 1
 & 
column 2
 & 
column 3
 & 
body row 1
 & 
column 2
 & 
column 3
 & 
body row 1
 & 
column 2
 & 
column 3
\\
\hline
body row 1
 & 
column 2
 & 
column 3
 & 
body row 1
 & 
column 2
 & 
column 3
 & 
body row 1
 & 
column 2
 & 
column 3
 & 
body row 1
 & 
column 2
 & 
column 3
\\
\hline
body row 1
 & 
column 2
 & 
column 3
 & 
body row 1
 & 
column 2
 & 
column 3
 & 
body row 1
 & 
column 2
 & 
column 3
 & 
body row 1
 & 
column 2
 & 
column 3
\\
\hline\end{tabulary}

\end{table}



\subsection{Très grand}
\label{rtd/instruction_base/tableaux:tres-grand}
\begin{longtable}{|l|l|l|l|l|l|l|l|l|l|l|l|}
\hline
\headcol \textsf{\relax\textcolor{white}{
Header 1
}} & \textsf{\relax\textcolor{white}{
Header 2
}} & \textsf{\relax\textcolor{white}{
Header 3
}} & \textsf{\relax\textcolor{white}{
Header 1
}} & \textsf{\relax\textcolor{white}{
Header 2
}} & \textsf{\relax\textcolor{white}{
Header 3
}} & \textsf{\relax\textcolor{white}{
Header 1
}} & \textsf{\relax\textcolor{white}{
Header 2
}} & \textsf{\relax\textcolor{white}{
Header 3
}} & \textsf{\relax\textcolor{white}{
Header 1
}} & \textsf{\relax\textcolor{white}{
Header 2
}} & \textsf{\relax\textcolor{white}{
Header 3
}}\\
\hline\endfirsthead

\hline \multicolumn{12}{|r|}{{\textsf{Suite de la page précédente}}} \\ \hline
\hline
\headcol \textsf{\relax\textcolor{white}{
Header 1
}} & \textsf{\relax\textcolor{white}{
Header 2
}} & \textsf{\relax\textcolor{white}{
Header 3
}} & \textsf{\relax\textcolor{white}{
Header 1
}} & \textsf{\relax\textcolor{white}{
Header 2
}} & \textsf{\relax\textcolor{white}{
Header 3
}} & \textsf{\relax\textcolor{white}{
Header 1
}} & \textsf{\relax\textcolor{white}{
Header 2
}} & \textsf{\relax\textcolor{white}{
Header 3
}} & \textsf{\relax\textcolor{white}{
Header 1
}} & \textsf{\relax\textcolor{white}{
Header 2
}} & \textsf{\relax\textcolor{white}{
Header 3
}}\\
\hline\endhead

\hline \multicolumn{12}{|r|}{{\textsf{Suite sur la page suivante}}} \\ \hline
\endfoot

\endlastfoot


body row 1
 & 
column 2
 & 
column 3
 & 
body row 1
 & 
column 2
 & 
column 3
 & 
body row 1
 & 
column 2
 & 
column 3
 & 
body row 1
 & 
column 2
 & 
column 3
\\
\hline
body row 1
 & 
column 2
 & 
column 3
 & 
body row 1
 & 
column 2
 & 
column 3
 & 
body row 1
 & 
column 2
 & 
column 3
 & 
body row 1
 & 
column 2
 & 
column 3
\\
\hline
body row 1
 & 
column 2
 & 
column 3
 & 
body row 1
 & 
column 2
 & 
column 3
 & 
body row 1
 & 
column 2
 & 
column 3
 & 
body row 1
 & 
column 2
 & 
column 3
\\
\hline
body row 1
 & 
column 2
 & 
column 3
 & 
body row 1
 & 
column 2
 & 
column 3
 & 
body row 1
 & 
column 2
 & 
column 3
 & 
body row 1
 & 
column 2
 & 
column 3
\\
\hline
body row 1
 & 
column 2
 & 
column 3
 & 
body row 1
 & 
column 2
 & 
column 3
 & 
body row 1
 & 
column 2
 & 
column 3
 & 
body row 1
 & 
column 2
 & 
column 3
\\
\hline
body row 1
 & 
column 2
 & 
column 3
 & 
body row 1
 & 
column 2
 & 
column 3
 & 
body row 1
 & 
column 2
 & 
column 3
 & 
body row 1
 & 
column 2
 & 
column 3
\\
\hline
body row 1
 & 
column 2
 & 
column 3
 & 
body row 1
 & 
column 2
 & 
column 3
 & 
body row 1
 & 
column 2
 & 
column 3
 & 
body row 1
 & 
column 2
 & 
column 3
\\
\hline
body row 1
 & 
column 2
 & 
column 3
 & 
body row 1
 & 
column 2
 & 
column 3
 & 
body row 1
 & 
column 2
 & 
column 3
 & 
body row 1
 & 
column 2
 & 
column 3
\\
\hline
body row 1
 & 
column 2
 & 
column 3
 & 
body row 1
 & 
column 2
 & 
column 3
 & 
body row 1
 & 
column 2
 & 
column 3
 & 
body row 1
 & 
column 2
 & 
column 3
\\
\hline
body row 1
 & 
column 2
 & 
column 3
 & 
body row 1
 & 
column 2
 & 
column 3
 & 
body row 1
 & 
column 2
 & 
column 3
 & 
body row 1
 & 
column 2
 & 
column 3
\\
\hline
body row 1
 & 
column 2
 & 
column 3
 & 
body row 1
 & 
column 2
 & 
column 3
 & 
body row 1
 & 
column 2
 & 
column 3
 & 
body row 1
 & 
column 2
 & 
column 3
\\
\hline
body row 1
 & 
column 2
 & 
column 3
 & 
body row 1
 & 
column 2
 & 
column 3
 & 
body row 1
 & 
column 2
 & 
column 3
 & 
body row 1
 & 
column 2
 & 
column 3
\\
\hline
body row 1
 & 
column 2
 & 
column 3
 & 
body row 1
 & 
column 2
 & 
column 3
 & 
body row 1
 & 
column 2
 & 
column 3
 & 
body row 1
 & 
column 2
 & 
column 3
\\
\hline
body row 1
 & 
column 2
 & 
column 3
 & 
body row 1
 & 
column 2
 & 
column 3
 & 
body row 1
 & 
column 2
 & 
column 3
 & 
body row 1
 & 
column 2
 & 
column 3
\\
\hline
body row 1
 & 
column 2
 & 
column 3
 & 
body row 1
 & 
column 2
 & 
column 3
 & 
body row 1
 & 
column 2
 & 
column 3
 & 
body row 1
 & 
column 2
 & 
column 3
\\
\hline
body row 1
 & 
column 2
 & 
column 3
 & 
body row 1
 & 
column 2
 & 
column 3
 & 
body row 1
 & 
column 2
 & 
column 3
 & 
body row 1
 & 
column 2
 & 
column 3
\\
\hline
body row 1
 & 
column 2
 & 
column 3
 & 
body row 1
 & 
column 2
 & 
column 3
 & 
body row 1
 & 
column 2
 & 
column 3
 & 
body row 1
 & 
column 2
 & 
column 3
\\
\hline
body row 1
 & 
column 2
 & 
column 3
 & 
body row 1
 & 
column 2
 & 
column 3
 & 
body row 1
 & 
column 2
 & 
column 3
 & 
body row 1
 & 
column 2
 & 
column 3
\\
\hline
body row 1
 & 
column 2
 & 
column 3
 & 
body row 1
 & 
column 2
 & 
column 3
 & 
body row 1
 & 
column 2
 & 
column 3
 & 
body row 1
 & 
column 2
 & 
column 3
\\
\hline
body row 1
 & 
column 2
 & 
column 3
 & 
body row 1
 & 
column 2
 & 
column 3
 & 
body row 1
 & 
column 2
 & 
column 3
 & 
body row 1
 & 
column 2
 & 
column 3
\\
\hline
body row 1
 & 
column 2
 & 
column 3
 & 
body row 1
 & 
column 2
 & 
column 3
 & 
body row 1
 & 
column 2
 & 
column 3
 & 
body row 1
 & 
column 2
 & 
column 3
\\
\hline
body row 1
 & 
column 2
 & 
column 3
 & 
body row 1
 & 
column 2
 & 
column 3
 & 
body row 1
 & 
column 2
 & 
column 3
 & 
body row 1
 & 
column 2
 & 
column 3
\\
\hline
body row 1
 & 
column 2
 & 
column 3
 & 
body row 1
 & 
column 2
 & 
column 3
 & 
body row 1
 & 
column 2
 & 
column 3
 & 
body row 1
 & 
column 2
 & 
column 3
\\
\hline
body row 1
 & 
column 2
 & 
column 3
 & 
body row 1
 & 
column 2
 & 
column 3
 & 
body row 1
 & 
column 2
 & 
column 3
 & 
body row 1
 & 
column 2
 & 
column 3
\\
\hline
body row 1
 & 
column 2
 & 
column 3
 & 
body row 1
 & 
column 2
 & 
column 3
 & 
body row 1
 & 
column 2
 & 
column 3
 & 
body row 1
 & 
column 2
 & 
column 3
\\
\hline
body row 1
 & 
column 2
 & 
column 3
 & 
body row 1
 & 
column 2
 & 
column 3
 & 
body row 1
 & 
column 2
 & 
column 3
 & 
body row 1
 & 
column 2
 & 
column 3
\\
\hline
body row 1
 & 
column 2
 & 
column 3
 & 
body row 1
 & 
column 2
 & 
column 3
 & 
body row 1
 & 
column 2
 & 
column 3
 & 
body row 1
 & 
column 2
 & 
column 3
\\
\hline
body row 1
 & 
column 2
 & 
column 3
 & 
body row 1
 & 
column 2
 & 
column 3
 & 
body row 1
 & 
column 2
 & 
column 3
 & 
body row 1
 & 
column 2
 & 
column 3
\\
\hline
body row 1
 & 
column 2
 & 
column 3
 & 
body row 1
 & 
column 2
 & 
column 3
 & 
body row 1
 & 
column 2
 & 
column 3
 & 
body row 1
 & 
column 2
 & 
column 3
\\
\hline
body row 1
 & 
column 2
 & 
column 3
 & 
body row 1
 & 
column 2
 & 
column 3
 & 
body row 1
 & 
column 2
 & 
column 3
 & 
body row 1
 & 
column 2
 & 
column 3
\\
\hline
body row 1
 & 
column 2
 & 
column 3
 & 
body row 1
 & 
column 2
 & 
column 3
 & 
body row 1
 & 
column 2
 & 
column 3
 & 
body row 1
 & 
column 2
 & 
column 3
\\
\hline
body row 1
 & 
column 2
 & 
column 3
 & 
body row 1
 & 
column 2
 & 
column 3
 & 
body row 1
 & 
column 2
 & 
column 3
 & 
body row 1
 & 
column 2
 & 
column 3
\\
\hline
body row 1
 & 
column 2
 & 
column 3
 & 
body row 1
 & 
column 2
 & 
column 3
 & 
body row 1
 & 
column 2
 & 
column 3
 & 
body row 1
 & 
column 2
 & 
column 3
\\
\hline
body row 1
 & 
column 2
 & 
column 3
 & 
body row 1
 & 
column 2
 & 
column 3
 & 
body row 1
 & 
column 2
 & 
column 3
 & 
body row 1
 & 
column 2
 & 
column 3
\\
\hline
body row 1
 & 
column 2
 & 
column 3
 & 
body row 1
 & 
column 2
 & 
column 3
 & 
body row 1
 & 
column 2
 & 
column 3
 & 
body row 1
 & 
column 2
 & 
column 3
\\
\hline
body row 1
 & 
column 2
 & 
column 3
 & 
body row 1
 & 
column 2
 & 
column 3
 & 
body row 1
 & 
column 2
 & 
column 3
 & 
body row 1
 & 
column 2
 & 
column 3
\\
\hline
body row 1
 & 
column 2
 & 
column 3
 & 
body row 1
 & 
column 2
 & 
column 3
 & 
body row 1
 & 
column 2
 & 
column 3
 & 
body row 1
 & 
column 2
 & 
column 3
\\
\hline
body row 1
 & 
column 2
 & 
column 3
 & 
body row 1
 & 
column 2
 & 
column 3
 & 
body row 1
 & 
column 2
 & 
column 3
 & 
body row 1
 & 
column 2
 & 
column 3
\\
\hline
body row 1
 & 
column 2
 & 
column 3
 & 
body row 1
 & 
column 2
 & 
column 3
 & 
body row 1
 & 
column 2
 & 
column 3
 & 
body row 1
 & 
column 2
 & 
column 3
\\
\hline
body row 1
 & 
column 2
 & 
column 3
 & 
body row 1
 & 
column 2
 & 
column 3
 & 
body row 1
 & 
column 2
 & 
column 3
 & 
body row 1
 & 
column 2
 & 
column 3
\\
\hline\end{longtable}



\section{Titres}
\label{rtd/instruction_base/titre:titres}\label{rtd/instruction_base/titre::doc}
\index{titre}
Les titres sont obtenus en soulignant ou surlignant les titres de section avec des signes de ponctuation, au moins aussi long que le titre.

Par convention nous utilisons les ponctuations suivantes :
\begin{itemize}
\item {} 
\code{\#} surligné pour les parties
\begin{quote}

\begin{Verbatim}[commandchars=\\\{\}]
\PYG{g+gh}{\PYGZsh{}\PYGZsh{}\PYGZsh{}\PYGZsh{}\PYGZsh{}\PYGZsh{}\PYGZsh{}\PYGZsh{}\PYGZsh{}\PYGZsh{}\PYGZsh{}\PYGZsh{}\PYGZsh{}\PYGZsh{}\PYGZsh{}\PYGZsh{}\PYGZsh{}\PYGZsh{}}
\PYG{g+gh}{Titre de la partie}
\PYG{g+gh}{\PYGZsh{}\PYGZsh{}\PYGZsh{}\PYGZsh{}\PYGZsh{}\PYGZsh{}\PYGZsh{}\PYGZsh{}\PYGZsh{}\PYGZsh{}\PYGZsh{}\PYGZsh{}\PYGZsh{}\PYGZsh{}\PYGZsh{}\PYGZsh{}\PYGZsh{}\PYGZsh{}}
\end{Verbatim}
\end{quote}

\item {} 
\code{*} surligné pour les chapitres
\begin{quote}

\begin{Verbatim}[commandchars=\\\{\}]
\PYG{g+gh}{*****************}
\PYG{g+gh}{Titre du chapitre}
\PYG{g+gh}{*****************}
\end{Verbatim}
\end{quote}

\item {} 
\code{=} souligné pour les sections
\begin{quote}

\begin{Verbatim}[commandchars=\\\{\}]
\PYG{g+gh}{Titre de la section}
\PYG{g+gh}{===================}
\end{Verbatim}
\end{quote}

\item {} 
\code{-} souligné pour les sous-sections
\begin{quote}

\begin{Verbatim}[commandchars=\\\{\}]
\PYG{g+gh}{Titre de la sous\PYGZhy{}section}
\PYG{g+gh}{\PYGZhy{}\PYGZhy{}\PYGZhy{}\PYGZhy{}\PYGZhy{}\PYGZhy{}\PYGZhy{}\PYGZhy{}\PYGZhy{}\PYGZhy{}\PYGZhy{}\PYGZhy{}\PYGZhy{}\PYGZhy{}\PYGZhy{}\PYGZhy{}\PYGZhy{}\PYGZhy{}\PYGZhy{}\PYGZhy{}\PYGZhy{}\PYGZhy{}\PYGZhy{}\PYGZhy{}}
\end{Verbatim}
\end{quote}

\item {} 
\code{\textasciicircum{}} souligné pour les sous-sous-sections
\begin{quote}

\begin{Verbatim}[commandchars=\\\{\}]
\PYG{g+gh}{Titre de la sous\PYGZhy{}sous\PYGZhy{}section}
\PYG{g+gh}{\PYGZca{}\PYGZca{}\PYGZca{}\PYGZca{}\PYGZca{}\PYGZca{}\PYGZca{}\PYGZca{}\PYGZca{}\PYGZca{}\PYGZca{}\PYGZca{}\PYGZca{}\PYGZca{}\PYGZca{}\PYGZca{}\PYGZca{}\PYGZca{}\PYGZca{}\PYGZca{}\PYGZca{}\PYGZca{}\PYGZca{}\PYGZca{}\PYGZca{}\PYGZca{}\PYGZca{}\PYGZca{}\PYGZca{}}
\end{Verbatim}
\end{quote}

\item {} 
\code{"} souligné pour paragraphes
\begin{quote}

\begin{Verbatim}[commandchars=\\\{\}]
\PYG{g+gh}{Titre du paragraphe}
\PYG{g+gh}{\PYGZdq{}\PYGZdq{}\PYGZdq{}\PYGZdq{}\PYGZdq{}\PYGZdq{}\PYGZdq{}\PYGZdq{}\PYGZdq{}\PYGZdq{}\PYGZdq{}\PYGZdq{}\PYGZdq{}\PYGZdq{}\PYGZdq{}\PYGZdq{}\PYGZdq{}\PYGZdq{}\PYGZdq{}}
\end{Verbatim}
\end{quote}

\item {} 
\code{.} souligné pour sous-paragraphes
\begin{quote}

\begin{Verbatim}[commandchars=\\\{\}]
\PYG{g+gh}{Titre du sous\PYGZhy{}paragraphe}
\PYG{g+gh}{........................}
\end{Verbatim}
\end{quote}

\end{itemize}


\section{Titre de la partie}
\label{rtd/instruction_base/titre:titre-de-la-partie}

\subsection{Titre du chapitre}
\label{rtd/instruction_base/titre:titre-du-chapitre}

\subsubsection{Titre de la section}
\label{rtd/instruction_base/titre:titre-de-la-section}

\paragraph{Titre de la sous-section}
\label{rtd/instruction_base/titre:titre-de-la-sous-section}

\subparagraph{Titre de la sous-sous-section}
\label{rtd/instruction_base/titre:titre-de-la-sous-sous-section}

\subparagraph{Titre du paragraphe}
\label{rtd/instruction_base/titre:titre-du-paragraphe}

\subparagraph{Titre du sous-paragraphe}
\label{rtd/instruction_base/titre:titre-du-sous-paragraphe}

\section{TOC tree}
\label{rtd/instruction_base/toc-tree:toc-tree}\label{rtd/instruction_base/toc-tree::doc}
\index{table des matières}\index{directive!toctree}
\begin{Verbatim}[commandchars=\\\{\}]
\PYG{p}{..} \PYG{o+ow}{toctree}\PYG{p}{::}
    \PYG{n+nc}{:maxdepth:} \PYG{n+nf}{2}
    \PYG{n+nc}{:numbered:}
    \PYG{n+nc}{:glob:}
    \PYG{n+nc}{:hidden:}

    intro/*
    All about strings \PYGZlt{}strings\PYGZgt{}
    datatypes
    numeric
    (many more documents listed here)
\end{Verbatim}
\begin{quote}\begin{description}
\item[{maxdepth}] \leavevmode
Profondeur à afficher

\item[{numbered}] \leavevmode
Affiche les numéros de chapitres

\item[{glob}] \leavevmode
À utiliser quand on veux inclure un répertoire ou un ensemble de fichiers comme \code{intro/*}

\item[{hidden}] \leavevmode
N'affiche pas le TOC. Les titres apparaissent dans la navigation de gauche (HTML)

\end{description}\end{quote}


\section{Topic!}
\label{rtd/instruction_base/topic:topic}\label{rtd/instruction_base/topic::doc}
\index{topic}\index{directive!topic}\begin{tcolorbox}
\begin{minipage}{0.95\linewidth}
\textbf{Rendu}

\medskip

\begin{tcolorbox}
\begin{minipage}{0.95\linewidth}
\textbf{Topic Title}

\medskip


Subsequent indented lines comprise
the body of the topic, and are
interpreted as body elements.
\end{minipage}
\end{tcolorbox}
\end{minipage}
\end{tcolorbox}

\begin{Verbatim}[commandchars=\\\{\}]
\PYG{p}{..} \PYG{o+ow}{topic}\PYG{p}{::} Topic Title

    Subsequent indented lines comprise
    the body of the topic, and are
    interpreted as body elements.
\end{Verbatim}




\section{Transition}
\label{rtd/instruction_base/transition:transition}\label{rtd/instruction_base/transition::doc}
\index{transition}
Toute répétition de 4 caractères de ponctuation ou plus sera remplacé par une ligne :


\bigskip\hrule{}\bigskip


\begin{Verbatim}[commandchars=\\\{\}]
\PYGZhy{}\PYGZhy{}\PYGZhy{}\PYGZhy{}\PYGZhy{}\PYGZhy{}\PYGZhy{}
\end{Verbatim}


\section{Typographie}
\label{rtd/instruction_base/typographie:typographie}\label{rtd/instruction_base/typographie::doc}
\index{typographie}\begin{tcolorbox}
\begin{minipage}{0.95\linewidth}
\textbf{Rendu}

\medskip


Smaller, 80\% text.
\end{minipage}
\end{tcolorbox}

\begin{Verbatim}[commandchars=\\\{\}]
\PYG{p}{..} \PYG{o+ow}{rst\PYGZhy{}class}\PYG{p}{::} wy\PYGZhy{}text\PYGZhy{}small

    Smaller, 80\PYGZpc{} text.
\end{Verbatim}


\begin{tcolorbox}
\begin{minipage}{0.95\linewidth}
\textbf{Rendu}

\medskip


Larger, 120\% text.
\end{minipage}
\end{tcolorbox}

\begin{Verbatim}[commandchars=\\\{\}]
\PYG{p}{..} \PYG{o+ow}{rst\PYGZhy{}class}\PYG{p}{::} wy\PYGZhy{}text\PYGZhy{}large

    Larger, 120\PYGZpc{} text.
\end{Verbatim}


\begin{tcolorbox}
\begin{minipage}{0.95\linewidth}
\textbf{Rendu}

\medskip


To the left.
\end{minipage}
\end{tcolorbox}

\begin{Verbatim}[commandchars=\\\{\}]
\PYG{p}{..} \PYG{o+ow}{rst\PYGZhy{}class}\PYG{p}{::} wy\PYGZhy{}text\PYGZhy{}left

    To the left.
\end{Verbatim}


\begin{tcolorbox}
\begin{minipage}{0.95\linewidth}
\textbf{Rendu}

\medskip


To the right.
\end{minipage}
\end{tcolorbox}

\begin{Verbatim}[commandchars=\\\{\}]
\PYG{p}{..} \PYG{o+ow}{rst\PYGZhy{}class}\PYG{p}{::} wy\PYGZhy{}text\PYGZhy{}right

    To the right.
\end{Verbatim}


\begin{tcolorbox}
\begin{minipage}{0.95\linewidth}
\textbf{Rendu}

\medskip


To the center.
\end{minipage}
\end{tcolorbox}

\begin{Verbatim}[commandchars=\\\{\}]
\PYG{p}{..} \PYG{o+ow}{rst\PYGZhy{}class}\PYG{p}{::} wy\PYGZhy{}text\PYGZhy{}center

    To the center.
\end{Verbatim}


\begin{tcolorbox}
\begin{minipage}{0.95\linewidth}
\textbf{Rendu}

\medskip


Strikethrough.
\end{minipage}
\end{tcolorbox}

\begin{Verbatim}[commandchars=\\\{\}]
\PYG{p}{..} \PYG{o+ow}{rst\PYGZhy{}class}\PYG{p}{::} wy\PYGZhy{}text\PYGZhy{}strike

    Strikethrough.
\end{Verbatim}


\begin{tcolorbox}
\begin{minipage}{0.95\linewidth}
\textbf{Rendu}

\medskip


Simply applies the normal text color.
\end{minipage}
\end{tcolorbox}

\begin{Verbatim}[commandchars=\\\{\}]
\PYG{p}{..} \PYG{o+ow}{rst\PYGZhy{}class}\PYG{p}{::} wy\PYGZhy{}text\PYGZhy{}neutral

    Simply applies the normal text color.
\end{Verbatim}


\begin{tcolorbox}
\begin{minipage}{0.95\linewidth}
\textbf{Rendu}

\medskip


Info text.
\end{minipage}
\end{tcolorbox}

\begin{Verbatim}[commandchars=\\\{\}]
\PYG{p}{..} \PYG{o+ow}{rst\PYGZhy{}class}\PYG{p}{::} wy\PYGZhy{}text\PYGZhy{}info

    Info text.
\end{Verbatim}


\begin{tcolorbox}
\begin{minipage}{0.95\linewidth}
\textbf{Rendu}

\medskip


Success text.
\end{minipage}
\end{tcolorbox}

\begin{Verbatim}[commandchars=\\\{\}]
\PYG{p}{..} \PYG{o+ow}{rst\PYGZhy{}class}\PYG{p}{::} wy\PYGZhy{}text\PYGZhy{}success

    Success text.
\end{Verbatim}


\begin{tcolorbox}
\begin{minipage}{0.95\linewidth}
\textbf{Rendu}

\medskip


Warning text.
\end{minipage}
\end{tcolorbox}

\begin{Verbatim}[commandchars=\\\{\}]
\PYG{p}{..} \PYG{o+ow}{rst\PYGZhy{}class}\PYG{p}{::} wy\PYGZhy{}text\PYGZhy{}warning

    Warning text.
\end{Verbatim}


\begin{tcolorbox}
\begin{minipage}{0.95\linewidth}
\textbf{Rendu}

\medskip


Danger text.
\end{minipage}
\end{tcolorbox}

\begin{Verbatim}[commandchars=\\\{\}]
\PYG{p}{..} \PYG{o+ow}{rst\PYGZhy{}class}\PYG{p}{::} wy\PYGZhy{}text\PYGZhy{}danger

    Danger text.
\end{Verbatim}




\chapter{Instructions spécifiques à la version Open Wide}
\label{rtd/instruction_openwide:instructions-specifiques-a-la-version-open-wide}\label{rtd/instruction_openwide::doc}

\section{Awesome font}
\label{rtd/instruction_openwide:awesome-font}
\href{http://fortawesome.github.io/Font-Awesome/}{http://fortawesome.github.io/Font-Awesome/}


\subsection{Icône de base}
\label{rtd/instruction_openwide:icone-de-base}\begin{tcolorbox}
\begin{minipage}{0.95\linewidth}
\textbf{Rendu}

\medskip



\end{minipage}
\end{tcolorbox}

\begin{Verbatim}[commandchars=\\\{\}]
\PYG{n+na}{:awesome:}\PYG{n+nv}{{}`camera\PYGZhy{}retro{}`}
\end{Verbatim}




\subsection{Tailles des icônes}
\label{rtd/instruction_openwide:tailles-des-icones}\begin{tcolorbox}
\begin{minipage}{0.95\linewidth}
\textbf{Rendu}

\medskip











\end{minipage}
\end{tcolorbox}

\begin{Verbatim}[commandchars=\\\{\}]
\PYG{n+na}{:awesome:}\PYG{n+nv}{{}`camera\PYGZhy{}retro fa\PYGZhy{}lg{}`}

\PYG{n+na}{:awesome:}\PYG{n+nv}{{}`camera\PYGZhy{}retro fa\PYGZhy{}2x{}`}

\PYG{n+na}{:awesome:}\PYG{n+nv}{{}`camera\PYGZhy{}retro fa\PYGZhy{}3x{}`}

\PYG{n+na}{:awesome:}\PYG{n+nv}{{}`camera\PYGZhy{}retro fa\PYGZhy{}4x{}`}

\PYG{n+na}{:awesome:}\PYG{n+nv}{{}`camera\PYGZhy{}retro fa\PYGZhy{}5x{}`}
\end{Verbatim}




\subsection{Icônes à largeur fixe}
\label{rtd/instruction_openwide:icones-a-largeur-fixe}\begin{tcolorbox}
\begin{minipage}{0.95\linewidth}
\textbf{Rendu}

\medskip


 Home

 Library

 Applications

 Settings
\end{minipage}
\end{tcolorbox}

\begin{Verbatim}[commandchars=\\\{\}]
\PYG{p}{..} \PYG{o+ow}{container}\PYG{p}{::} list\PYGZhy{}group

\PYG{p}{    ..} \PYG{o+ow}{rst\PYGZhy{}class}\PYG{p}{::} list\PYGZhy{}group\PYGZhy{}item

        \PYG{n+na}{:awesome:}\PYG{n+nv}{{}`home fa\PYGZhy{}fw{}`} Home

\PYG{p}{    ..} \PYG{o+ow}{rst\PYGZhy{}class}\PYG{p}{::} list\PYGZhy{}group\PYGZhy{}item


        \PYG{n+na}{:awesome:}\PYG{n+nv}{{}`book fa\PYGZhy{}fw{}`} Library

\PYG{p}{    ..} \PYG{o+ow}{rst\PYGZhy{}class}\PYG{p}{::} list\PYGZhy{}group\PYGZhy{}item


        \PYG{n+na}{:awesome:}\PYG{n+nv}{{}`pencil fa\PYGZhy{}fw{}`} Applications

\PYG{p}{    ..} \PYG{o+ow}{rst\PYGZhy{}class}\PYG{p}{::} list\PYGZhy{}group\PYGZhy{}item

        \PYG{n+na}{:awesome:}\PYG{n+nv}{{}`cog fa\PYGZhy{}fw{}`} Settings
\end{Verbatim}




\subsection{Icônes animés}
\label{rtd/instruction_openwide:icones-animes}\begin{tcolorbox}
\begin{minipage}{0.95\linewidth}
\textbf{Rendu}

\medskip











\end{minipage}
\end{tcolorbox}

\begin{Verbatim}[commandchars=\\\{\}]
\PYG{n+na}{:awesome:}\PYG{n+nv}{{}`spinner fa\PYGZhy{}spin fa\PYGZhy{}2x{}`}

\PYG{n+na}{:awesome:}\PYG{n+nv}{{}`circle\PYGZhy{}o\PYGZhy{}notch fa\PYGZhy{}spin fa\PYGZhy{}2x{}`}

\PYG{n+na}{:awesome:}\PYG{n+nv}{{}`refresh fa\PYGZhy{}spin fa\PYGZhy{}2x{}`}

\PYG{n+na}{:awesome:}\PYG{n+nv}{{}`cog fa\PYGZhy{}spin fa\PYGZhy{}2x{}`}

\PYG{n+na}{:awesome:}\PYG{n+nv}{{}`spinner fa\PYGZhy{}pulse fa\PYGZhy{}2x{}`}
\end{Verbatim}




\subsection{Liste avec icônes}
\label{rtd/instruction_openwide:liste-avec-icones}\begin{tcolorbox}
\begin{minipage}{0.95\linewidth}
\textbf{Rendu}

\medskip

\begin{itemize}
\item {} 
 List icons

\item {} 
 can be used

\item {} 
 as bullet

\item {} 
 in lists

\end{itemize}
\end{minipage}
\end{tcolorbox}

\begin{Verbatim}[commandchars=\\\{\}]
\PYG{p}{..} \PYG{o+ow}{rst\PYGZhy{}class}\PYG{p}{::} fa\PYGZhy{}ul

    \PYG{l+m}{*} \PYG{n+na}{:awesome:}\PYG{n+nv}{{}`check\PYGZhy{}square fa\PYGZhy{}li{}`} List icons
    \PYG{l+m}{*} \PYG{n+na}{:awesome:}\PYG{n+nv}{{}`check\PYGZhy{}square fa\PYGZhy{}li{}`} can be used
    \PYG{l+m}{*} \PYG{n+na}{:awesome:}\PYG{n+nv}{{}`spinner fa\PYGZhy{}spin fa\PYGZhy{}li{}`} as bullet
    \PYG{l+m}{*} \PYG{n+na}{:awesome:}\PYG{n+nv}{{}`square fa\PYGZhy{}li{}`} in lists
\end{Verbatim}




\subsection{Bordures \& alignement d'icônes}
\label{rtd/instruction_openwide:bordures-alignement-d-icones}\begin{tcolorbox}
\begin{minipage}{0.95\linewidth}
\textbf{Rendu}

\medskip



Tomorrow we will run faster, stretch out our arms farther
And then one fine morning — So we beat on, boats against the
current, borne back ceaselessly into the past.
\end{minipage}
\end{tcolorbox}

\begin{Verbatim}[commandchars=\\\{\}]
\PYG{n+na}{:awesome:}\PYG{n+nv}{{}`quote\PYGZhy{}left fa\PYGZhy{}3x pull\PYGZhy{}left fa\PYGZhy{}border{}`}
Tomorrow we will run faster, stretch out our arms farther
And then one fine morning — So we beat on, boats against the
current, borne back ceaselessly into the past.
\end{Verbatim}




\subsection{Rotation et miroir}
\label{rtd/instruction_openwide:rotation-et-miroir}\begin{tcolorbox}
\begin{minipage}{0.95\linewidth}
\textbf{Rendu}

\medskip













\end{minipage}
\end{tcolorbox}

\begin{Verbatim}[commandchars=\\\{\}]
\PYG{n+na}{:awesome:}\PYG{n+nv}{{}`shield{}`}

\PYG{n+na}{:awesome:}\PYG{n+nv}{{}`shield fa\PYGZhy{}rotate\PYGZhy{}90{}`}

\PYG{n+na}{:awesome:}\PYG{n+nv}{{}`shield fa\PYGZhy{}rotate\PYGZhy{}180{}`}

\PYG{n+na}{:awesome:}\PYG{n+nv}{{}`shield fa\PYGZhy{}rotate\PYGZhy{}270{}`}

\PYG{n+na}{:awesome:}\PYG{n+nv}{{}`shield fa\PYGZhy{}flip\PYGZhy{}horizontal{}`}

\PYG{n+na}{:awesome:}\PYG{n+nv}{{}`shield fa\PYGZhy{}flip\PYGZhy{}vertical{}`}
\end{Verbatim}




\subsection{Empilement d'icônes}
\label{rtd/instruction_openwide:empilement-d-icones}\begin{tcolorbox}
\begin{minipage}{0.95\linewidth}
\textbf{Rendu}

\medskip


 

 

 

 
\end{minipage}
\end{tcolorbox}

\begin{Verbatim}[commandchars=\\\{\}]
\PYG{p}{..} \PYG{o+ow}{container}\PYG{p}{::} fa\PYGZhy{}stack fa\PYGZhy{}lg

    \PYG{n+na}{:awesome:}\PYG{n+nv}{{}`square\PYGZhy{}o fa\PYGZhy{}stack\PYGZhy{}2x{}`} \PYG{n+na}{:awesome:}\PYG{n+nv}{{}`twitter fa\PYGZhy{}stack\PYGZhy{}1x{}`}

\PYG{p}{..} \PYG{o+ow}{container}\PYG{p}{::} fa\PYGZhy{}stack fa\PYGZhy{}lg

    \PYG{n+na}{:awesome:}\PYG{n+nv}{{}`circle fa\PYGZhy{}stack\PYGZhy{}2x{}`} \PYG{n+na}{:awesome:}\PYG{n+nv}{{}`flag fa\PYGZhy{}stack\PYGZhy{}1x fa\PYGZhy{}inverse{}`}

\PYG{p}{..} \PYG{o+ow}{container}\PYG{p}{::} fa\PYGZhy{}stack fa\PYGZhy{}lg

    \PYG{n+na}{:awesome:}\PYG{n+nv}{{}`square fa\PYGZhy{}stack\PYGZhy{}2x{}`} \PYG{n+na}{:awesome:}\PYG{n+nv}{{}`terminal fa\PYGZhy{}stack\PYGZhy{}1x fa\PYGZhy{}inverse{}`}

\PYG{p}{..} \PYG{o+ow}{container}\PYG{p}{::} fa\PYGZhy{}stack fa\PYGZhy{}lg

    \PYG{n+na}{:awesome:}\PYG{n+nv}{{}`camera fa\PYGZhy{}stack\PYGZhy{}1x{}`} \PYG{n+na}{:awesome:}\PYG{n+nv}{{}`ban fa\PYGZhy{}stack\PYGZhy{}2x text\PYGZhy{}danger{}`}
\end{Verbatim}




\section{Tags}
\label{rtd/instruction_openwide:tags}\begin{tcolorbox}
\begin{minipage}{0.95\linewidth}
\textbf{Rendu}

\medskip









\end{minipage}
\end{tcolorbox}

\begin{Verbatim}[commandchars=\\\{\}]
\PYG{n+na}{:tag:}\PYG{n+nv}{{}`default{}`}
\PYG{n+na}{:tag:}\PYG{n+nv}{{}`default::text default{}`}
\PYG{n+na}{:tag:}\PYG{n+nv}{{}`primary::text primary{}`}
\PYG{n+na}{:tag:}\PYG{n+nv}{{}`success::text success{}`}
\PYG{n+na}{:tag:}\PYG{n+nv}{{}`info::text info{}`}
\PYG{n+na}{:tag:}\PYG{n+nv}{{}`warning::text warning{}`}
\PYG{n+na}{:tag:}\PYG{n+nv}{{}`danger::text danger{}`}
\end{Verbatim}




\section{Typographie}
\label{rtd/instruction_openwide:typographie}\begin{tcolorbox}
\begin{minipage}{0.95\linewidth}
\textbf{Rendu}

\medskip


Fusce dapibus, tellus ac cursus commodo, tortor mauris nibh.

Nullam id dolor id nibh ultricies vehicula ut id elit.

Duis mollis, est non commodo luctus, nisi erat porttitor ligula.

Maecenas sed diam eget risus varius blandit sit amet non magna.

Etiam porta sem malesuada magna mollis euismod.

Donec ullamcorper nulla non metus auctor fringilla.
\end{minipage}
\end{tcolorbox}

\begin{Verbatim}[commandchars=\\\{\}]
\PYG{p}{..} \PYG{o+ow}{rst\PYGZhy{}class}\PYG{p}{::} text\PYGZhy{}muted

    Fusce dapibus, tellus ac cursus commodo, tortor mauris nibh.

\PYG{p}{..} \PYG{o+ow}{rst\PYGZhy{}class}\PYG{p}{::} text\PYGZhy{}primary

    Nullam id dolor id nibh ultricies vehicula ut id elit.

\PYG{p}{..} \PYG{o+ow}{rst\PYGZhy{}class}\PYG{p}{::} text\PYGZhy{}success

    Duis mollis, est non commodo luctus, nisi erat porttitor ligula.

\PYG{p}{..} \PYG{o+ow}{rst\PYGZhy{}class}\PYG{p}{::} text\PYGZhy{}info

    Maecenas sed diam eget risus varius blandit sit amet non magna.

\PYG{p}{..} \PYG{o+ow}{rst\PYGZhy{}class}\PYG{p}{::} text\PYGZhy{}warning

    Etiam porta sem malesuada magna mollis euismod.

\PYG{p}{..} \PYG{o+ow}{rst\PYGZhy{}class}\PYG{p}{::} text\PYGZhy{}danger

    Donec ullamcorper nulla non metus auctor fringilla.
\end{Verbatim}


\begin{tcolorbox}
\begin{minipage}{0.95\linewidth}
\textbf{Rendu}

\medskip


Nullam id dolor id nibh ultricies vehicula ut id elit.

Duis mollis, est non commodo luctus, nisi erat porttitor ligula.

Maecenas sed diam eget risus varius blandit sit amet non magna.

Etiam porta sem malesuada magna mollis euismod.

Donec ullamcorper nulla non metus auctor fringilla.
\end{minipage}
\end{tcolorbox}

\begin{Verbatim}[commandchars=\\\{\}]
\PYG{p}{..} \PYG{o+ow}{rst\PYGZhy{}class}\PYG{p}{::} bg\PYGZhy{}muted

    Fusce dapibus, tellus ac cursus commodo, tortor mauris nibh.

\PYG{p}{..} \PYG{o+ow}{rst\PYGZhy{}class}\PYG{p}{::} bg\PYGZhy{}primary

    Nullam id dolor id nibh ultricies vehicula ut id elit.

\PYG{p}{..} \PYG{o+ow}{rst\PYGZhy{}class}\PYG{p}{::} text\PYGZhy{}success

    Duis mollis, est non commodo luctus, nisi erat porttitor ligula.

\PYG{p}{..} \PYG{o+ow}{rst\PYGZhy{}class}\PYG{p}{::} bg\PYGZhy{}info

    Maecenas sed diam eget risus varius blandit sit amet non magna.

\PYG{p}{..} \PYG{o+ow}{rst\PYGZhy{}class}\PYG{p}{::} bg\PYGZhy{}warning

    Etiam porta sem malesuada magna mollis euismod.

\PYG{p}{..} \PYG{o+ow}{rst\PYGZhy{}class}\PYG{p}{::} bg\PYGZhy{}danger

    Donec ullamcorper nulla non metus auctor fringilla.
\end{Verbatim}


\begin{tcolorbox}
\begin{minipage}{0.95\linewidth}
\textbf{Rendu}

\medskip


Nullam id dolor id nibh ultricies vehicula ut id elit.

Duis mollis, est non commodo luctus, nisi erat porttitor ligula.

Maecenas sed diam eget risus varius blandit sit amet non magna.

Etiam porta sem malesuada magna mollis euismod.

Donec ullamcorper nulla non metus auctor fringilla.

Nullam id dolor id nibh ultricies vehicula ut id elit.

Duis mollis, est non commodo luctus, nisi erat porttitor ligula.

Maecenas sed diam eget risus varius blandit sit amet non magna.
\end{minipage}
\end{tcolorbox}

\begin{Verbatim}[commandchars=\\\{\}]
\PYG{p}{..} \PYG{o+ow}{rst\PYGZhy{}class}\PYG{p}{::} text\PYGZhy{}left

    Nullam id dolor id nibh ultricies vehicula ut id elit.

\PYG{p}{..} \PYG{o+ow}{rst\PYGZhy{}class}\PYG{p}{::} text\PYGZhy{}right

    Duis mollis, est non commodo luctus, nisi erat porttitor ligula.

\PYG{p}{..} \PYG{o+ow}{rst\PYGZhy{}class}\PYG{p}{::} text\PYGZhy{}center

    Maecenas sed diam eget risus varius blandit sit amet non magna.

\PYG{p}{..} \PYG{o+ow}{rst\PYGZhy{}class}\PYG{p}{::} text\PYGZhy{}justify

    Etiam porta sem malesuada magna mollis euismod.

\PYG{p}{..} \PYG{o+ow}{rst\PYGZhy{}class}\PYG{p}{::} text\PYGZhy{}nowrap

    Donec ullamcorper nulla non metus auctor fringilla.

\PYG{p}{..} \PYG{o+ow}{rst\PYGZhy{}class}\PYG{p}{::} text\PYGZhy{}lowercase

    Nullam id dolor id nibh ultricies vehicula ut id elit.

\PYG{p}{..} \PYG{o+ow}{rst\PYGZhy{}class}\PYG{p}{::} text\PYGZhy{}uppercase

    Duis mollis, est non commodo luctus, nisi erat porttitor ligula.

\PYG{p}{..} \PYG{o+ow}{rst\PYGZhy{}class}\PYG{p}{::} text\PYGZhy{}capitalize

    Maecenas sed diam eget risus varius blandit sit amet non magna.
\end{Verbatim}




\chapter{Les extensions}
\label{extensions::doc}\label{extensions:les-extensions}

\section{sphinx-scruffy}
\label{extensions/sphinx-scruffy::doc}\label{extensions/sphinx-scruffy:sphinx-scruffy}\begin{itemize}
\item {} 
\href{https://pypi.python.org/pypi/sphinx-scruffy}{https://pypi.python.org/pypi/sphinx-scruffy}

\item {} 
\href{https://github.com/paylogic/sphinx-scruffy}{https://github.com/paylogic/sphinx-scruffy}

\item {} 
\href{https://github.com/aivarsk/scruffy}{https://github.com/aivarsk/scruffy}

\end{itemize}


\subsection{Dépendances}
\label{extensions/sphinx-scruffy:dependances}
\begin{Verbatim}[commandchars=\\\{\}]
\PYG{g+go}{sudo apt\PYGZhy{}get install graphviz librsvg2\PYGZhy{}bin plotutils}
\PYG{g+go}{sudo apt\PYGZhy{}get install ttf\PYGZhy{}thai\PYGZhy{}tlwg}
\end{Verbatim}


\subsection{Exemple simple}
\label{extensions/sphinx-scruffy:exemple-simple}
\begin{Verbatim}[commandchars=\\\{\}]
\PYG{p}{..} \PYG{o+ow}{scruffy}\PYG{p}{::}

    [Node A]\PYGZhy{}\PYGZgt{}[Node B]
    [Node B]\PYGZhy{}\PYGZgt{}[Node C]
    [Group [Node A][Node B]]
\end{Verbatim}


\subsection{Diagrammes de classe}
\label{extensions/sphinx-scruffy:diagrammes-de-classe}
\begin{Verbatim}[commandchars=\\\{\}]
\PYG{p}{..} \PYG{o+ow}{scruffy}\PYG{p}{::}

    [User\textbar{}+Forename;+Surname;+HashedPassword;\PYGZhy{}Salt\textbar{}+Login();+Logout()]
\end{Verbatim}

\begin{Verbatim}[commandchars=\\\{\}]
\PYG{p}{..} \PYG{o+ow}{scruffy}\PYG{p}{::}

    [note: You can stick notes on diagrams too!\PYGZob{}bg:cornsilk\PYGZcb{}]
    [Customer]\PYGZlt{}\PYGZgt{}1\PYGZhy{}orders 0..*\PYGZgt{}[Order]
    [Order]++*\PYGZhy{}\PYGZgt{}[LineItem]
    [Order]\PYGZhy{}1\PYGZgt{}[DeliveryMethod]
    [Order]\PYGZhy{}*\PYGZgt{}[Product]
    [Category]\PYGZlt{}\PYGZhy{}\PYGZgt{}[Product]
    [DeliveryMethod]\PYGZca{}[National]
    [DeliveryMethod]\PYGZca{}[International]
\end{Verbatim}


\subsection{Diagrammes de séquence}
\label{extensions/sphinx-scruffy:diagrammes-de-sequence}
\begin{Verbatim}[commandchars=\\\{\}]
\PYG{p}{..} \PYG{o+ow}{scruffy}\PYG{p}{::}
    \PYG{n+nc}{:sequence:}

    [Patron]order food\PYGZgt{}[Waiter]
    [Waiter]order food\PYGZgt{}[Cook]
    [Waiter]serve wine\PYGZgt{}[Patron]
    [Cook]pickup\PYGZgt{}[Waiter]
    [Waiter]serve food\PYGZgt{}[Patron]
    [Patron]pay\PYGZgt{}[Cashier]
\end{Verbatim}


\section{sphinx.ext.extlinks}
\label{extensions/sphinx.ext.extlinks:sphinx-ext-extlinks}\label{extensions/sphinx.ext.extlinks::doc}
L'extension \code{sphinx.ext.extlinks} est une aide quand on a plusieurs liens qui pointent vers le même site ou quand on a des liens avec les paramètres.

Dans le fichier \code{config.py}, on définit les liens :

\begin{Verbatim}[commandchars=\\\{\}]
\PYG{n}{extlinks} \PYG{o}{=} \PYG{p}{\PYGZob{}}\PYG{l+s}{\PYGZsq{}}\PYG{l+s}{alias}\PYG{l+s}{\PYGZsq{}}\PYG{p}{:} \PYG{p}{(}\PYG{l+s}{\PYGZsq{}}\PYG{l+s}{http://url.de.la.page/}\PYG{l+s+si}{\PYGZpc{}s}\PYG{l+s}{\PYGZsq{}}\PYG{p}{,} \PYG{l+s}{\PYGZsq{}}\PYG{l+s}{préfixe du lien }\PYG{l+s}{\PYGZsq{}}\PYG{p}{)}\PYG{p}{\PYGZcb{}}

\PYG{n}{extlinks} \PYG{o}{=} \PYG{p}{\PYGZob{}}\PYG{l+s}{\PYGZsq{}}\PYG{l+s}{helios}\PYG{l+s}{\PYGZsq{}}\PYG{p}{:} \PYG{p}{(}\PYG{l+s}{\PYGZsq{}}\PYG{l+s}{https://helios.openwide.fr/ticket/}\PYG{l+s+si}{\PYGZpc{}s}\PYG{l+s}{\PYGZsq{}}\PYG{p}{,} \PYG{l+s}{\PYGZsq{}}\PYG{l+s}{ticket }\PYG{l+s}{\PYGZsq{}}\PYG{p}{)}\PYG{p}{\PYGZcb{}}
\end{Verbatim}

Ensuite, on peut utiliser l'alias comme un rôle :
\begin{tcolorbox}
\begin{minipage}{0.95\linewidth}
\textbf{Rendu}

\medskip


\href{https://helios.openwide.fr/ticket/29400}{ticket 29400}
\end{minipage}
\end{tcolorbox}

\begin{Verbatim}[commandchars=\\\{\}]
\PYG{n+na}{:helios:}\PYG{n+nv}{{}`29400{}`}
\end{Verbatim}



On peut également spécifier un nom explicite pour le lien :
\begin{tcolorbox}
\begin{minipage}{0.95\linewidth}
\textbf{Rendu}

\medskip


\href{https://helios.openwide.fr/ticket/29400}{ce ticket Helios}
\end{minipage}
\end{tcolorbox}

\begin{Verbatim}[commandchars=\\\{\}]
\PYG{n+na}{:helios:}\PYG{n+nv}{{}`ce ticket Helios \PYGZlt{}29400\PYGZgt{}{}`}
\end{Verbatim}




\section{sphinx.ext.ifconfig}
\label{extensions/sphinx.ext.ifconfig::doc}\label{extensions/sphinx.ext.ifconfig:sphinx-ext-ifconfig}
L'extension \code{sphinx.ext.ifconfig} permet conditionner la construction de certaines parties du document en fonction de paramètre de configuration.

Dans un premier temps, il faut définir le paramètre de configuration dans \code{config.py} :

\begin{Verbatim}[commandchars=\\\{\}]
\PYG{k}{def} \PYG{n+nf}{setup}\PYG{p}{(}\PYG{n}{app}\PYG{p}{)}\PYG{p}{:}
    \PYG{n}{app}\PYG{o}{.}\PYG{n}{add\PYGZus{}config\PYGZus{}value}\PYG{p}{(}\PYG{l+s}{\PYGZsq{}}\PYG{l+s}{newconf}\PYG{l+s}{\PYGZsq{}}\PYG{p}{,} \PYG{l+s}{\PYGZsq{}}\PYG{l+s}{default}\PYG{l+s}{\PYGZsq{}}\PYG{p}{,} \PYG{n+nb+bp}{True}\PYG{p}{)}
\end{Verbatim}

Ensuite, dans les fichiers \code{rst}, vous pouvez utiliser ce paramètre pour conditionner l'affiche de certaines valeurs :

\begin{Verbatim}[commandchars=\\\{\}]
\PYG{p}{..} \PYG{o+ow}{ifconfig}\PYG{p}{::} newconf == \PYGZsq{}default\PYGZsq{}

    Texte affiché que quand la config \PYG{l+s}{{}`{}`}\PYG{l+s}{newconf}\PYG{l+s}{{}`{}`} est égale à \PYG{l+s}{{}`{}`}\PYG{l+s}{default}\PYG{l+s}{{}`{}`}
\end{Verbatim}

Dans votre \code{Makefile}, vous pouvez changer la valeur du paramètre avec l'option \code{-D} :

\begin{Verbatim}[commandchars=\\\{\}]
\PYG{g+gp}{\PYGZdl{}}\PYG{o}{(}SPHINXBUILD\PYG{o}{)} \PYGZhy{}b html \PYGZhy{}D \PYG{n+nv}{newconf}\PYG{o}{=}value \PYG{k}{\PYGZdl{}(}ALLSPHINXOPTS\PYG{k}{)} \PYG{k}{\PYGZdl{}(}BUILDDIR\PYG{k}{)}/html
\end{Verbatim}


\section{sphinx.ext.intersphinx}
\label{extensions/sphinx.ext.intersphinx::doc}\label{extensions/sphinx.ext.intersphinx:sphinx-ext-intersphinx}
\href{http://sphinx-doc.org/latest/ext/intersphinx.html}{http://sphinx-doc.org/latest/ext/intersphinx.html}


\section{sphinx.ext.todo}
\label{extensions/sphinx.ext.todo:sphinx-ext-todo}\label{extensions/sphinx.ext.todo::doc}
L'extension \code{sphinx.ext.todo} permet d'insérer des TODO :

\begin{notice}{note}{À faire}
\begin{itemize}
\item {} 
un truc à faire

\item {} 
un autre

\item {} 
encore un autre

\end{itemize}
\end{notice}

\begin{Verbatim}[commandchars=\\\{\}]
\PYG{p}{..} \PYG{o+ow}{todo}\PYG{p}{::}

    \PYG{l+m}{*} un truc à faire
    \PYG{l+m}{*} un autre
    \PYG{l+m}{*} encore un autre
\end{Verbatim}

La directives \code{todolist} permet d'afficher toutes les TODO présentent dans le document si le paramètre de configuration \code{todo\_include\_todos} est à \code{True}.

\begin{Verbatim}[commandchars=\\\{\}]
\PYG{p}{..} \PYG{o+ow}{todolist}\PYG{p}{::}
\end{Verbatim}

\begin{notice}{note}{À faire}
\begin{itemize}
\item {} 
un truc à faire

\item {} 
un autre

\item {} 
encore un autre

\end{itemize}
\end{notice}

(L'{\hyperref[extensions/sphinx.ext.todo:index-0]{\emph{\emph{entrée originale}}}} (\autopageref*{extensions/sphinx.ext.todo:index-0}) se trouve dans /data/sources/docs-polephp/docs/sphinx/extensions/sphinx.ext.todo.rst, à la ligne 7.)


\section{sphinxcontrib-phpdomain}
\label{extensions/sphinxcontrib-phpdomain:sphinxcontrib-phpdomain}\label{extensions/sphinxcontrib-phpdomain::doc}\begin{quote}

\begin{Verbatim}[commandchars=\\\{\}]
\PYG{p}{..} \PYG{o+ow}{php:class}\PYG{p}{::} DateTime

    Datetime class

\PYG{p}{    ..} \PYG{o+ow}{php:method}\PYG{p}{::} setDate(\PYGZdl{}year, \PYGZdl{}month, \PYGZdl{}day)

        Set the date.

        \PYG{n+nc}{:param int \PYGZdl{}year:} \PYG{n+nf}{The year.}
        \PYG{n+nc}{:param int \PYGZdl{}month:} \PYG{n+nf}{The month.}
        \PYG{n+nc}{:param int \PYGZdl{}day:} \PYG{n+nf}{The day.}
        \PYG{n+nc}{:returns:} \PYG{n+nf}{Either false on failure, or the datetime object for method chaining.}


\PYG{p}{    ..} \PYG{o+ow}{php:method}\PYG{p}{::} setTime(\PYGZdl{}hour, \PYGZdl{}minute[, \PYGZdl{}second])

        Set the time.

        \PYG{n+nc}{:param int \PYGZdl{}hour:} \PYG{n+nf}{The hour}
        \PYG{n+nc}{:param int \PYGZdl{}minute:} \PYG{n+nf}{The minute}
        \PYG{n+nc}{:param int \PYGZdl{}second:} \PYG{n+nf}{The second}
        \PYG{n+nc}{:returns:} \PYG{n+nf}{Either false on failure, or the datetime object for method chaining.}

\PYG{p}{    ..} \PYG{o+ow}{php:const}\PYG{p}{::} ATOM

        Y\PYGZhy{}m\PYGZhy{}d\PYGZbs{}TH:i:sP
\end{Verbatim}
\end{quote}
\index{DateTime (class)|textbf}

\begin{fulllineitems}
\phantomsection\label{extensions/sphinxcontrib-phpdomain:DateTime}\pysigline{\strong{class }\bfcode{DateTime}}
Datetime class
\index{setDate() (méthode DateTime)|textbf}

\begin{fulllineitems}
\phantomsection\label{extensions/sphinxcontrib-phpdomain:DateTime::setDate}\pysiglinewithargsret{\bfcode{setDate}}{\emph{\$year}, \emph{\$month}, \emph{\$day}}{}
Set the date.
\begin{quote}\begin{description}
\item[{Paramètres}] \leavevmode\begin{itemize}
\item {} 
\textbf{\texttt{\$year}} (\emph{int}) -- The year.

\item {} 
\textbf{\texttt{\$month}} (\emph{int}) -- The month.

\item {} 
\textbf{\texttt{\$day}} (\emph{int}) -- The day.

\end{itemize}

\item[{Retourne}] \leavevmode
Either false on failure, or the datetime object for method chaining.

\end{description}\end{quote}

\end{fulllineitems}

\index{setTime() (méthode DateTime)|textbf}

\begin{fulllineitems}
\phantomsection\label{extensions/sphinxcontrib-phpdomain:DateTime::setTime}\pysiglinewithargsret{\bfcode{setTime}}{\emph{\$hour}, \emph{\$minute}\optional{, \emph{\$second}}}{}
Set the time.
\begin{quote}\begin{description}
\item[{Paramètres}] \leavevmode\begin{itemize}
\item {} 
\textbf{\texttt{\$hour}} (\emph{int}) -- The hour

\item {} 
\textbf{\texttt{\$minute}} (\emph{int}) -- The minute

\item {} 
\textbf{\texttt{\$second}} (\emph{int}) -- The second

\end{itemize}

\item[{Retourne}] \leavevmode
Either false on failure, or the datetime object for method chaining.

\end{description}\end{quote}

\end{fulllineitems}

\index{DateTime::ATOM (class constant)|textbf}

\begin{fulllineitems}
\phantomsection\label{extensions/sphinxcontrib-phpdomain:DateTime::ATOM}\pysigline{\strong{constant }\bfcode{ATOM}}
Y-m-dTH:i:sP

\end{fulllineitems}


\end{fulllineitems}



\section{sphinxcontrib.googlechart}
\label{extensions/sphinxcontrib.googlechart:sphinxcontrib-googlechart}\label{extensions/sphinxcontrib.googlechart::doc}\begin{itemize}
\item {} 
\href{https://pypi.python.org/pypi/sphinxcontrib-googlechart}{https://pypi.python.org/pypi/sphinxcontrib-googlechart}

\item {} 
\href{https://pythonhosted.org/sphinxcontrib-googlechart/}{https://pythonhosted.org/sphinxcontrib-googlechart/}

\end{itemize}


\subsection{piechart}
\label{extensions/sphinxcontrib.googlechart:piechart}
\begin{Verbatim}[commandchars=\\\{\}]
\PYG{p}{..} \PYG{o+ow}{piechart}\PYG{p}{::}

    dog: 100
    cat: 80
    rabbit: 40
\end{Verbatim}




\subsection{piechart3d}
\label{extensions/sphinxcontrib.googlechart:piechart3d}
\begin{Verbatim}[commandchars=\\\{\}]
\PYG{p}{..} \PYG{o+ow}{piechart3d}\PYG{p}{::}
    \PYG{n+nc}{:size:} \PYG{n+nf}{480x240}

    dog: 100
    cat: 80
    rabbit: 40
\end{Verbatim}




\subsection{linechart}
\label{extensions/sphinxcontrib.googlechart:linechart}
\begin{Verbatim}[commandchars=\\\{\}]
\PYG{p}{..} \PYG{o+ow}{linechart}\PYG{p}{::}

    bicycle: 15, 35, 20, 40
    bicycle.color: ff0000
    car: 60, 75, 60, 30
    car.color: 0000ff
\end{Verbatim}




\subsection{linechartxy}
\label{extensions/sphinxcontrib.googlechart:linechartxy}
\begin{Verbatim}[commandchars=\\\{\}]
\PYG{p}{..} \PYG{o+ow}{linechartxy}\PYG{p}{::}

    bicycle: (0, 15), (30, 35), (60, 20), (90, 40)
    car: (0, 60), (20, 75), (40, 60), (90, 30)
\end{Verbatim}




\subsection{holizontal\_barchart}
\label{extensions/sphinxcontrib.googlechart:holizontal-barchart}
\begin{Verbatim}[commandchars=\\\{\}]
\PYG{p}{..} \PYG{o+ow}{holizontal\PYGZus{}barchart}\PYG{p}{::}

    bicycle: 15, 25, 20, 30
    bicycle.color: ff0000
    bicycle.axis: x
    bicycle.axis\PYGZus{}label: slow, fast
    car: 40, 50, 60, 45
    car.color: 0000ff
\end{Verbatim}




\subsection{vertical\_barchart}
\label{extensions/sphinxcontrib.googlechart:vertical-barchart}
\begin{Verbatim}[commandchars=\\\{\}]
\PYG{p}{..} \PYG{o+ow}{vertical\PYGZus{}barchart}\PYG{p}{::}

    bicycle: 15, 25, 20, 30
    bicycle.color: ff0000
    bicycle.axis: y
    bicycle.axis\PYGZus{}label: slow, fast
    car: 40, 50, 60, 45
    car.color: 0000ff
\end{Verbatim}




\subsection{holizontal\_bargraph}
\label{extensions/sphinxcontrib.googlechart:holizontal-bargraph}
\begin{Verbatim}[commandchars=\\\{\}]
\PYG{p}{..} \PYG{o+ow}{holizontal\PYGZus{}bargraph}\PYG{p}{::}

    bicycle: 15, 25, 20, 30
    bicycle.color: ff0000
    bicycle.axis: x
    bicycle.axis\PYGZus{}label: slow, fast
    car: 40, 50, 60, 45
    car.color: 0000ff
\end{Verbatim}




\subsection{vertical\_bargraph}
\label{extensions/sphinxcontrib.googlechart:vertical-bargraph}
\begin{Verbatim}[commandchars=\\\{\}]
\PYG{p}{..} \PYG{o+ow}{vertical\PYGZus{}bargraph}\PYG{p}{::}

    bicycle: 15, 25, 20, 30
    bicycle.color: ff0000
    bicycle.axis: x
    bicycle.axis\PYGZus{}label: slow, fast
    car: 40, 50, 60, 45
    car.color: 0000ff
\end{Verbatim}




\subsection{venndiagram}
\label{extensions/sphinxcontrib.googlechart:venndiagram}
\begin{Verbatim}[commandchars=\\\{\}]
\PYG{p}{..} \PYG{o+ow}{venndiagram}\PYG{p}{::}

    data: 100, 80, 40, 20, 20, 20, 10
\end{Verbatim}




\subsection{plotchart}
\label{extensions/sphinxcontrib.googlechart:plotchart}
\begin{Verbatim}[commandchars=\\\{\}]
\PYG{p}{..} \PYG{o+ow}{plotchart}\PYG{p}{::}

    data: (50, 60), (75, 20), (20, 30), (10, 70), (45, 10)
\end{Verbatim}




\subsection{mapchart}
\label{extensions/sphinxcontrib.googlechart:mapchart}
\begin{Verbatim}[commandchars=\\\{\}]
\PYG{p}{..} \PYG{o+ow}{mapchart}\PYG{p}{::}

    data: CN, JP, KR
    color: ff0000, 00ff00, 0000ff
\end{Verbatim}



\begin{Verbatim}[commandchars=\\\{\}]
\PYG{p}{..} \PYG{o+ow}{mapchart}\PYG{p}{::}

    CN: \PYGZdq{}People\PYGZsq{}s Republic of China\PYGZdq{}
    CN.color: ff0000
    JP: Japan
    JP.color: 00ff00
    KR: \PYGZdq{}Republic of Korea\PYGZdq{}
    KR.color: 0000ff
\end{Verbatim}




\section{sphinxcontrib.googlechart.graphviz}
\label{extensions/sphinxcontrib.googlechart:sphinxcontrib-googlechart-graphviz}
Cette extension fait la même chose que \code{sphinx.ext.graphviz} mais ne nécessite pas les binaires \textbf{\texttt{graphviz}}.

\begin{Verbatim}[commandchars=\\\{\}]
\PYG{p}{..} \PYG{o+ow}{graphviz}\PYG{p}{::}

    digraph \PYGZob{}
        A \PYGZhy{}\PYGZgt{} B;
    \PYGZcb{}
\end{Verbatim}



\begin{Verbatim}[commandchars=\\\{\}]
\PYG{p}{..} \PYG{o+ow}{graph}\PYG{p}{::}

    A \PYGZhy{}\PYGZhy{} B;
\end{Verbatim}



\begin{Verbatim}[commandchars=\\\{\}]
\PYG{p}{..} \PYG{o+ow}{graph}\PYG{p}{::}

    C\PYGZus{}0\PYGZhy{}\PYGZhy{}H\PYGZus{}0[type=s];C\PYGZus{}0\PYGZhy{}\PYGZhy{}H\PYGZus{}1[type=s];C\PYGZus{}0\PYGZhy{}\PYGZhy{}H\PYGZus{}2[type=s];C\PYGZus{}0\PYGZhy{}\PYGZhy{}C\PYGZus{}1[type=s];C\PYGZus{}1\PYGZhy{}\PYGZhy{}H\PYGZus{}3[type=s];C\PYGZus{}1\PYGZhy{}\PYGZhy{}H\PYGZus{}4[type=s];C\PYGZus{}1\PYGZhy{}\PYGZhy{}H\PYGZus{}5[type=s];
\end{Verbatim}



\begin{Verbatim}[commandchars=\\\{\}]
\PYG{p}{..} \PYG{o+ow}{digraph}\PYG{p}{::}

    A \PYGZhy{}\PYGZhy{} B;
\end{Verbatim}


\section{sphinxcontrib.images}
\label{extensions/sphinxcontrib.images:sphinxcontrib-images}\label{extensions/sphinxcontrib.images::doc}\begin{itemize}
\item {} 
\href{https://pypi.python.org/pypi/sphinxcontrib-images}{https://pypi.python.org/pypi/sphinxcontrib-images}

\item {} 
\href{https://github.com/spinus/sphinxcontrib-images}{https://github.com/spinus/sphinxcontrib-images}

\item {} 
\href{https://pythonhosted.org/sphinxcontrib-images/}{https://pythonhosted.org/sphinxcontrib-images/}

\end{itemize}

En cliquant sur l'image, on peut l'afficher sans redimensionnement dans une popin.
\begin{tcolorbox}
\begin{minipage}{0.95\linewidth}
\textbf{Rendu}

\medskip


\includegraphics{mouse1.jpg}
\end{minipage}
\end{tcolorbox}

\begin{Verbatim}[commandchars=\\\{\}]
\PYG{p}{..} \PYG{o+ow}{thumbnail}\PYG{p}{::} /\PYGZus{}static/images/mouse.jpg
\end{Verbatim}




\section{sphinxcontrib.nvd3}
\label{extensions/sphinxcontrib.nvd3::doc}\label{extensions/sphinxcontrib.nvd3:sphinxcontrib-nvd3}

\subsection{Line Chart}
\label{extensions/sphinxcontrib.nvd3:line-chart}
\begin{Verbatim}[commandchars=\\\{\}]
\PYG{p}{..} \PYG{o+ow}{nvd3\PYGZhy{}linechart}\PYG{p}{::} /\PYGZus{}static/nvd3\PYGZhy{}chart/line\PYGZhy{}sample.csv
    :jquery\PYGZus{}on\PYGZus{}ready:
    :use\PYGZus{}interactive\PYGZus{}guideline:
    \PYG{n+nc}{:x\PYGZus{}axis\PYGZus{}format:} \PYG{n+nf}{AM\PYGZus{}PM}
    \PYG{n+nc}{:extras:} \PYG{n+nf}{\PYGZob{}\PYGZdq{}tooltip\PYGZdq{}: \PYGZob{}\PYGZdq{}y\PYGZus{}start\PYGZdq{}: \PYGZdq{}\PYGZdq{}, \PYGZdq{}y\PYGZus{}end\PYGZdq{}: \PYGZdq{}\PYGZdq{}\PYGZcb{}\PYGZcb{}}
\end{Verbatim}


\subsection{Line Plus Bar Chart}
\label{extensions/sphinxcontrib.nvd3:line-plus-bar-chart}
\begin{Verbatim}[commandchars=\\\{\}]
\PYG{p}{..} \PYG{o+ow}{nvd3\PYGZhy{}lineplusbarchart}\PYG{p}{::} /\PYGZus{}static/nvd3\PYGZhy{}chart/line\PYGZhy{}sample.csv
    \PYG{n+nc}{:chart\PYGZus{}kwargs:} \PYG{n+nf}{\PYGZob{}\PYGZdq{}ydata1\PYGZdq{}: \PYGZob{}\PYGZdq{}bar\PYGZdq{}: true\PYGZcb{}\PYGZcb{}}
\end{Verbatim}


\subsection{Line With Focus Chart}
\label{extensions/sphinxcontrib.nvd3:line-with-focus-chart}
\begin{Verbatim}[commandchars=\\\{\}]
\PYG{p}{..} \PYG{o+ow}{nvd3\PYGZhy{}linewithfocuschart}\PYG{p}{::} /\PYGZus{}static/nvd3\PYGZhy{}chart/line\PYGZhy{}sample.csv
\end{Verbatim}


\subsection{Cumulative Line Chart}
\label{extensions/sphinxcontrib.nvd3:cumulative-line-chart}

\subsection{Discrete Bar Chart}
\label{extensions/sphinxcontrib.nvd3:discrete-bar-chart}
\begin{Verbatim}[commandchars=\\\{\}]
\PYG{p}{..} \PYG{o+ow}{nvd3\PYGZhy{}discretebarchart}\PYG{p}{::}

    xdata,ydata
    A,3
    B,12
    C,\PYGZhy{}10
    D,5
    E,25
    F,\PYGZhy{}7
    G,2
\end{Verbatim}


\subsection{Multi Bar Chart}
\label{extensions/sphinxcontrib.nvd3:multi-bar-chart}
\begin{Verbatim}[commandchars=\\\{\}]
\PYG{p}{..} \PYG{o+ow}{nvd3\PYGZhy{}multibarchart}\PYG{p}{::}

    xdata,ydata1,ydata2
    0,8,16
    1,1,2
    2,7,14
    3,9,18
    4,8,16
    5,7,14
    6,4,8
    7,6,12
    8,9,18
    9,4,8
\end{Verbatim}


\subsection{Multi Bar Horizontal Chart}
\label{extensions/sphinxcontrib.nvd3:multi-bar-horizontal-chart}
\begin{Verbatim}[commandchars=\\\{\}]
\PYG{p}{..} \PYG{o+ow}{nvd3\PYGZhy{}multibarhorizontalchart}\PYG{p}{::}

    xdata,ydata1,ydata2
    0,8,16
    1,1,2
    2,7,14
    3,9,18
    4,8,16
    5,7,14
    6,4,8
    7,6,12
    8,9,18
    9,4,8
\end{Verbatim}


\subsection{Pie Chart}
\label{extensions/sphinxcontrib.nvd3:pie-chart}
\begin{Verbatim}[commandchars=\\\{\}]
\PYG{p}{..} \PYG{o+ow}{nvd3\PYGZhy{}piechart}\PYG{p}{::}

    xdata,ydata
    Orange,3
    Banana,4
    pear,0
    Kiwi,1
    Apple,5
    Strawberry,7
    Pineapple,3
\end{Verbatim}


\subsection{Scatter Chart}
\label{extensions/sphinxcontrib.nvd3:scatter-chart}
\begin{Verbatim}[commandchars=\\\{\}]
\PYG{p}{..} \PYG{o+ow}{nvd3\PYGZhy{}scatterchart}\PYG{p}{::} /\PYGZus{}static/nvd3\PYGZhy{}chart/scatter\PYGZhy{}sample.csv
    \PYG{n+nc}{:height:} \PYG{n+nf}{450}
    \PYG{n+nc}{:extras:} \PYG{n+nf}{\PYGZob{}\PYGZdq{}tooltip\PYGZdq{}: \PYGZob{}\PYGZdq{}y\PYGZus{}start\PYGZdq{}: \PYGZdq{}\PYGZdq{}, \PYGZdq{}y\PYGZus{}end\PYGZdq{}: \PYGZdq{} calls\PYGZdq{}\PYGZcb{}\PYGZcb{}}
\end{Verbatim}


\subsection{Stacked Area Chart}
\label{extensions/sphinxcontrib.nvd3:stacked-area-chart}
\begin{Verbatim}[commandchars=\\\{\}]
\PYG{p}{..} \PYG{o+ow}{nvd3\PYGZhy{}stackedareachart}\PYG{p}{::}

    xdata,ydata1,ydata2
    100,6,8
    101,11,20
    102,12,16
    103,7,12
    104,11,20
    105,10,28
    106,11,28
\end{Verbatim}


\section{sphinxcontrib.programoutput}
\label{extensions/sphinxcontrib.programoutput:sphinxcontrib-programoutput}\label{extensions/sphinxcontrib.programoutput::doc}\begin{itemize}
\item {} 
\href{https://pypi.python.org/pypi/sphinxcontrib-programoutput}{https://pypi.python.org/pypi/sphinxcontrib-programoutput}

\item {} 
\href{https://pythonhosted.org/sphinxcontrib-programoutput/}{https://pythonhosted.org/sphinxcontrib-programoutput/}

\end{itemize}

\begin{Verbatim}[commandchars=\\\{\}]
\PYGZdl{} php \PYGZhy{}v
PHP 5.5.9\PYGZhy{}1ubuntu4.11 (cli) (built: Jul  2 2015 15:23:08) 
Copyright (c) 1997\PYGZhy{}2014 The PHP Group
Zend Engine v2.5.0, Copyright (c) 1998\PYGZhy{}2014 Zend Technologies
    with Zend OPcache v7.0.3, Copyright (c) 1999\PYGZhy{}2014, by Zend Technologies
    with Xdebug v2.2.3, Copyright (c) 2002\PYGZhy{}2013, by Derick Rethans
\end{Verbatim}

\begin{Verbatim}[commandchars=\\\{\}]
\PYG{p}{..} \PYG{o+ow}{command\PYGZhy{}output}\PYG{p}{::} php \PYGZhy{}v
\end{Verbatim}


\section{sphinxcontrib.taglist}
\label{extensions/sphinxcontrib.taglist::doc}\label{extensions/sphinxcontrib.taglist:sphinxcontrib-taglist}\begin{itemize}
\item {} 
\href{https://github.com/spinus/sphinxcontrib-taglist}{https://github.com/spinus/sphinxcontrib-taglist}

\end{itemize}
\phantomsection\label{extensions/sphinxcontrib.taglist:taglist-0}
\begin{notice}{note}
   Some tagged information
\end{notice}

\begin{Verbatim}[commandchars=\\\{\}]
\PYG{p}{..} \PYG{o+ow}{tag}\PYG{p}{::} Some tagged information
    \PYG{n+nc}{:tag:} \PYG{n+nf}{success::tag1 tag2 tag3}
\end{Verbatim}
\phantomsection\label{extensions/sphinxcontrib.taglist:taglist-1}
\begin{notice}{note}
   Some tagged information
\end{notice}

\begin{Verbatim}[commandchars=\\\{\}]
\PYG{p}{..} \PYG{o+ow}{tag}\PYG{p}{::} [tag1 tag2 tag3] Some tagged information
\end{Verbatim}

Dans le \code{config.py}, on peut customiser l'apparence du tag :

\begin{Verbatim}[commandchars=\\\{\}]
\PYG{n}{taglist\PYGZus{}tags} \PYG{o}{=} \PYG{p}{\PYGZob{}}
    \PYG{l+s}{\PYGZsq{}}\PYG{l+s}{tag1}\PYG{l+s}{\PYGZsq{}}\PYG{p}{:} \PYG{p}{\PYGZob{}}\PYG{l+s}{\PYGZsq{}}\PYG{l+s}{background\PYGZhy{}color}\PYG{l+s}{\PYGZsq{}}\PYG{p}{:} \PYG{l+s}{\PYGZsq{}}\PYG{l+s}{green}\PYG{l+s}{\PYGZsq{}}\PYG{p}{\PYGZcb{}}
\PYG{p}{\PYGZcb{}}
\end{Verbatim}

\begin{notice}{attention}{Attention:}
Le rôle \code{tag} définit dans cette extension est en conflit avec celui définit dans le thème \code{sphinx\_rtd\_theme}.
\end{notice}


\section{sphinxcontrib.webmocks}
\label{extensions/sphinxcontrib.webmocks::doc}\label{extensions/sphinxcontrib.webmocks:sphinxcontrib-webmocks}\begin{itemize}
\item {} 
\href{https://pythonhosted.org/sphinxcontrib-webmocks/}{https://pythonhosted.org/sphinxcontrib-webmocks/}

\end{itemize}


\subsection{Exemple}
\label{extensions/sphinxcontrib.webmocks:exemple}\begin{quote}\begin{description}
\item[{Name}] \leavevmode


\item[{Address}] \leavevmode


\end{description}\end{quote}

 

\begin{Verbatim}[commandchars=\\\{\}]
\PYG{n+nc}{:Name:} \PYG{n+nf}{:text:{}`Input your name{}`}
\PYG{n+nc}{:Address:} \PYG{n+nf}{:text:{}`Input your address{}`}

\PYG{n+na}{:button:}\PYG{n+nv}{{}`OK{}`} \PYG{n+na}{:button:}\PYG{n+nv}{{}`Cancel{}`}
\end{Verbatim}


\subsection{text}
\label{extensions/sphinxcontrib.webmocks:text}\begin{tcolorbox}
\begin{minipage}{0.95\linewidth}
\textbf{Rendu}

\medskip


Text form is here: 

Empty text form is here: 
\end{minipage}
\end{tcolorbox}

\begin{Verbatim}[commandchars=\\\{\}]
Text form is here: \PYG{n+na}{:text:}\PYG{n+nv}{{}`default value{}`}

Empty text form is here: \PYG{n+na}{:text:}\PYG{n+nv}{{}`\PYGZus{}{}`}
\end{Verbatim}




\subsection{textarea}
\label{extensions/sphinxcontrib.webmocks:textarea}\begin{tcolorbox}
\begin{minipage}{0.95\linewidth}
\textbf{Rendu}

\medskip


Textarea form is here: 

Empty textarea form is here: 
\end{minipage}
\end{tcolorbox}

\begin{Verbatim}[commandchars=\\\{\}]
Textarea form is here: \PYG{n+na}{:textarea:}\PYG{n+nv}{{}`default value{}`}

Empty textarea form is here: \PYG{n+na}{:textarea:}\PYG{n+nv}{{}`\PYGZus{}{}`}
\end{Verbatim}




\subsection{select}
\label{extensions/sphinxcontrib.webmocks:select}\begin{tcolorbox}
\begin{minipage}{0.95\linewidth}
\textbf{Rendu}

\medskip


Select form is here: 
\end{minipage}
\end{tcolorbox}

\begin{Verbatim}[commandchars=\\\{\}]
Select form is here: \PYG{n+na}{:select:}\PYG{n+nv}{{}`Item 1,Item 2,Item 3,Item 4{}`}
\end{Verbatim}




\subsection{radio}
\label{extensions/sphinxcontrib.webmocks:radio}\begin{tcolorbox}
\begin{minipage}{0.95\linewidth}
\textbf{Rendu}

\medskip


Radio buttons are here: 
\end{minipage}
\end{tcolorbox}

\begin{Verbatim}[commandchars=\\\{\}]
Radio buttons are here: \PYG{n+na}{:radio:}\PYG{n+nv}{{}`Item 1,Item 2,Item 3,Item 4{}`}
\end{Verbatim}




\subsection{checkbox}
\label{extensions/sphinxcontrib.webmocks:checkbox}\begin{tcolorbox}
\begin{minipage}{0.95\linewidth}
\textbf{Rendu}

\medskip


Checkboxes are here: 
\end{minipage}
\end{tcolorbox}

\begin{Verbatim}[commandchars=\\\{\}]
Checkboxes are here: \PYG{n+na}{:checkbox:}\PYG{n+nv}{{}`Item 1,Item 2,Item 3,Item 4{}`}
\end{Verbatim}




\subsection{button}
\label{extensions/sphinxcontrib.webmocks:button}\begin{tcolorbox}
\begin{minipage}{0.95\linewidth}
\textbf{Rendu}

\medskip


Button is here: 
\end{minipage}
\end{tcolorbox}

\begin{Verbatim}[commandchars=\\\{\}]
Button is here: \PYG{n+na}{:button:}\PYG{n+nv}{{}`OK{}`}
\end{Verbatim}




\subsection{menulist}
\label{extensions/sphinxcontrib.webmocks:menulist}
Cette directive est utilisé pour le breadcrumb dans la directive \code{page}. (fonctionne pas)

\begin{Verbatim}[commandchars=\\\{\}]
\PYG{p}{..} \PYG{o+ow}{menulist}\PYG{p}{::}

    \PYG{l+m}{*} Menu1
        \PYG{l+m}{*} Sub\PYGZhy{}Menu 1\PYGZhy{}1
        \PYG{l+m}{*} Sub\PYGZhy{}Menu 1\PYGZhy{}2
        \PYG{l+m}{*} Sub\PYGZhy{}Menu 1\PYGZhy{}3
    \PYG{l+m}{*} Menu2
    \PYG{l+m}{*} Menu3
    \PYG{l+m}{*} Menu4
\end{Verbatim}




\subsection{page}
\label{extensions/sphinxcontrib.webmocks:page}

\subsubsection{create\_user}
\label{extensions/sphinxcontrib.webmocks:create-user}\begin{table}[H]
\centering

\begin{tabular}{|p{0.475\linewidth}|p{0.475\linewidth}|}
\hline
\begin{quote}\begin{description}
\item[{UserId}] \leavevmode


\item[{E-mail}] \leavevmode


\end{description}\end{quote}

 
\\
\hline\end{tabular}

\end{table}


\begin{Verbatim}[commandchars=\\\{\}]
\PYG{p}{..} \PYG{o+ow}{page}\PYG{p}{::} create\PYGZus{}user

    \PYG{n+nc}{:UserId:} \PYG{n+nf}{:text:{}`\PYGZus{}{}`}
    \PYG{n+nc}{:E\PYGZhy{}mail:} \PYG{n+nf}{:text:{}`\PYGZus{}{}`}

    \PYG{n+na}{:button:}\PYG{n+nv}{{}`OK{}`} \PYG{n+na}{:button:}\PYG{n+nv}{{}`Cancel{}`}
\end{Verbatim}


\subsubsection{create\_user2}
\label{extensions/sphinxcontrib.webmocks:create-user2}\begin{table}[H]
\centering

\begin{tabular}{|p{0.475\linewidth}|p{0.475\linewidth}|}
\hline

{\hyperref[extensions/sphinxcontrib.webmocks:]{\emph{\emph{Users}}}} (\autopageref*{extensions/sphinxcontrib.webmocks:}) \emph{\textgreater{}\textgreater{}} {\hyperref[extensions/sphinxcontrib.webmocks:]{\emph{\emph{Create User}}}} (\autopageref*{extensions/sphinxcontrib.webmocks:})
\begin{quote}\begin{description}
\item[{UserId}] \leavevmode


\item[{E-mail}] \leavevmode


\end{description}\end{quote}

 
\\
\hline\end{tabular}

\end{table}


\begin{Verbatim}[commandchars=\\\{\}]
\PYG{p}{..} \PYG{o+ow}{page}\PYG{p}{::} create\PYGZus{}user2
   \PYG{n+nc}{:breadcrumb:} \PYG{n+nf}{Users \PYGZgt{} Create User}

   \PYG{n+nc}{:UserId:} \PYG{n+nf}{:text:{}`\PYGZus{}{}`}
   \PYG{n+nc}{:E\PYGZhy{}mail:} \PYG{n+nf}{:text:{}`\PYGZus{}{}`}

   \PYG{n+na}{:button:}\PYG{n+nv}{{}`OK{}`} \PYG{n+na}{:button:}\PYG{n+nv}{{}`Cancel{}`}
\end{Verbatim}

\begin{Verbatim}[commandchars=\\\{\}]
\PYG{p}{..} \PYG{o+ow}{page}\PYG{p}{::} create\PYGZus{}user3
   \PYG{n+nc}{:breadcrumb:} \PYG{n+nf}{Users \PYGZgt{} Create User}
   \PYG{n+nc}{:desctable:}

   \PYG{n+nc}{:UserId:} \PYG{n+nf}{:text:{}`\PYGZus{} \PYGZlt{}required, description=Allows only ASCII chars\PYGZgt{}{}`}
   \PYG{n+nc}{:E\PYGZhy{}mail:} \PYG{n+nf}{:text:{}`\PYGZus{} \PYGZlt{}required\PYGZgt{}{}`}

   \PYG{n+na}{:button:}\PYG{n+nv}{{}`OK{}`} \PYG{n+na}{:button:}\PYG{n+nv}{{}`Cancel{}`}
\end{Verbatim}


\section{sphinxcontrib.yuml}
\label{extensions/sphinxcontrib.yuml::doc}\label{extensions/sphinxcontrib.yuml:sphinxcontrib-yuml}\begin{itemize}
\item {} 
\href{https://github.com/njouanin/sphinxcontrib-yuml}{https://github.com/njouanin/sphinxcontrib-yuml}

\end{itemize}


\chapter{Documentation commune pôle PHP}
\label{espace_polephp::doc}\label{espace_polephp:documentation-commune-pole-php}

\section{Présentation}
\label{espace_polephp:presentation}
L'espace documentaire du pôle PHP est accessible depuis l'URL : \href{http://docs.owsi-vm-polephp.accelance.net}{http://docs.owsi-vm-polephp.accelance.net}.

L'idée est de regrouper le maximum de documentations HTML / PDF / présentations générées via Sphinx.

Elles sont organisées par catégories :
\begin{itemize}
\item {} 
Outils

\item {} 
Clients

\item {} 
Clients archivés

\end{itemize}


\section{Contribution sur le dépôt}
\label{espace_polephp:contribution-sur-le-depot}
\begin{Verbatim}[commandchars=\\\{\}]
\PYG{n+nv}{\PYGZdl{} }git clone https://gitlab.openwide.fr/open\PYGZhy{}wide/docs\PYGZhy{}polephp.git
\end{Verbatim}


\subsection{Création d'une documentation client}
\label{espace_polephp:creation-d-une-documentation-client}
Exemple : Je souhaite créer des spécifications fonctionnelles pour la refonte du site de mon client \textbf{Ministère de l'Intérieur}.
\begin{itemize}
\item {} 
S'il n'existe pas déjà, il faut au préalable créer un dépôt Git spécifique \textbf{au client}

\item {} 
A la racine de ce dépôt, placer une image \code{/logo.png} relative au client

\item {} 
A la racine également, créer un fichier \code{/README} optionnel qui peut donner des détails sur le client / projets ...

\item {} 
Créer un dossier qui va contenir la documentation souhaitée, et nommer le dossier comme suit : \code{{[}AAAA{]}-{[}type\_projet{]}-{[}type\_de\_document{]}}, comme par exemple : \code{2015-Site\_ez-Specifications\_fonctionnelles}.

\item {} 
Dans le dépôt de documentation générale du pôle PHP, lier ce submodule nouvellement créé :

\end{itemize}

\begin{Verbatim}[commandchars=\\\{\}]
\PYG{n+nv}{\PYGZdl{} }git submodule add https://gitlab.openwide.fr/open\PYGZhy{}wide/\PYG{o}{[}XXX\PYGZus{}CLIENT\PYGZus{}DOCS\PYGZus{}XXX\PYG{o}{]}.git docs/clients/\PYG{o}{[}XXX\PYGZus{}CLIENT\PYGZus{}XXX\PYG{o}{]}
\end{Verbatim}


\subsection{Création d'une documentation outils}
\label{espace_polephp:creation-d-une-documentation-outils}
Les outils du pôle ne sont pas dans des dépôts séparés, donc il n'est pas nécessaire de faire un submodule.

Placer simplement la documentation dans le dossier \code{/outils/}.


\subsection{Affichage dans le menu}
\label{espace_polephp:affichage-dans-le-menu}
L'affichage des différents clients/outils n'est pas automatique, mais provient d'un fichier de configuration \code{/config.php} :

\begin{Verbatim}[commandchars=\\\{\}]
\PYG{c+cp}{\PYGZlt{}?php}
\PYG{n+nv}{\PYGZdl{}categories} \PYG{o}{=} \PYG{k}{array}\PYG{p}{(}
    \PYG{c+cm}{/*}
\PYG{c+cm}{    \PYGZsq{}GROUP\PYGZus{}LABEL\PYGZsq{} =\PYGZgt{} array(}
\PYG{c+cm}{        \PYGZsq{}doc\PYGZus{}folder\PYGZus{}name\PYGZsq{} =\PYGZgt{} \PYGZsq{}LABEL\PYGZsq{},}
\PYG{c+cm}{    ),}
\PYG{c+cm}{    */}
    \PYG{l+s+s1}{\PYGZsq{}Outils\PYGZsq{}} \PYG{o}{=\PYGZgt{}} \PYG{k}{array}\PYG{p}{(}
        \PYG{l+s+s1}{\PYGZsq{}outils/docker\PYGZsq{}} \PYG{o}{=\PYGZgt{}} \PYG{l+s+s1}{\PYGZsq{}Docker\PYGZsq{}}\PYG{p}{,}
        \PYG{l+s+s1}{\PYGZsq{}outils/drupal\PYGZus{}7\PYGZsq{}} \PYG{o}{=\PYGZgt{}} \PYG{l+s+s1}{\PYGZsq{}Drupal 7\PYGZsq{}}\PYG{p}{,}
        \PYG{l+s+s1}{\PYGZsq{}outils/sphinx\PYGZsq{}} \PYG{o}{=\PYGZgt{}} \PYG{l+s+s1}{\PYGZsq{}Sphinx\PYGZsq{}}\PYG{p}{,}
        \PYG{l+s+s1}{\PYGZsq{}outils/symfony\PYGZsq{}} \PYG{o}{=\PYGZgt{}} \PYG{l+s+s1}{\PYGZsq{}Symfony\PYGZsq{}}\PYG{p}{,}
        \PYG{c+c1}{// ...}
    \PYG{p}{),}
    \PYG{l+s+s1}{\PYGZsq{}Clients\PYGZsq{}} \PYG{o}{=\PYGZgt{}} \PYG{k}{array}\PYG{p}{(}
        \PYG{l+s+s1}{\PYGZsq{}clients/cfdt\PYGZhy{}f3c\PYGZsq{}} \PYG{o}{=\PYGZgt{}} \PYG{l+s+s1}{\PYGZsq{}CFDT F3C\PYGZsq{}}\PYG{p}{,}
        \PYG{c+c1}{// ...}
    \PYG{p}{),}
    \PYG{l+s+s1}{\PYGZsq{}Clients archivés\PYGZsq{}} \PYG{o}{=\PYGZgt{}} \PYG{k}{array}\PYG{p}{(}
        \PYG{c+c1}{// ...}
    \PYG{p}{),}
\PYG{p}{);}
\end{Verbatim}

\begin{thebibliography}{Ref}
\bibitem[Ref]{Ref}{\phantomsection\label{rtd/instruction_base/citation:ref} 
Book or article reference, URL or whatever.
}
\end{thebibliography}



\renewcommand{\indexname}{Index}
\printindex
\end{document}
